\begin{Question}{A.1.1}
Simplify the difference quotient for $z$.
\end{Question}
\begin{Question}{A.1.2}
Use the graph to find the slope of the secant line to $z$ between the points where $x=-1$ and $x=2$. Check your simplified difference quotient for $z$ by using it to find the slope of the same secant line.
\end{Question}
\begin{Question}{A.1.3}
Replace $x$ with $4$ in your difference quotient formula and simplify the result. Then copy \cref{ratesofchange:tab:z} onto your paper and fill in the missing values.
\end{Question}
\begin{Question}{A.1.4}
\label{ratesofchange:exert:hto0} As the value of $h$ gets closer to $0$, the values in the $y$ column of \cref{ratesofchange:tab:z} appear to be converging on a single number; what is this number?
\end{Question}
\begin{Question}{A.1.5}
The value found in \cref{ratesofchange:exert:hto0} is called \emph{the slope of the tangent line to $z$ at $4$}. Draw onto \cref{ratesofchange:fig:z} the line that passes through the point $(4,2)$ with this slope. The line you just drew is called \emph{the tangent line to $z$ at $4$}.
\end{Question}
