\documentclass[12pt]{article}

\usepackage[margin=2cm]{geometry}

\begin{document}

Earlier today I sent an email with the links to this material.

\begin{enumerate}
\item The print-ready PDF
\begin{enumerate}
\item This is the same content (99.5\%) as it has been. But numbering is different. So you may need to update your syllabi, etc., if they refer to problems by number.
\item A final draft will go to print center in July. I still need to finish supplementary exercises for Labs 4--9, the appendices with formulas and limit laws, and Steve is updating Lab 8.
\item I'll do some final tidying up on its presentation, and also try to minimize how many pages.
\item Phil is working on new cover art.
\item In July I'll send the new order number to all FDCs for Fall term book ordering.
\end{enumerate}
\item The PDF for on-screen
\begin{enumerate}
\item there will also be a PDF that is formatted differently, for optimal reading \emph{offline} on a tablet or laptop.
\item page numbering will likely be different. So you should be in the habit of referring to everything by section, not by page number. Especially because of\ldots
\end{enumerate}
\item The online version
\begin{enumerate}
\item Responsive design. Google rates this as ``Mobile Friendly''. Try it with your own smartphone.
\item Everything is web accessible according to standards. Except I have not yet put verbal descriptions in for all of the graphs. I'll do this once the other content is complete. Won't affect PDF versions.
\item Graphics, as in the Secant Line section, are .svg images. This means the browser renders the image based on primitive HTML5 code. The images will load faster than typical web images, and lose no resolution from zooming. 
\item Once I have a day to set aside, I'll make it so that an additional .png image can be copied/downloaded (for use in lab write-ups, etc.).
\item Power Rule section demonstrates several features:
\begin{enumerate}
\item Definition references and Example references as knowls. (References to Figures, Tables,  and Exercises will eventually behave this way too, but for now are links.)
\item At bottom, a WeBWorK interface. This will be removed for the live version for fall, but over the next year we'll add a few of these. (And we'll make the interface look nicer.)
\item At bottom, a GeoGebra applet. This one is being loaded directly from the huge library at GeoGebraTube. It's HTML5, and has passed a round of testing by DS. However we'll need to use things like this with care. A static, helpful version needs to go into the PDF. 
\end{enumerate}
\end{enumerate}
\item General
\begin{enumerate}
\item The fall versions will reside on the Math SAC's spot account.
\item Send typos and other editing suggestions to me.
\item I've only been copying the existing content. With this setup, it will be pretty easy to add or alter the content once you have suggestions. 
\item I think the online version will come to dominate. In your lab sessions, consider putting groups together such that every group has access. A recent survey of students from WeBWorK-using courses had 70 out of 100 using a smart phone. Adding laptops and tablets, it should be possible to always have all groups having internet access.
\item In July, I will put all the important information into a Spaces page for future referencing and send an update email.
\item If we move to OER for any of our major sequences, this style of production and management is what I would recommend. Here, I had all the content from Steve's previous draft, so I didn't have to think too hard about content. But I was able to make all this using MBX with about 8 hours of work per week since this term started. With a couple more people and bit of release time, we could either write or convert OER for 60/65/95, 20, 111/112, or 243/244. Improvements to MBX in general will trickle down to all MBX-based projects. For example, Chris Hughes and other MBX developers are working on the .EPUB output format, so you could read this (and all other MBX-based projects) on your Kindle, etc.
\end{enumerate}
\end{enumerate}

\end{document}