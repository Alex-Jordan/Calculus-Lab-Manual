\documentclass[12pt,border=2pt]{standalone}
\usepackage{amsmath,amssymb}

    
    \usepackage{pgfplots}
    \usepackage{xparse}
    \usepgfplotslibrary{patchplots}
    \usepackage{tkz-euclide}
    \usetkzobj{all}
    \usetikzlibrary{3d,calc}
    \usepackage{xltxtra}

    % curve, dot, and graph custom styles
    \pgfplotsset{pccplot/.style={color=red,mark=none,line width=1pt,<->,solid}} % primary style for curves
    \pgfplotsset{asymptote/.style={color=gray,mark=none,line width=1pt,<->,dashed}}
    \pgfplotsset{soldot/.style={color=red,only marks,mark=*}}
    \pgfplotsset{holdot/.style={color=red,fill=white,only marks,mark=*}}
    \pgfplotsset{blankgraph/.style={xmin=-10,xmax=10,ymin=-10,ymax=10,axis line style= {-, draw opacity=0 },axis lines=box,major tick length=0mm,xtick={-10,-9,...,10},ytick={-10,-9,...,10},grid=major,yticklabels={,,},xticklabels={,,},minor xtick=,minor ytick=,xlabel={},ylabel={},width=0.75\textwidth,grid style={solid,gray!40}}}
    
    % cycle list of plot styles for graphs with multiple plots
    \pgfplotscreateplotcyclelist{pccstylelist}{%
        pccplot\\%
        color=blue,mark=none,line width=1pt,<->,dashdotted\\%
        color=gray,mark=none,line width=1pt,<->,dashdotdotted\\%
    }

    \pgfplotsset{every axis/.append style={
        axis x line=middle,    % put the x axis in the middle
        axis y line=middle,    % put the y axis in the middle
        axis line style={<->}, % arrows on the axis
        xlabel={$x$},          % default put x on x-axis
        ylabel={$y$},          % default put y on y-axis
        xmin = -7,xmax = 7,    % most graphs have this window
        ymin = -7,ymax = 7,    % most graphs have this window
        xtick = {-6,-4,...,6},       % fix ticks 
        %extra x ticks={-6, -4, -2},
        %extra x tick labels={$-6\phantom{-}$, $-4\phantom{-}$, $-2\phantom{-}$},
        ytick = {-6,-4,...,6}, % fix ticks  
        yticklabel style={inner sep=0.333ex},
        minor xtick = {-7,-6,...,7}, % fix ticks
        minor ytick = {-7,-6,...,7}, % fix ticks
        scale only axis,       % don't factor in axis and tick labels for width= and height=
        cycle list name=pccstylelist,
        tick label style={font=\footnotesize},
        %label style={font=\scriptsize},
        legend cell align=left,
        %legend style={font=\scriptsize},
        width = 0.45\textwidth,
        grid = minor,
        grid style = {solid,gray!40},
        %every node near coord/.append style={
        %    font=\scriptsize
        %},
    }}

    %\tikzset{axisnode/.style={font=\scriptsize,text=black}}

    % arrow style
    \tikzset{>=stealth}

    % framing the graphs
    \pgfplotsset{framed/.style={axis background/.style ={draw=gray}}}
    % next line is a bit more colorful
    %\pgfplotsset{framed/.style={axis background/.style ={draw=gray,fill=yellow!20,rounded corners=3ex}}}

    
    
% These macros are automatically generated from the "macros"
% XML element.  Make permanent edits there.
%
%%%%%%%%%%%%%%%%%%%%%
%
%     Conveniences
%
%%%%%%%%%%%%%%%%%%%%%
%
%  Integers
%  Usage:  \Z
\newcommand{\Z}{\mathbb{Z}}
%
%  Real numbers, as set of scalars
%  Usage:  \reals
\newcommand{\reals}{\mathbb{R}}
%
%  n-space over real field
%  Usage: \complex{integer-dimension}
\newcommand{\real}[1]{\mathbb{R}^{#1}}
%
%  evaluate a function
%  Usage: \fe{function-name}{input}
\newcommand{\fe}[2]{#1\mathopen{}\left(#2\right)\mathclose{}}
%
%  closed interval
%  Usage: \cinterval{left-endpoint}{right-endpoint}
\newcommand{\cinterval}[2]{\left[#1,#2\right]}
%
%  open interval
%  Usage: \ointerval{left-endpoint}{right-endpoint}
\newcommand{\ointerval}[2]{\left(#1,#2\right)}
%
%  closed-open interval
%  Usage: \cointerval{left-endpoint}{right-endpoint}
\newcommand{\cointerval}[2]{\left[\left.#1,#2\right)\right.}
%
%  open-closed interval
%  Usage: \ocinterval{left-endpoint}{right-endpoint}
\newcommand{\ocinterval}[2]{\left(\left.#1,#2\right]\right.}
%
%  point
%  Usage: \point{x}{y}
\newcommand{\point}[2]{\left(#1,#2\right)}
%
%  first derivative
%  Usage: \fd{f}
\newcommand{\fd}[1]{#1'}
%
%  second derivative
%  Usage: \sd{f}
\newcommand{\sd}[1]{#1''}
%
%  third derivative
%  Usage: \td{f}
\newcommand{\td}[1]{#1'''}
%
%  Leibniz notation
%  Usage: \lz{y}{x}
\newcommand{\lz}[2]{\frac{d#1}{d#2}}
%
%  higher Leibniz notation
%  Usage: \lzn{n}{y}{x}
\newcommand{\lzn}[3]{\frac{d^{#1}#2}{d#3^{#1}}}
%
%  Leibniz operator
%  Usage: \lzo{x}
\newcommand{\lzo}[1]{\frac{d}{d#1}}
%
%  Leibniz operator on ....
%  Usage: \lzoo{x}{y}
\newcommand{\lzoo}[2]{{\frac{d}{d#1}}{\left(#2\right)}}
%
%  higher Leibniz operator
%  Usage: \lzon{n}{x}{y}
\newcommand{\lzon}[2]{\frac{d^{#1}}{d#2^{#1}}}
%
%  Leibniz operator at ....
%  Usage: \lzoa{y}{x}{a}
\newcommand{\lzoa}[3]{\left.{\frac{d#1}{d#2}}\right|_{#3}}
%
%  Absolute Value
%  Usage: \abs{x}
\newcommand{\abs}[1]{\left|#1\right|}
%
%  sech
%  Usage: \sech
\newcommand{\sech}{\operatorname{sech}}
%
%  csch
%  Usage: \csch
\newcommand{\csch}{\operatorname{csch}}
%



    \begin{document}

                            \begin{tikzpicture}[
                                         declare function={
                                             c(\x)= cos(\x^2*180/3.14159265359);
                                             sp(\x)=sqrt(4+\x^2*c(\x));
                                             sm(\x)=sqrt(4-\x^2*c(\x));
                                             m(\x)= 2+sm(\x);
                                             n(\x)= 2-sm(\x);
                                             p(\x)= 2+sp(\x);
                                             q(\x)= 2-sp(\x);
                                             f(\x)= \x^2/m(\x);
                                             g(\x)= m(\x)/c(\x);
                                             h(\x)= -\x^2/p(\x);
                                             k(\x)= p(\x)/c(\x);
                                         }]
                                     \begin{axis}[
                                         xmin=-7,
                                         xmax=7,
                                         ymin=-4,
                                         ymax=10.6,
                                         ytick={-4,-2,...,10},
                                         minor ytick={-4,-1,...,10},
                                         variable=v,
                                     ]
                                     \addplot[pccplot,domain=0:2.2,-] ({f(v)},{m(v)});
                                     \addplot[pccplot,domain=2.2:2.3499859792,-] ({f(v)},{m(v)});
                                     \addplot[pccplot,domain=2.6965731707:2.72,-] ({f(v)},{m(v)});
                                     \addplot[pccplot,domain=2.72:3.1,-] ({f(v)},{m(v)});
                                     \addplot[pccplot,domain=3.1:3.36,-] ({f(v)},{m(v)});              
                                     \addplot[pccplot,domain=3.36:3.369,-] ({f(v)},{m(v)});              
                                     \addplot[pccplot,domain=3.7207600255:3.75,<-] ({f(v)},{m(v)});
                                     \addplot[pccplot,domain=3.75:4.1,-] ({f(v)},{m(v)});
                                     \addplot[pccplot,domain=4.1:4.1826819343,->] ({f(v)},{m(v)});
                                     \addplot[pccplot,domain=4.5013070528:4.6,<-] ({f(v)},{m(v)});
                                     \addplot[pccplot,domain=4.6:4.8,-] ({f(v)},{m(v)});
                                     \addplot[pccplot,domain=4.8:4.8632109381,->] ({f(v)},{m(v)});
                                     \addplot[pccplot,domain=5.1649321939:5.2,<-] ({f(v)},{m(v)});
                                     \addplot[pccplot,domain=5.2:5.4,-] ({f(v)},{m(v)});
                                     \addplot[pccplot,domain=5.4:5.4596904128,->] ({f(v)},{m(v)});
                                     \addplot[pccplot,domain=5.7524607477:5.9970858912] ({f(v)},{m(v)});
                                     \addplot[pccplot,domain=6.2854560605:6.4899313929] ({f(v)},{m(v)});
                                     \addplot[pccplot,domain=6.7769950285:6.9475348363] ({f(v)},{m(v)});
                                     \addplot[pccplot,domain=7.2357791015:7.3762238487] ({f(v)},{m(v)});
                                     \addplot[pccplot,domain=7.6680608384:7.7804196739] ({f(v)},{m(v)});
                                     \addplot[pccplot,domain=8.0787873209:8.1630812697] ({f(v)},{m(v)});
                                     \addplot[pccplot,domain=8.4729783943:8.525239357] ({f(v)},{m(v)});
                                    %
                                     \addplot[pccplot,mark=none,-] coordinates {({f(2.3499859792)},{m(2.3499859792)}) ({g(2.3499859793)},{n(2.3499859793)})};
                                     \addplot[pccplot,domain=2.2795725971:2.3499859793,<-] ({g(v)},{n(v)});
                                     \addplot[pccplot,mark=none,-] coordinates {({f(2.6965731707)},{m(2.6965731707)}) ({g(2.696573171)},{n(2.696573171)})};
                                     \addplot[pccplot,domain=2.696573171:2.7240684007,->] ({g(v)},{n(v)});
                                     \addplot[pccplot,mark=none,-] coordinates {({f(3.369)},{m(3.369)}) ({g(3.3697995683)},{n(3.3697995683)})};
                                     \addplot[pccplot,domain=3.36795:3.3697995683,<-] ({g(v)},{n(v)});
                                     \addplot[pccplot,domain=0.15:0.9934229315,->] ({g(v)},{n(v)});
                                     \addplot[pccplot,domain=1.4952570458:2.0001872665,<->] ({g(v)},{n(v)});
                                     \addplot[pccplot,domain=2.9496641015:3.178782971,<->] ({g(v)},{n(v)});
                                     \addplot[pccplot,domain=3.9009258917:4.0194955921,<->] ({g(v)},{n(v)});
                                    %
                                     \addplot[pccplot,domain=0:1.8,-] ({h(v)},{p(v)});
                                     \addplot[pccplot,domain=1.8:2.2,-] ({h(v)},{p(v)});
                                     \addplot[pccplot,domain=2.2:2.8,-] ({h(v)},{p(v)});
                                     \addplot[pccplot,domain=2.8:2.89,-] ({h(v)},{p(v)});
                                     \addplot[pccplot,domain=3.2571424315:3.3,-] ({h(v)},{p(v)});
                                     \addplot[pccplot,domain=3.3:3.75,-] ({h(v)},{p(v)});
                                     \addplot[pccplot,domain=3.75:3.7971078497,->] ({h(v)},{p(v)});
                                     \addplot[pccplot,domain=4.1295834659:4.2,<-] ({h(v)},{p(v)});
                                     \addplot[pccplot,domain=4.2:4.5,-] ({h(v)},{p(v)});
                                     \addplot[pccplot,domain=4.5:4.5356939968,->] ({h(v)},{p(v)});
                                     \addplot[pccplot,domain=4.844517562:4.9,<-] ({h(v)},{p(v)});
                                     \addplot[pccplot,domain=4.9:5,-] ({h(v)},{p(v)});
                                     \addplot[pccplot,domain=5:5.170055092,->] ({h(v)},{p(v)});
                                     \addplot[pccplot,domain=5.4665876095:5.5,<-] ({h(v)},{p(v)});
                                     \addplot[pccplot,domain=5.5:5.7,-] ({h(v)},{p(v)});
                                     \addplot[pccplot,domain=5.7:5.7347036241,->] ({h(v)},{p(v)});
                                     \addplot[pccplot,domain=6.0248230446:6.2484075213,<->] ({h(v)},{p(v)});
                                     \addplot[pccplot,domain=6.5357897386:6.7226893127,<->] ({h(v)},{p(v)});
                                     \addplot[pccplot,domain=7.0100484639:7.1651804638,<->] ({h(v)},{p(v)});
                                     \addplot[pccplot,domain=7.4549045114:7.5811656127,<->] ({h(v)},{p(v)});
                                     \addplot[pccplot,domain=7.8758321028:7.9743147713,<->] ({h(v)},{p(v)});
                                     \addplot[pccplot,domain=8.2775481783:8.3468023025,<->] ({h(v)},{p(v)});
                                     \addplot[pccplot,domain=8.6671994161:8.6968286602,<->] ({h(v)},{p(v)});
                                    %
                                     \addplot[pccplot,mark=none,-] coordinates {({g(0.15)},{n(0.15)}) ({k(0.15)},{q(0.15)})};
                                     \addplot[pccplot,domain=0.15:0.9690222774,->] ({k(v)},{q(v)});
                                     \addplot[pccplot,domain=2.3375568425:2.6564493594,<->] ({k(v)},{q(v)});
                                     \addplot[pccplot,domain=3.4567020839:3.6249982793,<->] ({k(v)},{q(v)});
                                     \addplot[pccplot,domain=4.3047344897:4.3734360266,<->] ({k(v)},{q(v)});
                                    %
                                     \addplot[pccplot,mark=none,-] coordinates {({h(3.2571424315)},{p(3.2571424315)}) ({k(3.2571424316)},{q(3.2571424316)})}; 
                                    %
                                    \addplot[pccplot,domain=3.2571424316:3.2608706576,->] ({k(v)},{q(v)});
                                    \addplot[pccplot,mark=none,-] coordinates {({h(2.89)},{p(2.89)}) ({k(2.89)},{q(2.89)})};
                                    \addplot[pccplot,domain=2.8754877926:2.89,<-] ({k(v)},{q(v)});
                                    \addplot[pccplot,domain=1.4584452866:2.0506909475,<->] ({k(v)},{q(v)});
                                \end{axis}
                            \end{tikzpicture}
                            \end{document}
