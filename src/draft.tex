%%                                    %%
%% Generated from MathBook XML source %%
%%    on 2015-05-29T13:44:02-07:00    %%
%%                                    %%
%%   http://mathbook.pugetsound.edu   %%
%%                                    %%
\documentclass[12pt,]{book}
%% Load geometry package to allow page margin adjustments
\usepackage{geometry}
\geometry{letterpaper,total={5.0in,9.0in}}
%% Custom Preamble Entries, early (use latex.preamble.early)

\usepackage{titlesec}
\usepackage{fancyhdr}
\usepackage{longtable}

%% Inline math delimiters, \(, \), made robust with next package
\usepackage{fixltx2e}
%% Page Layout Adjustments (latex.geometry)
\geometry{letterpaper,total={6.25in,9.0in}}
%% For unicode character support, use the "xelatex" executable
%% If never using xelatex, the next three lines can be removed
\usepackage{ifxetex}
\ifxetex\usepackage{xltxtra}\fi
%% Symbols, align environment, bracket-matrix
\usepackage{amsmath}
\usepackage{amssymb}
%% allow more columns to a matrix
%% can make this even bigger by overiding with  latex.preamble.late  processing option
\setcounter{MaxMatrixCols}{30}
%% XML, MathJax Conflict Macros
%% Two nonstandard macros that MathJax supports automatically
%% so we always define them in order to allow their use and
%% maintain source level compatibility
%% This avoids using two XML entities in source mathematics
\newcommand{\lt}{<}
\newcommand{\gt}{>}
%% xfrac package for 'beveled fractions': http://tex.stackexchange.com/questions/3372/how-do-i-typeset-arbitrary-fractions-like-the-standard-symbol-for-5-%C2%BD
\usepackage{xfrac}
%% Semantic Macros
%% To preserve meaning in a LaTeX file
%% Only defined here if required in this document
%% Used for units and number formatting
\usepackage[per-mode=fraction]{siunitx}
\ifxetex\sisetup{math-micro=\text{µ},text-micro=µ}\fi%% Common non-SI units
\DeclareSIUnit\degreeFahrenheit{\SIUnitSymbolDegree{F}}
\DeclareSIUnit\fahrenheit{\degreeFahrenheit}
\DeclareSIUnit\pound{lb}
\DeclareSIUnit\foot{ft}
\DeclareSIUnit\inch{in}
\DeclareSIUnit\yard{yd}
\DeclareSIUnit\mile{mi}
\DeclareSIUnit\mileperhour{mph}
\DeclareSIUnit\gallon{gal}
%% Subdivision Numbering, Chapters, Sections, Subsections, etc
%% Subdivision numbers may be turned off at some level ("depth")
%% A section *always* has depth 1, contrary to us counting from the document root
%% The latex default is 3.  If a larger number is present here, then
%% removing this command may make some cross-references ambiguous
%% The precursor variable $numbering-maxlevel is checked for consistency in the common XSL file
\setcounter{secnumdepth}{3}
%% Environments with amsthm package
%% Theorem-like enviroments in "plain" style, with or without proof
\usepackage{amsthm}
\theoremstyle{plain}
%% Numbering for Theorems, Conjectures, Examples, Figures, etc
%% Controlled by  numbering.theorems.level  processing parameter
%% Always need a theorem environment to set base numbering scheme
%% even if document has no theorems (but has other environments)
\newtheorem{theorem}{Theorem}[section]
\renewcommand*{\proofname}{Proof}%% Only variants actually used in document appear here
%% Numbering: all theorem-like numbered consecutively
%% i.e. Corollary 4.3 follows Theorem 4.2
%% Definition-like environments, normal text
%% Numbering for definition, examples is in sync with theorems, etc
%% also for free-form exercises, not in exercise sections
\theoremstyle{definition}
\newtheorem{exercise}[theorem]{Exercise}
%% For improved tables
\usepackage{array}
%% Some extra height on each row is desirable, especially with horizontal rules
%% Increment determined experimentally
\setlength{\extrarowheight}{0.2ex}
%% Define variable thickness horizontal rules, full and partial
%% Thicknesses are 0.03, 0.05, 0.08 in the  booktabs  package
\makeatletter
\newcommand{\hrulethin}  {\noalign{\hrule height 0.04em}}
\newcommand{\hrulemedium}{\noalign{\hrule height 0.07em}}
\newcommand{\hrulethick} {\noalign{\hrule height 0.11em}}
%% We preserve a copy of the \setlength package before other
%% packages (extpfeil) get a change to load packages that redefine it
\let\oldsetlength\setlength
\newlength{\Oldarrayrulewidth}
\newcommand{\crulethin}[1]%
{\noalign{\global\oldsetlength{\Oldarrayrulewidth}{\arrayrulewidth}}%
\noalign{\global\oldsetlength{\arrayrulewidth}{0.04em}}\cline{#1}%
\noalign{\global\oldsetlength{\arrayrulewidth}{\Oldarrayrulewidth}}}%
\newcommand{\crulemedium}[1]%
{\noalign{\global\oldsetlength{\Oldarrayrulewidth}{\arrayrulewidth}}%
\noalign{\global\oldsetlength{\arrayrulewidth}{0.07em}}\cline{#1}%
\noalign{\global\oldsetlength{\arrayrulewidth}{\Oldarrayrulewidth}}}
\newcommand{\crulethick}[1]%
{\noalign{\global\oldsetlength{\Oldarrayrulewidth}{\arrayrulewidth}}%
\noalign{\global\oldsetlength{\arrayrulewidth}{0.11em}}\cline{#1}%
\noalign{\global\oldsetlength{\arrayrulewidth}{\Oldarrayrulewidth}}}
%% Single letter column specifiers defined via array package
\newcolumntype{A}{!{\vrule width 0.04em}}
\newcolumntype{B}{!{\vrule width 0.07em}}
\newcolumntype{C}{!{\vrule width 0.11em}}
\makeatother
%% Figures, Tables, Floats
%% The [H]ere option of the float package fixes floats in-place,
%% in deference to web usage, where floats are totally irrelevant
%% We redefine the figure and table environments, if used
%%   1) New mbxfigure and/or mbxtable environments are defined with float package
%%   2) Standard LaTeX environments redefined to use new environments
%%   3) Standard LaTeX environments redefined to step theorem counter
%%   4) Counter for new enviroments is set to the theorem counter before caption
%% You can remove all this figure/table setup, to restore standard LaTeX behavior
%% HOWEVER, numbering of figures/tables AND theorems/examples/remarks, etc
%% WILL ALL de-synchronize with the numbering in the HTML version
%% You can remove the [H] argument of the \newfloat command, to allow flotation and 
%% preserve numbering, BUT the numbering may then appear "out-of-order"
\usepackage{float}
\usepackage[bf]{caption} % http://tex.stackexchange.com/questions/95631/defining-a-new-type-of-floating-environment 
\usepackage{newfloat}
\usepackage{subcaption}
\captionsetup[subfigure]{labelformat=simple}
\captionsetup[subtable]{labelformat=simple}
\renewcommand\thesubfigure{(\alph{subfigure})}
\makeatletter
% we plan to use subtables within figure environments, so they need to reset accordingly
\@addtoreset{subtable}{figure}
\makeatother
% Side-by-side elements need careful treatement for aligning captions, see: 
% http://tex.stackexchange.com/questions/230335/vertically-aligning-minipages-subfigures-and-subtables-not-with-baseline 
\usepackage{stackengine,ifthen}
\newcounter{figstack}
\newcounter{figindex}
\newlength\fight
\newcommand\pushValignCaptionBottom[5][b]{%
\stepcounter{figstack}%
\expandafter\def\csname %
figalign\romannumeral\value{figstack}\endcsname{#1}%
\expandafter\def\csname %
figtype\romannumeral\value{figstack}\endcsname{#2}%
\expandafter\def\csname %
figwd\romannumeral\value{figstack}\endcsname{#3}%
\expandafter\def\csname %
figcontent\romannumeral\value{figstack}\endcsname{#4}%
\expandafter\def\csname %
figcap\romannumeral\value{figstack}\endcsname{#5}%
\setbox0=\hbox{%
\begin{#2}{#3}#4\end{#2}}%
\ifdim\dimexpr\ht0+\dp0\relax>\fight\global\setlength{\fight}{%
\dimexpr\ht0+\dp0\relax}\fi%
}
\newcommand\popValignCaptionBottom{%
\setcounter{figindex}{0}%
\hfill%
\whiledo{\value{figindex}<\value{figstack}}{%
\stepcounter{figindex}%
\def\tmp{\csname figwd\romannumeral\value{figindex}\endcsname}%
\begin{\csname figtype\romannumeral\value{figindex}\endcsname}[t]{\tmp}%
\centering%
\stackinset{c}{}%
{\csname figalign\romannumeral\value{figindex}\endcsname}{}%
{\csname figcontent\romannumeral\value{figindex}\endcsname}%
{\rule{0pt}{\fight}}\par%
\csname figcap\romannumeral\value{figindex}\endcsname%
\end{\csname figtype\romannumeral\value{figindex}\endcsname}%
\hfill%
}%
\setcounter{figstack}{0}%
\setlength{\fight}{0pt}%
\hfill%
}
% Figure environment setup so that it no longer floats
\SetupFloatingEnvironment{figure}{fileext=lof,placement={H},within=section,name=Figure}
% figures have the same number as theorems: http://tex.stackexchange.com/questions/16195/how-to-make-equations-figures-and-theorems-use-the-same-numbering-scheme 
\makeatletter
\let\c@figure\c@theorem
\makeatother
% Table environment setup so that it no longer floats
\SetupFloatingEnvironment{table}{fileext=lot,placement={H},within=section,name=Table}
% tables have the same number as theorems: http://tex.stackexchange.com/questions/16195/how-to-make-equations-figures-and-theorems-use-the-same-numbering-scheme 
\makeatletter
\let\c@table\c@theorem
\makeatother
%% Raster graphics inclusion, wrapped figures in paragraphs
\usepackage{graphicx}
%% Colors for Sage boxes and author tools (red hilites)
\usepackage[usenames,dvipsnames,svgnames,table]{xcolor}
%% Multiple column, column-major lists
\usepackage{multicol}
%% More flexible list management, esp. for references and exercises
%% But also for specifying labels (ie custom order) on nested lists
\usepackage{enumitem}
%% Lists of exercises in their own section, maximum depth 4
\newlist{exerciselist}{description}{4}
\setlist[exerciselist]{leftmargin=0em,itemsep=-1.0ex,topsep=1.0ex,partopsep=0pt,parsep=0pt}
\newenvironment{exercisegroup}%
{\medskip\noindent}%
{\par\bigskip}%
\usepackage{changepage}%
\newlength{\exercisegroupindent}%
\setlength{\exercisegroupindent}{2em}%
\newlength{\exercisegroupitemwidth}%
\newenvironment{exercisegrouplist}%
{\vspace{-\partopsep}%
\begin{adjustwidth}{\exercisegroupindent}{0em}}%
{\end{adjustwidth}%
\vspace{-\partopsep}%
\vspace{\baselineskip}}%
\newenvironment{exercisegroupbyrow}[1]%
{\begin{exercisegrouplist}%
\setlength{\parindent}{0em}%
\setlength{\exercisegroupitemwidth}{\linewidth}%
\addtolength{\exercisegroupitemwidth}{\columnsep}%
\divide\exercisegroupitemwidth by #1%
\addtolength{\exercisegroupitemwidth}{-\columnsep}}%
{\end{exercisegrouplist}}%
%% To allow for multicols to just have one column
%% http://tex.stackexchange.com/questions/233866/one-column-multicol-environment#answer-233904
\usepackage{xparse}%
\let\multicolmulticols\multicols%
\let\endmulticolmulticols\endmulticols%
\RenewDocumentEnvironment{multicols}{mO{}}%
 {%
  \ifnum#1=1%
    \vspace{\multicolsep}#2%
  \else % More than 1 column%
    \multicolmulticols{#1}[#2]%
  \fi%
 }%
 {%
  \ifnum#1=1%
    \vspace{\multicolsep}%
  \else % More than 1 column%
    \endmulticolmulticols%
  \fi%
 }%
\newenvironment{exercisegroupbycol}[1]%
{\begin{exercisegrouplist}%
\vspace{-\multicolsep}%
\begin{multicols}{#1}%
\setlength{\parindent}{0em}%
\setlength{\exercisegroupitemwidth}{\linewidth}}%
{\end{multicols}%
\vspace{-\multicolsep}%
\end{exercisegrouplist}}%
\setlength{\fboxsep}{0pt}%
\newenvironment{exercisegroupitem}[1]%
{\begin{minipage}[t]{\exercisegroupitemwidth}
\vspace{0pt}%
{\bfseries#1}%
\rule{0pt}{\baselineskip}}{\strut%
\end{minipage}%
\hspace{\columnsep}}%
%% hyperref driver does not need to be specified
\usepackage{hyperref}
%% Hyperlinking active in PDFs, all links solid and blue
\hypersetup{colorlinks=true,linkcolor=blue,citecolor=blue,filecolor=blue,urlcolor=blue}
\hypersetup{pdftitle={Portland Community College MTH 251 Lab Manual}}
%% If you manually remove hyperref, leave in this next command
\providecommand\phantomsection{}
%% Graphics Preamble Entries
\usepackage{pgfplots}
\usepackage{xparse}

% cycle list- truly awesome; see section 4.6.7, pg 129 of pgfplots
\pgfplotscreateplotcyclelist{pccstylelist}{%
    color=red,mark=none,line width=1pt,<->,solid\\%
    color=blue,mark=none,line width=1pt,<->,dashdotted\\%
    color=gray,mark=none,line width=1pt,<->,dashdotdotted\\%
}

\pgfplotsset{every axis/.append style={
    axis x line=middle,    % put the x axis in the middle
    axis y line=middle,    % put the y axis in the middle
    axis line style={<->}, % arrows on the axis
    xlabel={$x$},          % default put x on x-axis
    ylabel={$y$},          % default put y on y-axis
    xmin = -7,
    xmax = 7,
    ymin = -7,
    ymax = 7,
    xtick = {-6,-4,...,6},
    ytick = {-6,-4,...,6},
    minor xtick = {-7,-6,...,7},
    minor ytick = {-7,-6,...,7},
    scale only axis,       % otherwise width won't be as intended: http://tex.stackexchange.com/questions/36297/pgfplots-how-can-i-scale-to-text-width
    cycle list name=pccstylelist,
    %tick label style={font=\small},
    %label style={font=\small},
    legend cell align=left,
    %legend style={font=\tiny},
    width=0.4\textwidth,
    grid=minor,
    every node near coord/.append style={
        %font=\small
    },
}}

%\tikzset{axisnode/.style={font=\tiny,text=black}}

% line style
\pgfplotsset{pccplot/.style={color=red,mark=none,line width=1pt,<->}} % this is pretty redundant in most cases now that cycle list is implemented
\pgfplotsset{asymptote/.style={color=gray,mark=none,line width=1pt,<->,dashed}}
\pgfplotsset{soldot/.style={color=red,only marks,mark=*}}
\pgfplotsset{holdot/.style={color=red,fill=white,only marks,mark=*}}
\pgfplotsset{blankgraph/.style={xmin=-10,xmax=10,ymin=-10,ymax=10,axis line style= {-, draw opacity=0 },axis lines=box,major tick length=0mm,xtick={-10,-9,...,10},ytick={-10,-9,...,10},grid=major,yticklabels={,,},xticklabels={,,},minor xtick=,minor ytick=,xlabel={},ylabel={}}}


% arrow style
\tikzset{>=stealth}

% framing the graphs
\pgfplotsset{framed/.style={axis background/.style ={draw=gray}}}
% next line is a bit more colourful
%\pgfplotsset{framed/.style={axis background/.style ={draw=gray,fill=yellow!20,rounded corners=3ex}}}

% grid style
\pgfplotsset{grid style={dotted,gray!90}}

% for annotating equations
\newcommand{\tikzmark}[1]{\tikz[overlay,remember picture] \node[baseline] (#1) {};}%
\NewDocumentCommand{\Annotate}{O{} O{} m m m m m m}{%
    % #1 = line draw options
    % #2 = node options
    % #3 = x-coordinate displacement start
    % #4 = y-coordinate displacement start
    % #5 = x-coordinate displacement end
    % #6 = y-coordinate displacement end
    % #7 = tikzmark name
    % #8 = node text
    % Example:
    % X =\tikzmark{Point A}  Y
    % \Annotate[][anchor=south west]{-1.5ex}{1.5ex}{0.7}{0.5}{Point A}{Always true}
\begin{tikzpicture}[remember picture, overlay]
    \draw [thick, -latex, shorten >=1pt, #1]
        (#7) ++ (#3,#4) -- +(#5,#6)
        node [black, right, draw=black, #2] {#8};
\end{tikzpicture}%
}%
%% extpfeil package for certain extensible arrows,
%% as also provided by MathJax extension of the same name
%% NB: this package loads mtools, which loads calc, which redefines
%%     \setlength, so it can be removed if it seems to be in the 
%%     way and your math does not use:
%%     
%%     \xtwoheadrightarrow, \xtwoheadleftarrow, \xmapsto, \xlongequal, \xtofrom
%%     
%%     we have had to be extra careful with variable thickness
%%     lines in tables, and so also load this package late
\usepackage{extpfeil}
%% Custom Preamble Entries, late (use latex.preamble.late)


%%%%%%%%%%%%%%%%%%%%%%%%%%%%%%%%%%%%%%%%%%%%%%%%%%%%%%
% Rename "Chapter" to "Lab"
%%%%%%%%%%%%%%%%%%%%%%%%%%%%%%%%%%%%%%%%%%%%%%%%%%%%%%
\renewcommand{\chaptername}{Lab}
%%%%%%%%%%%%%%%%%%%%%%%%%%%%%%%%%%%%%%%%%%%%%%%%%%%%%%
% Header and Footer
%%%%%%%%%%%%%%%%%%%%%%%%%%%%%%%%%%%%%%%%%%%%%%%%%%%%%%
\renewcommand{\sectionmark}[1]{%
 \markright{\slshape\MakeUppercase{%
 Activity \thesection.%
 \ #1}}}%

%%%%%%%%%%%%%%%%%%%%%%%%%%%%%%%%%%%%%%%%%%%%%%%%%%%%%%
\allowdisplaybreaks


%% Convenience macros
% These macros are automatically generated from the "macros"
% XML element.  Make permanent edits there.
%
%%%%%%%%%%%%%%%%%%%%%
%
%     Conveniences
%
%%%%%%%%%%%%%%%%%%%%%
%
%  Integers
%  Usage:  \Z
\newcommand{\Z}{\mathbb{Z}}
%
%  Real numbers, as set of scalars
%  Usage:  \reals
\newcommand{\reals}{\mathbb{R}}
%
%  n-space over real field
%  Usage: \complex{integer-dimension}
\newcommand{\real}[1]{\mathbb{R}^{#1}}
%
%  evaluate a function
%  Usage: \fe{function-name}{input}
\newcommand{\fe}[2]{\mathop{{#1}{\left(#2\right)}}}
%
%  closed interval
%  Usage: \cinterval{left-endpoint}{right-endpoint}
\newcommand{\cinterval}[2]{\left[#1,#2\right]}
%
%  open interval
%  Usage: \ointerval{left-endpoint}{right-endpoint}
\newcommand{\ointerval}[2]{\left(#1,#2\right)}
%
%  closed-open interval
%  Usage: \cointerval{left-endpoint}{right-endpoint}
\newcommand{\cointerval}[2]{\left[\left.#1,#2\right)\right.}
%
%  open-closed interval
%  Usage: \ocinterval{left-endpoint}{right-endpoint}
\newcommand{\ocinterval}[2]{\left(\left.#1,#2\right]\right.}
%
%  point
%  Usage: \point{x}{y}
\newcommand{\point}[2]{\left(#1,#2\right)}
%
%  first derivative
%  Usage: \fd{f}
\newcommand{\fd}[1]{#1'}
%
%  second derivative
%  Usage: \sd{f}
\newcommand{\sd}[1]{#1''}
%
%  third derivative
%  Usage: \td{f}
\newcommand{\td}[1]{#1'''}
%
%  Leibniz notation
%  Usage: \lz{y}{x}
\newcommand{\lz}[2]{\frac{d#1}{d#2}}
%
%  higher Leibniz notation
%  Usage: \lzn{n}{y}{x}
\newcommand{\lzn}[3]{\frac{d^{#1}#2}{d#3^{#1}}}
%
%  Leibniz operator
%  Usage: \lzo{x}
\newcommand{\lzo}[1]{\frac{d}{d#1}}
%
%  Leibniz operator on ....
%  Usage: \lzoo{x}{y}
\newcommand{\lzoo}[2]{{\frac{d}{d#1}}{\left(#2\right)}}
%
%  higher Leibniz operator
%  Usage: \lzon{x}{y}{n}
\newcommand{\lzon}[2]{\frac{d^{#1}}{d#2^{#1}}}
%
%  Leibniz operator at ....
%  Usage: \lzoa{y}{x}{a}
\newcommand{\lzoa}[3]{\left.{\frac{d#1}{d#2}}\right|_{#3}}
%
%  Absolute Value
%  Usage: \abs{x}
\newcommand{\abs}[1]{\left|#1\right|}
%
%% Title page information for book
\title{Portland Community College MTH 251 Lab Manual}
\author{}
\date{}
\begin{document}
\frontmatter
%% begin: half-title
\thispagestyle{empty}
{\centering
\vspace*{0.28\textheight}
{\Huge Portland Community College MTH 251 Lab Manual}}
\clearpage
%% end:   half-title
%% begin: adcard
\thispagestyle{empty}
\null%
\clearpage
%% end:   adcard
%% begin: title page
%% Inspired by Peter Wilson's "titleDB" in "titlepages" CTAN package
\thispagestyle{empty}
{\centering
\vspace*{0.14\textheight}
{\Huge Portland Community College MTH 251 Lab Manual}\\[2\baselineskip]
}
\clearpage
%% end:   title page
%% begin: copyright-page
\thispagestyle{empty}
\vspace*{\stretch{2}}
\vspace*{\stretch{1}}
\null\clearpage
%% end:   copyright-page
%% begin: acknowledgements
%% end:   acknowledgements
%% begin: forewords
%% end:   forewords
%% begin: prefaces
%% end:   prefaces
%% begin: table of contents
\setcounter{tocdepth}{1}
\renewcommand*\contentsname{Contents}
\tableofcontents
%% end:   table of contents
\mainmatter
\typeout{************************************************}
\typeout{Chapter 1 Functions, Derivatives, and Antiderivatives}
\typeout{************************************************}
\chapter[Functions, Derivatives, and Antiderivatives]{Functions, Derivatives, and Antiderivatives}\label{chapter-functions-derivatives-antiderivatives}
\typeout{************************************************}
\typeout{Section 1.1 Graphical Features from Derivatives}
\typeout{************************************************}
\section[Graphical Features from Derivatives]{Graphical Features from Derivatives}\label{section-graphical-features-from-derivatives}
When given a function formula, we  often find the first and second derivative formulas to determine behaviors of the given function.  In a later lab we will use the first and second derivative formulas to help us graph a function given the formula for the function.  One thing we do with the derivative formulas is determine where they are positive, negative, zero, and undefined. This helps us determine where the given function is increasing, decreasing, concave up, concave down, and linear.%
\typeout{************************************************}
\typeout{Exercises}
\typeout{************************************************}
\section*{Exercises}\label{exercises-1}

\begin{exercisegroup}%
In the next few exercises, you will construct a function and then answer questions about it.%
\begin{figure}
\centering
\pushValignCaptionBottom[b]{minipage}{.60\textwidth}{%
\pgfplotsset{every axis/.append style={width=\linewidth}}%
\centering% horizontal alignment 
\begin{tabular}{ccc}\hrulethick
Interval&\(\fd{f}\)&\(\sd{f}\)\\\hrulemedium
\(\ointerval{-\infty}{-1}\)&Positive&Negative\\
\(\ointerval{-1}{\infty}\)&Positive&Positive
\end{tabular}
}% end body 
{\captionof{table}{Signs on \(\fd{f}\) and \(\sd{f}\)\label{table-based-on-derivatives}}
}% caption 
\pushValignCaptionBottom[b]{minipage}{.40\textwidth}{%
\pgfplotsset{every axis/.append style={width=\linewidth}}%
\centering% horizontal alignment 
{
\begin{tikzpicture}
\begin{axis}[]
\end{axis}
\end{tikzpicture}
}
}% end body 
{\captionof{figure}{\(y=\fe{f}{x}\)\label{figure-based-on-derivatives}}
}% caption 
\popValignCaptionBottom
\end{figure}
\par
\begin{exercisegroupbycol}{1}%
\begin{exercisegroupitem}{1. }\phantomsection\hypertarget{exercise-based-on-derivatives}{\null}
Draw onto \hyperref[figure-based-on-derivatives]{Figure~\ref*{figure-based-on-derivatives}} a \emph{continuous} function \(f\) that has a horizontal tangent line at the point \(\point{1}{2}\) along with the properties stated in \hyperref[table-based-on-derivatives]{Table~\ref*{table-based-on-derivatives}}.%
\end{exercisegroupitem}%
\par%
\begin{exercisegroupitem}{2. }\phantomsection\hypertarget{exercise-2}{\null}
Given the conditions stated in \hyperref[exercise-based-on-derivatives]{Exercise~1}, does it have to be the case that \(\fe{\sd{f}}{-1}=0\)?%
\end{exercisegroupitem}%
\par%
\begin{exercisegroupitem}{3. }\phantomsection\hypertarget{exercise-3}{\null}
Is \(f\) increasing at \(-1\)?  How do you know?%
\end{exercisegroupitem}%
\par%
\begin{exercisegroupitem}{4. }\phantomsection\hypertarget{exercise-4}{\null}
Can a continuous, everywhere differentiable function satisfy the properties stated in \hyperref[table-based-on-derivatives]{Table~\ref*{table-based-on-derivatives}} and not have a slope of zero at \(1\)?  Draw a picture that supports your answer.%
\end{exercisegroupitem}%
\par%
\end{exercisegroupbycol}%
\end{exercisegroup}%
\begin{exercisegroup}%
In the next few exercises, you will construct a function and then answer questions about it.%
\begin{figure}
\centering
\pushValignCaptionBottom[b]{minipage}{.60\textwidth}{%
\pgfplotsset{every axis/.append style={width=\linewidth}}%
\centering% horizontal alignment 
\begin{tabular}{ccc}\hrulethick
Interval&\(\fd{g}\)&\(\sd{g}\)\\\hrulemedium
\(\ointerval{-\infty}{-1}\)&Positive&Positive\\
\(\ointerval{-1}{\infty}\)&Positive&Negative
\end{tabular}
}% end body 
{\captionof{table}{Signs on \(\fd{g}\) and \(\sd{g}\)\label{table-based-on-derivatives-two}}
}% caption 
\pushValignCaptionBottom[b]{minipage}{.40\textwidth}{%
\pgfplotsset{every axis/.append style={width=\linewidth}}%
\centering% horizontal alignment 
{
\begin{tikzpicture}
\begin{axis}[]
\end{axis} 
\end{tikzpicture}
}
}% end body 
{\captionof{figure}{\(y=\fe{g}{x}\)\label{figure-based-on-derivatives-two}}
}% caption 
\popValignCaptionBottom
\end{figure}
\par
\begin{exercisegroupbycol}{1}%
\begin{exercisegroupitem}{5. }\phantomsection\hypertarget{exercise-based-on-derivatives-two}{\null}
Draw onto \hyperref[figure-based-on-derivatives-two]{Figure~\ref*{figure-based-on-derivatives-two}} a continuous function \(g\) that has a vertical tangent line at the point \(\point{-1}{2}\) along with the properties stated in \hyperref[table-based-on-derivatives-two]{Table~\ref*{table-based-on-derivatives-two}}.%
\end{exercisegroupitem}%
\par%
\begin{exercisegroupitem}{6. }\phantomsection\hypertarget{exercise-6}{\null}
Given the conditions stated in \hyperref[exercise-based-on-derivatives-two]{Exercise~5}, does it have to be the case that isundefined?%
\end{exercisegroupitem}%
\par%
\begin{exercisegroupitem}{7. }\phantomsection\hypertarget{exercise-7}{\null}
Is \(g\) increasing at \(-1\)?  How do you know?%
\end{exercisegroupitem}%
\par%
\begin{exercisegroupitem}{8. }\phantomsection\hypertarget{exercise-8}{\null}
Can a continuous, everywhere differentiable function satisfy the properties stated in \hyperref[table-based-on-derivatives-two]{Table~\ref*{table-based-on-derivatives-two}} and not have a vertical tangent line at \(1\)?  Draw a picture that supports your answer.%
\end{exercisegroupitem}%
\par%
\end{exercisegroupbycol}%
\end{exercisegroup}%
\begin{exercisegroup}%
In the next few exercises, you will construct a function and then answer questions about it.%
\begin{figure}
\centering
\pushValignCaptionBottom[b]{minipage}{.60\textwidth}{%
\pgfplotsset{every axis/.append style={width=\linewidth}}%
\centering% horizontal alignment 
\begin{tabular}{ccc}\hrulethick
Interval&\(\fd{k}\)&\(\sd{k}\)\\\hrulemedium
\(\ointerval{-\infty}{-1}\)&Negative&Positive\\
\(\ointerval{-1}{\infty}\)&Positive&Negative
\end{tabular}
}% end body 
{\captionof{table}{Signs on \(\fd{k}\) and \(\sd{k}\)\label{table-based-on-derivatives-three}}
}% caption 
\pushValignCaptionBottom[b]{minipage}{.40\textwidth}{%
\pgfplotsset{every axis/.append style={width=\linewidth}}%
\centering% horizontal alignment 
{
\begin{tikzpicture}
\begin{axis}[]
\end{axis} 
\end{tikzpicture}
}
}% end body 
{\captionof{figure}{\(y=\fe{k}{x}\)\label{figure-based-on-derivatives-three}}
}% caption 
\popValignCaptionBottom
\end{figure}
\par
\begin{exercisegroupbycol}{1}%
\begin{exercisegroupitem}{9. }\phantomsection\hypertarget{exercise-based-on-derivatives-three}{\null}
Draw onto \hyperref[figure-based-on-derivatives-three]{Figure~\ref*{figure-based-on-derivatives-three}} a continuous function \(k\) that passes through the point \(\point{-1}{2}\) and also satisfies the properties stated in \hyperref[table-based-on-derivatives-three]{Table~\ref*{table-based-on-derivatives-three}}.%
\end{exercisegroupitem}%
\par%
\begin{exercisegroupitem}{10. }\phantomsection\hypertarget{exercise-10}{\null}
At what values of \(x\) is \(k\) nondifferentiable?%
\end{exercisegroupitem}%
\par%
\end{exercisegroupbycol}%
\end{exercisegroup}%
\begin{exercisegroup}%
You should know by now that the first derivative  continually increases over intervals where the second derivative is constantly positive and that the first derivative continually decreases over intervals where the second derivative is constantly negative.  A person might infer from this that a function changes more and more quickly over intervals where the second derivative is constantly positive and that a function changes more and more slowly over intervals where the second derivative is constantly negative.   We are going to explore that idea in this problem.%
\par
\begin{exercisegroupbycol}{1}%
\begin{exercisegroupitem}{11. }\phantomsection\hypertarget{exercise-ice-cube}{\null}
Suppose that \(\fe{V}{t}\) is the volume of water in an ice cube (\si{\milli\liter}) where \(t\) is the amount of time that has passed since noon (measure in minutes).  Suppose that \(\fe{\fd{V}}{6}=0\,\frac{\text{ml}}{\text{min}}\) and that \(\fe{\sd{V}}{t}\) has a constant value of \(-0.3\,\frac{\sfrac{\text{ml}}{\text{min}}}{\text{min}}\) over the interval \(\cinterval{6}{11}\).  What is the value of \(\fe{\fd{V}}{11}\)?  When is \(V\) changing more quickly, at 12:06 pm or at 12:11 pm?%
\end{exercisegroupitem}%
\par%
\begin{exercisegroupitem}{12. }\phantomsection\hypertarget{exercise-12}{\null}
Referring to the function \(V\) in \hyperref[exercise-ice-cube]{Exercise~11}, what would the shape of \(V\) be over the interval \(\cinterval{6}{11}\)?  (Choose from options \hyperref[figure-curve-one]{(a)}\textendash{}\hyperref[figure-curve-four]{(d)}.)%
\begin{figure}
\centering
\pushValignCaptionBottom[b]{subfigure}{.12\textwidth}{%
\pgfplotsset{every axis/.append style={width=\linewidth}}%
\centering% horizontal alignment 
{
\begin{tikzpicture}
\begin{axis}[axis x line=none, axis y line=none,xmin=0,xmax=1,ymin=,ymax=]
    \addplot[domain=0:1,-]{sqrt((1-((x-1)/1.1)^2))};
\end{axis}
\end{tikzpicture}
}
}% end body 
{\caption{\label{figure-curve-one}}
}% caption 
\pushValignCaptionBottom[b]{subfigure}{.12\textwidth}{%
\pgfplotsset{every axis/.append style={width=\linewidth}}%
\centering% horizontal alignment 
{
\begin{tikzpicture}
\begin{axis}[axis x line=none, axis y line=none,xmin=0,xmax=1,ymin=,ymax=]
    \addplot[domain=0:1,-]{(x/1.1)^2};
\end{axis}
\end{tikzpicture}
}
}% end body 
{\caption{\label{figure-curve-two}}
}% caption 
\pushValignCaptionBottom[b]{subfigure}{.12\textwidth}{%
\pgfplotsset{every axis/.append style={width=\linewidth}}%
\centering% horizontal alignment 
{
\begin{tikzpicture}
\begin{axis}[axis x line=none, axis y line=none,xmin=0,xmax=1,ymin=,ymax=]
    \addplot[domain=0:1,-]{((x-1)/1.1)^2};
\end{axis}
\end{tikzpicture}
}
}% end body 
{\caption{\label{figure-curve-three}}
}% caption 
\pushValignCaptionBottom[b]{subfigure}{.12\textwidth}{%
\pgfplotsset{every axis/.append style={width=\linewidth}}%
\centering% horizontal alignment 
{
\begin{tikzpicture}
\begin{axis}[axis x line=none, axis y line=none,xmin=0,xmax=1,ymin=,ymax=]
    \addplot[domain=0:1,-]{1-(x/1.1)^2};
\end{axis}
\end{tikzpicture}
}
}% end body 
{\caption{\label{figure-curve-four}}
}% caption 
\popValignCaptionBottom
\caption{\label{sidebyside-curves}}
\end{figure}
\end{exercisegroupitem}%
\par%
\begin{exercisegroupitem}{13. }\phantomsection\hypertarget{exercise-13}{\null}
Again referring to options \hyperref[figure-curve-one]{(a)}\textendash{}\hyperref[figure-curve-four]{(d)}, which functions are changing more and more rapidly from left to right and which functions are changing more and more slowly from left to right? Which functions have positive second derivatives and which functions have negative second derivatives?  Do the functions with positive second derivative values both change more and more quickly from left to right?%
\end{exercisegroupitem}%
\par%
\begin{exercisegroupitem}{14. }\phantomsection\hypertarget{exercise-14}{\null}
Consider the signs on both the first and second derivatives in options \hyperref[figure-curve-one]{(a)}\textendash{}\hyperref[figure-curve-four]{(d)}.  Is there something that the two functions that change more and more quickly have in common that is different in the functions that change more and more slowly?%
\end{exercisegroupitem}%
\par%
\end{exercisegroupbycol}%
\end{exercisegroup}%
\begin{exercisegroup}%
Resolve each of the following disputes.%
\par
\begin{exercisegroupbycol}{1}%
\begin{exercisegroupitem}{15. }\phantomsection\hypertarget{exercise-15}{\null}
One day Sara and Jermaine were working on an assignment.  One question asked them to draw a function over the domain \(\ointerval{-2}{\infty}\) with the properties that the function is always increasing and always concave down.  Sara insisted that the curve must have a vertical asymptote at \(-2\) and Jermaine insisted that the function must have a horizontal asymptote somewhere.  Were either of these students correct?%
\end{exercisegroupitem}%
\par%
\begin{exercisegroupitem}{16. }\phantomsection\hypertarget{exercise-16}{\null}
The next question Sara and Jermaine encountered described the same function with the added condition that the function is never positive.  Sara and Jermaine made the same contentions about asymptotes.  Is one of them now correct?%
\end{exercisegroupitem}%
\par%
\begin{exercisegroupitem}{17. }\phantomsection\hypertarget{exercise-17}{\null}
At another table Pedro and Yoshi were asked to draw a continuous curve that, among other properties, was never concave up.  Pedro said ``OK, so the curve is always concave down'' to which Yoshi replied ``Pedro, you need to open your  mind to other possibilities.''  Who's right?%
\end{exercisegroupitem}%
\par%
\begin{exercisegroupitem}{18. }\phantomsection\hypertarget{exercise-18}{\null}
In the next problem Pedro and Yoshi were asked to draw a function that is everywhere continuous and that is concave down at every value of \(x\) \emph{except} \(3\).  Yoshi declared ``impossible'' and Pedro responded ``have some faith, Yosh-man.''  Pedro then began to draw. Is it possible that Pedro came up with such a function?%
\end{exercisegroupitem}%
\par%
\end{exercisegroupbycol}%
\end{exercisegroup}%
\begin{exercisegroup}%
Determine the correct answer to each of the following questions.  Pictures of the situation may help you determine the correct answers.%
\par
\begin{exercisegroupbycol}{1}%
\begin{exercisegroupitem}{19. }\phantomsection\hypertarget{exercise-19}{\null}
Which of the following propositions is true? If a given proposition is not true, draw a graph that illustrates its untruth.%
\begin{enumerate}[label=(\alph*)]
\item{}If the graph of \(f\) has a vertical asymptote, then the graph of \(\fd{f}\) must also have a vertical asymptote.\item{}If the graph of \(\fd{f}\) has a vertical asymptote, then the graph of \(f\) must also have a vertical asymptote.\end{enumerate}
\end{exercisegroupitem}%
\par%
\begin{exercisegroupitem}{20. }\phantomsection\hypertarget{exercise-20}{\null}
Suppose that the function \(f\) is everywhere continuous and concave down.  Suppose further that \(\fe{f}{7}=5\) and \(\fe{\fd{f}}{7}=3\).  Which of the following is true?%
\begin{enumerate}[label=(\alph*)]
\item{}\(\fe{f}{9}\lt11\)\item{}\(\fe{f}{9}=11\)\item{}\(\fe{f}{9}\gt11\)\item{}There is not enough information to determine the relationship between \(\fe{f}{9}\) and \(11\).\end{enumerate}
\end{exercisegroupitem}%
\par%
\end{exercisegroupbycol}%
\end{exercisegroup}%
\typeout{************************************************}
\typeout{Section 1.2 Supplement}
\typeout{************************************************}
\section[Supplement]{Supplement}\label{functions-derivatives-antiderivatives-supplementary-exercises}
\typeout{************************************************}
\typeout{Exercises}
\typeout{************************************************}
\section*{Exercises}\label{exercises-2}

\begin{exercisegroup}%
Sketch the first derivatives of the functions shown in \hyperref[exercise-sketch-first-supplement]{Exercises~1}\textendash{}{$\langle\langle$Unresolved xref, ref="exercise-sketch-last-supplement"; check spelling or use "provisional" attribute$\rangle\rangle$} onto the blank graphs provided.%
\par
\begin{exercisegroupbycol}{1}%
\begin{exercisegroupitem}{1. }\phantomsection\hypertarget{exercise-sketch-first-supplement}{\null}
\begin{figure}
\centering
\pushValignCaptionBottom[b]{minipage}{.40\textwidth}{%
\pgfplotsset{every axis/.append style={width=\linewidth}}%
\centering% horizontal alignment 
{
\begin{tikzpicture}
\begin{axis}[]
    \addplot+[samples=50,
        domain=-6.8:6.8,
    ]{atan((x-2)*2)/30-1};
    \addplot[asymptote,domain=-7:7]{2};
    \addplot[asymptote,domain=-7:7]{-4};
\end{axis}
\end{tikzpicture}
}
}% end body 
{}% caption 
\pushValignCaptionBottom[b]{minipage}{.40\textwidth}{%
\pgfplotsset{every axis/.append style={width=\linewidth}}%
\centering% horizontal alignment 
{
\begin{tikzpicture}
\begin{axis}[]
\end{axis}
\end{tikzpicture}
}
}% end body 
{}% caption 
\popValignCaptionBottom
\end{figure}
\end{exercisegroupitem}%
\par%
\end{exercisegroupbycol}%
\end{exercisegroup}%
%
\backmatter
%
\end{document}