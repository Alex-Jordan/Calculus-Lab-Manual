%%                                    %%
%% Generated from MathBook XML source %%
%%    on 2015-04-30T12:20:31-07:00    %%
%%                                    %%
%%   http://mathbook.pugetsound.edu   %%
%%                                    %%
\documentclass[12pt,]{book}
%% Load geometry package to allow page margin adjustments
\usepackage{geometry}
\geometry{letterpaper,total={5.0in,9.0in}}
%% Custom Preamble Entries, early (use latex.preamble.early)

\usepackage{titlesec}
\usepackage{fancyhdr}
\usepackage{longtable}

%% Inline math delimiters, \(, \), made robust with next package
\usepackage{fixltx2e}
%% Page Layout Adjustments (latex.geometry)
\geometry{letterpaper,total={6.25in,9.0in}}
%% For unicode character support, use the "xelatex" executable
%% If never using xelatex, the next three lines can be removed
\usepackage{ifxetex}
\ifxetex\usepackage{xltxtra}\fi
%% Symbols, align environment, bracket-matrix
\usepackage{amsmath}
\usepackage{amssymb}
%% allow more columns to a matrix
%% can make this even bigger by overiding with  latex.preamble.late  processing option
\setcounter{MaxMatrixCols}{30}
%% XML, MathJax Conflict Macros
%% Two nonstandard macros that MathJax supports automatically
%% so we always define them in order to allow their use and
%% maintain source level compatibility
%% This avoids using two XML entities in source mathematics
\newcommand{\lt}{<}
\newcommand{\gt}{>}
%% xfrac package for 'beveled fractions': http://tex.stackexchange.com/questions/3372/how-do-i-typeset-arbitrary-fractions-like-the-standard-symbol-for-5-%C2%BD
\usepackage{xfrac}
%% Semantic Macros
%% To preserve meaning in a LaTeX file
%% Only defined here if required in this document
%% Used for inline definitions of terms
\newcommand{\terminology}[1]{\textbf{#1}}
%% Used for units and number formatting
\usepackage[per-mode=fraction]{siunitx}
\ifxetex\sisetup{math-micro=\text{µ},text-micro=µ}\fi%% Common non-SI units
\DeclareSIUnit\degreeFahrenheit{\SIUnitSymbolDegree{F}}
\DeclareSIUnit\fahrenheit{\degreeFahrenheit}
\DeclareSIUnit\pound{lb}
\DeclareSIUnit\foot{ft}
\DeclareSIUnit\inch{in}
\DeclareSIUnit\yard{yd}
\DeclareSIUnit\mile{mi}
\DeclareSIUnit\mileperhour{mph}
\DeclareSIUnit\gallon{gal}
%% Subdivision Numbering, Chapters, Sections, Subsections, etc
%% Subdivision numbers may be turned off at some level ("depth")
%% A section *always* has depth 1, contrary to us counting from the document root
%% The latex default is 3.  If a larger number is present here, then
%% removing this command may make some cross-references ambiguous
%% The precursor variable $numbering-maxlevel is checked for consistency in the common XSL file
\setcounter{secnumdepth}{3}
%% Environments with amsthm package
%% Theorem-like enviroments in "plain" style, with or without proof
\usepackage{amsthm}
\theoremstyle{plain}
%% Numbering for Theorems, Conjectures, Examples, Figures, etc
%% Controlled by  numbering.theorems.level  processing parameter
%% Always need a theorem environment to set base numbering scheme
%% even if document has no theorems (but has other environments)
\newtheorem{theorem}{Theorem}[section]
\renewcommand*{\proofname}{Proof}%% Only variants actually used in document appear here
%% Numbering: all theorem-like numbered consecutively
%% i.e. Corollary 4.3 follows Theorem 4.2
%% Definition-like environments, normal text
%% Numbering for definition, examples is in sync with theorems, etc
%% also for free-form exercises, not in exercise sections
\theoremstyle{definition}
\newtheorem{definition}[theorem]{Definition}
\newtheorem{example}[theorem]{Example}
\newtheorem{exercise}[theorem]{Exercise}
%% Equation Numbering
%% Controlled by  numbering.equations.level  processing parameter
\numberwithin{equation}{section}
%% For improved tables
\usepackage{array}
%% Some extra height on each row is desirable, especially with horizontal rules
%% Increment determined experimentally
\setlength{\extrarowheight}{0.2ex}
%% Define variable thickness horizontal rules, full and partial
%% Thicknesses are 0.03, 0.05, 0.08 in the  booktabs  package
\makeatletter
\newcommand{\hrulethin}  {\noalign{\hrule height 0.04em}}
\newcommand{\hrulemedium}{\noalign{\hrule height 0.07em}}
\newcommand{\hrulethick} {\noalign{\hrule height 0.11em}}
%% We preserve a copy of the \setlength package before other
%% packages (extpfeil) get a change to load packages that redefine it
\let\oldsetlength\setlength
\newlength{\Oldarrayrulewidth}
\newcommand{\crulethin}[1]%
{\noalign{\global\oldsetlength{\Oldarrayrulewidth}{\arrayrulewidth}}%
\noalign{\global\oldsetlength{\arrayrulewidth}{0.04em}}\cline{#1}%
\noalign{\global\oldsetlength{\arrayrulewidth}{\Oldarrayrulewidth}}}%
\newcommand{\crulemedium}[1]%
{\noalign{\global\oldsetlength{\Oldarrayrulewidth}{\arrayrulewidth}}%
\noalign{\global\oldsetlength{\arrayrulewidth}{0.07em}}\cline{#1}%
\noalign{\global\oldsetlength{\arrayrulewidth}{\Oldarrayrulewidth}}}
\newcommand{\crulethick}[1]%
{\noalign{\global\oldsetlength{\Oldarrayrulewidth}{\arrayrulewidth}}%
\noalign{\global\oldsetlength{\arrayrulewidth}{0.11em}}\cline{#1}%
\noalign{\global\oldsetlength{\arrayrulewidth}{\Oldarrayrulewidth}}}
%% Single letter column specifiers defined via array package
\newcolumntype{A}{!{\vrule width 0.04em}}
\newcolumntype{B}{!{\vrule width 0.07em}}
\newcolumntype{C}{!{\vrule width 0.11em}}
\makeatother
%% Figures, Tables, Floats
%% The [H]ere option of the float package fixes floats in-place,
%% in deference to web usage, where floats are totally irrelevant
%% We redefine the figure and table environments, if used
%%   1) New mbxfigure and/or mbxtable environments are defined with float package
%%   2) Standard LaTeX environments redefined to use new environments
%%   3) Standard LaTeX environments redefined to step theorem counter
%%   4) Counter for new enviroments is set to the theorem counter before caption
%% You can remove all this figure/table setup, to restore standard LaTeX behavior
%% HOWEVER, numbering of figures/tables AND theorems/examples/remarks, etc
%% WILL ALL de-synchronize with the numbering in the HTML version
%% You can remove the [H] argument of the \newfloat command, to allow flotation and 
%% preserve numbering, BUT the numbering may then appear "out-of-order"
\usepackage{float}
\usepackage[bf]{caption} % http://tex.stackexchange.com/questions/95631/defining-a-new-type-of-floating-environment 
\usepackage{newfloat}
\usepackage{subcaption}
\captionsetup[subfigure]{labelformat=simple}
\captionsetup[subtable]{labelformat=simple}
\renewcommand\thesubfigure{(\alph{subfigure})}
\makeatletter
% we plan to use subtables within figure environments, so they need to reset accordingly
\@addtoreset{subtable}{figure}
\makeatother
% Side-by-side elements need careful treatement for aligning captions, see: 
% http://tex.stackexchange.com/questions/230335/vertically-aligning-minipages-subfigures-and-subtables-not-with-baseline 
\usepackage{stackengine,ifthen}
\newcounter{figstack}
\newcounter{figindex}
\newlength\fight
\newcommand\pushValignCaptionBottom[5][b]{%
\stepcounter{figstack}%
\expandafter\def\csname %
figalign\romannumeral\value{figstack}\endcsname{#1}%
\expandafter\def\csname %
figtype\romannumeral\value{figstack}\endcsname{#2}%
\expandafter\def\csname %
figwd\romannumeral\value{figstack}\endcsname{#3}%
\expandafter\def\csname %
figcontent\romannumeral\value{figstack}\endcsname{#4}%
\expandafter\def\csname %
figcap\romannumeral\value{figstack}\endcsname{#5}%
\setbox0=\hbox{%
\begin{#2}{#3}#4\end{#2}}%
\ifdim\dimexpr\ht0+\dp0\relax>\fight\global\setlength{\fight}{%
\dimexpr\ht0+\dp0\relax}\fi%
}
\newcommand\popValignCaptionBottom{%
\setcounter{figindex}{0}%
\hfill%
\whiledo{\value{figindex}<\value{figstack}}{%
\stepcounter{figindex}%
\def\tmp{\csname figwd\romannumeral\value{figindex}\endcsname}%
\begin{\csname figtype\romannumeral\value{figindex}\endcsname}[t]{\tmp}%
\centering%
\stackinset{c}{}%
{\csname figalign\romannumeral\value{figindex}\endcsname}{}%
{\csname figcontent\romannumeral\value{figindex}\endcsname}%
{\rule{0pt}{\fight}}\par%
\csname figcap\romannumeral\value{figindex}\endcsname%
\end{\csname figtype\romannumeral\value{figindex}\endcsname}%
\hfill%
}%
\setcounter{figstack}{0}%
\setlength{\fight}{0pt}%
\hfill%
}
% Figure environment setup so that it no longer floats
\SetupFloatingEnvironment{figure}{fileext=lof,placement={H},within=section,name=Figure}
% figures have the same number as theorems: http://tex.stackexchange.com/questions/16195/how-to-make-equations-figures-and-theorems-use-the-same-numbering-scheme 
\makeatletter
\let\c@figure\c@theorem
\makeatother
% Table environment setup so that it no longer floats
\SetupFloatingEnvironment{table}{fileext=lot,placement={H},within=section,name=Table}
% tables have the same number as theorems: http://tex.stackexchange.com/questions/16195/how-to-make-equations-figures-and-theorems-use-the-same-numbering-scheme 
\makeatletter
\let\c@table\c@theorem
\makeatother
%% Raster graphics inclusion, wrapped figures in paragraphs
\usepackage{graphicx}
%% Colors for Sage boxes and author tools (red hilites)
\usepackage[usenames,dvipsnames,svgnames,table]{xcolor}
%% Multiple column, column-major lists
\usepackage{multicol}
%% More flexible list management, esp. for references and exercises
%% But also for specifying labels (ie custom order) on nested lists
\usepackage{enumitem}
%% Lists of exercises in their own section, maximum depth 4
\newlist{exerciselist}{description}{4}
\setlist[exerciselist]{leftmargin=0em,itemsep=-1.0ex,topsep=1.0ex,partopsep=0pt,parsep=0pt}
\newenvironment{exercisegroup}%
{\medskip\noindent}%
{\par\bigskip}%
\usepackage{changepage}%
\newlength{\exercisegroupindent}%
\setlength{\exercisegroupindent}{2em}%
\newlength{\exercisegroupitemwidth}%
\newenvironment{exercisegrouplist}%
{\vspace{-\partopsep}%
\begin{adjustwidth}{\exercisegroupindent}{0em}}%
{\end{adjustwidth}%
\vspace{-\partopsep}%
\vspace{\baselineskip}}%
\newenvironment{exercisegroupbyrow}[1]%
{\begin{exercisegrouplist}%
\setlength{\parindent}{0em}%
\setlength{\exercisegroupitemwidth}{\linewidth}%
\addtolength{\exercisegroupitemwidth}{\columnsep}%
\divide\exercisegroupitemwidth by #1%
\addtolength{\exercisegroupitemwidth}{-\columnsep}}%
{\end{exercisegrouplist}}%
%% To allow for multicols to just have one column
%% http://tex.stackexchange.com/questions/233866/one-column-multicol-environment#answer-233904
\usepackage{xparse}%
\let\multicolmulticols\multicols%
\let\endmulticolmulticols\endmulticols%
\RenewDocumentEnvironment{multicols}{mO{}}%
 {%
  \ifnum#1=1%
    \vspace{\multicolsep}#2%
  \else % More than 1 column%
    \multicolmulticols{#1}[#2]%
  \fi%
 }%
 {%
  \ifnum#1=1%
    \vspace{\multicolsep}%
  \else % More than 1 column%
    \endmulticolmulticols%
  \fi%
 }%
\newenvironment{exercisegroupbycol}[1]%
{\begin{exercisegrouplist}%
\vspace{-\multicolsep}%
\begin{multicols}{#1}%
\setlength{\parindent}{0em}%
\setlength{\exercisegroupitemwidth}{\linewidth}}%
{\end{multicols}%
\vspace{-\multicolsep}%
\end{exercisegrouplist}}%
\setlength{\fboxsep}{0pt}%
\newenvironment{exercisegroupitem}[1]%
{\begin{minipage}[t]{\exercisegroupitemwidth}
\vspace{0pt}%
{\bfseries#1}%
\rule{0pt}{\baselineskip}}{\strut%
\end{minipage}%
\hspace{\columnsep}}%
%% hyperref driver does not need to be specified
\usepackage{hyperref}
%% Hyperlinking active in PDFs, all links solid and blue
\hypersetup{colorlinks=true,linkcolor=blue,citecolor=blue,filecolor=blue,urlcolor=blue}
\hypersetup{pdftitle={Portland Community College MTH 251 Lab Manual}}
%% If you manually remove hyperref, leave in this next command
\providecommand\phantomsection{}
%% Graphics Preamble Entries
\usepackage{pgfplots}
\usepackage{xparse}

% cycle list- truly awesome; see section 4.6.7, pg 129 of pgfplots
\pgfplotscreateplotcyclelist{pccstylelist}{%
    color=red,mark=none,line width=1pt,<->,solid\\%
    color=blue,mark=none,line width=1pt,<->,dashdotted\\%
    color=gray,mark=none,line width=1pt,<->,dashdotdotted\\%
}

\pgfplotsset{every axis/.append style={
    axis x line=middle,    % put the x axis in the middle
    axis y line=middle,    % put the y axis in the middle
    axis line style={<->}, % arrows on the axis
    xlabel={$x$},          % default put x on x-axis
    ylabel={$y$},          % default put y on y-axis
    xmin = -7,
    xmax = 7,
    ymin = -7,
    ymax = 7,
    xtick = {-6,-4,...,6},
    ytick = {-6,-4,...,6},
    minor xtick = {-7,-6,...,7},
    minor ytick = {-7,-6,...,7},
    scale only axis,       % otherwise width won't be as intended: http://tex.stackexchange.com/questions/36297/pgfplots-how-can-i-scale-to-text-width
    cycle list name=pccstylelist,
    %tick label style={font=\small},
    %label style={font=\small},
    legend cell align=left,
    %legend style={font=\tiny},
    width=0.4\textwidth,
    grid=minor,
    every node near coord/.append style={
        %font=\small
    },
}}

%\tikzset{axisnode/.style={font=\tiny,text=black}}

% line style
\pgfplotsset{pccplot/.style={color=red,mark=none,line width=1pt,<->}} % this is pretty redundant in most cases now that cycle list is implemented
\pgfplotsset{asymptote/.style={color=gray,mark=none,line width=1pt,<->,dashed}}
\pgfplotsset{soldot/.style={color=red,only marks,mark=*}}
\pgfplotsset{holdot/.style={color=red,fill=white,only marks,mark=*}}


% arrow style
\tikzset{>=stealth}

% framing the graphs
\pgfplotsset{framed/.style={axis background/.style ={draw=gray}}}
% next line is a bit more colourful
%\pgfplotsset{framed/.style={axis background/.style ={draw=gray,fill=yellow!20,rounded corners=3ex}}}

% grid style
\pgfplotsset{grid style={dotted,gray!90}}

% for annotating equations
\newcommand{\tikzmark}[1]{\tikz[overlay,remember picture] \node[baseline] (#1) {};}%
\NewDocumentCommand{\Annotate}{O{} O{} m m m m m m}{%
    % #1 = line draw options
    % #2 = node options
    % #3 = x-coordinate displacement start
    % #4 = y-coordinate displacement start
    % #5 = x-coordinate displacement end
    % #6 = y-coordinate displacement end
    % #7 = tikzmark name
    % #8 = node text
    % Example:
    % X =\tikzmark{Point A}  Y
    % \Annotate[][anchor=south west]{-1.5ex}{1.5ex}{0.7}{0.5}{Point A}{Always true}
\begin{tikzpicture}[remember picture, overlay]
    \draw [thick, -latex, shorten >=1pt, #1]
        (#7) ++ (#3,#4) -- +(#5,#6)
        node [black, right, draw=black, #2] {#8};
\end{tikzpicture}%
}%
%% extpfeil package for certain extensible arrows,
%% as also provided by MathJax extension of the same name
%% NB: this package loads mtools, which loads calc, which redefines
%%     \setlength, so it can be removed if it seems to be in the 
%%     way and your math does not use:
%%     
%%     \xtwoheadrightarrow, \xtwoheadleftarrow, \xmapsto, \xlongequal, \xtofrom
%%     
%%     we have had to be extra careful with variable thickness
%%     lines in tables, and so also load this package late
\usepackage{extpfeil}
%% Custom Preamble Entries, late (use latex.preamble.late)


%%%%%%%%%%%%%%%%%%%%%%%%%%%%%%%%%%%%%%%%%%%%%%%%%%%%%%
% Rename "Chapter" to "Lab"
%%%%%%%%%%%%%%%%%%%%%%%%%%%%%%%%%%%%%%%%%%%%%%%%%%%%%%
\renewcommand{\chaptername}{Lab}
%%%%%%%%%%%%%%%%%%%%%%%%%%%%%%%%%%%%%%%%%%%%%%%%%%%%%%
% Header and Footer
%%%%%%%%%%%%%%%%%%%%%%%%%%%%%%%%%%%%%%%%%%%%%%%%%%%%%%
\renewcommand{\sectionmark}[1]{%
 \markright{\slshape\MakeUppercase{%
 Activity \thesection.%
 \ #1}}}%

%%%%%%%%%%%%%%%%%%%%%%%%%%%%%%%%%%%%%%%%%%%%%%%%%%%%%%
\allowdisplaybreaks


%% Convenience macros
% These macros are automatically generated from the "macros"
% XML element.  Make permanent edits there.
%
%%%%%%%%%%%%%%%%%%%%%
%
%     Conveniences
%
%%%%%%%%%%%%%%%%%%%%%
%
%  Integers
%  Usage:  \Z
\newcommand{\Z}{\mathbb{Z}}
%
%  Real numbers, as set of scalars
%  Usage:  \reals
\newcommand{\reals}{\mathbb{R}}
%
%  n-space over real field
%  Usage: \complex{integer-dimension}
\newcommand{\real}[1]{\mathbb{R}^{#1}}
%
%  evaluate a function
%  Usage: \fe{function-name}{input}
\newcommand{\fe}[2]{\mathop{{#1}{\left(#2\right)}}}
%
%  closed interval
%  Usage: \cinterval{left-endpoint}{right-endpoint}
\newcommand{\cinterval}[2]{\left[#1,#2\right]}
%
%  open interval
%  Usage: \ointerval{left-endpoint}{right-endpoint}
\newcommand{\ointerval}[2]{\left(#1,#2\right)}
%
%  closed-open interval
%  Usage: \cointerval{left-endpoint}{right-endpoint}
\newcommand{\cointerval}[2]{\left[\left.#1,#2\right)\right.}
%
%  open-closed interval
%  Usage: \ocinterval{left-endpoint}{right-endpoint}
\newcommand{\ocinterval}[2]{\left(\left.#1,#2\right]\right.}
%
%  point
%  Usage: \point{x}{y}
\newcommand{\point}[2]{\left(#1,#2\right)}
%
%  first derivative
%  Usage: \fd{f}
\newcommand{\fd}[1]{#1'}
%
%  second derivative
%  Usage: \sd{f}
\newcommand{\sd}[1]{#1''}
%
%  third derivative
%  Usage: \td{f}
\newcommand{\td}[1]{#1'''}
%
%  Leibniz notation
%  Usage: \lz{y}{x}
\newcommand{\lz}[2]{\frac{d#1}{d#2}}
%
%  higher Leibniz notation
%  Usage: \lzn{n}{y}{x}
\newcommand{\lzn}[3]{\frac{d^{#1}#2}{d#3^{#1}}}
%
%  Leibniz operator
%  Usage: \lzo{x}
\newcommand{\lzo}[1]{\frac{d}{d#1}}
%
%  Leibniz operator on ....
%  Usage: \lzoo{x}{y}
\newcommand{\lzoo}[2]{{\frac{d}{d#1}}{\left(#2\right)}}
%
%  higher Leibniz operator
%  Usage: \lzon{x}{y}{n}
\newcommand{\lzon}[2]{\frac{d^{#1}}{d#2^{#1}}}
%
%  Leibniz operator at ....
%  Usage: \lzoa{y}{x}{a}
\newcommand{\lzoa}[3]{\left.{\frac{d#1}{d#2}}\right|_{#3}}
%
%% Title page information for book
\title{Portland Community College MTH 251 Lab Manual}
\author{Steve Simonds\\
Department of Mathematics\\
Portland Community College\\
\href{mailto:ssimonds@pcc.edu}{\nolinkurl{ssimonds@pcc.edu}}
\and
Alex Jordan, Editor\\
Department of Mathematics\\
Portland Community College\\
\href{mailto:alex.jordan@pcc.edu}{\nolinkurl{alex.jordan@pcc.edu}}
}
\date{DRAFT April 30, 2015 DRAFT}
\begin{document}
\frontmatter
%% half-title
\thispagestyle{empty}
\vspace*{\stretch{1}}
\begin{center}
{\Huge Portland Community College MTH 251 Lab Manual}
\end{center}\par
\vspace*{\stretch{2}}
\clearpage
\thispagestyle{empty}
\clearpage
\maketitle
\clearpage
\thispagestyle{empty}
\vspace*{\stretch{2}}
\vspace*{\stretch{1}}
\clearpage
\setcounter{tocdepth}{1}
\renewcommand*\contentsname{Contents}
\tableofcontents
\chapter*{To the Student}\label{to-the-student}
\addcontentsline{toc}{chapter}{To the Student}
MTH 251 is taught at Portland Community College using a lecture/lab format. The laboratory time is set aside for students to investigate the topic and practice the skills that are covered during their lecture periods.%
\par
The lab activities have been written under the presumption that students will be working in groups and will be actively discussing the examples and problems included in each activity. Many of the exercises and problems lend themselves quite naturally to discussion. Some of the more algebraic problems are not so much discussion problems as they are ``practice and help'' problems.%
\par
You do not need to fully understand an example be fore starting on the associated problems. The intent is that your understanding of the material will grow while you work on the problems.%
\par
When working through the lab activities, the students in a given group should be working on the same activity at the same time. Sometimes this means an individual student will have to go a little more slowly than he or she may like and sometimes it means an individual student will need to move on to the next activity before he or she fully grasps the current activity.%
\par
Many instructors will want you to focus some of your energy on the way you write your mathematics. It is important that you do not rush through the activities. Write your solutions as if they are going to be graded; that way you will know during lab time if you understand the proper way to write and organize your work.%
\par
If your lab section meets more than once a week, \emph{you should not work on lab activities between lab sections that occur during the same week.} It is OK to work on lab activities outside of class once the entire classroom time allotted for that lab has passed.%
\par
There are no written solutions for the lab activity problems. Between your group mates, your instructor, and (if you have one) your lab assistant, you should know whether or not you have the correct answers and proper writing strategies for these problems.%
\par
Each lab has a section of supplementary exercises; these exercises are fully keyed with solutions. The supplementary exercises are not simply copies of the problems in the lab activities. While some questions will look familiar, many others will challenge you to apply the material covered in the lab to a new type of problem. These questions are meant to supplement your textbook homework, not replace your textbook homework.%
\chapter*{To the Instructor}\label{to-the-instructor}
\addcontentsline{toc}{chapter}{To the Instructor}
This manual is significantly different from earlier versions of the lab manual. The topics have been arranged in a developmental order. Because of this, students who work each activity in the order they appear may not get to all of the topics covered in a particular week.%
\par
It is strongly recommended that the instructor pick and choose what they consider to be the most vital activities for a given week, and that the instructor have the students work those activities first. For some activities you might also want to have the students only work selected problems in the activity. Students who complete the high priority activities and problems can then go back and work the activities that they initially skipped. There are also fully keyed problems in the supplementary exercises that the students could work on both during lab time and outside of class.%
\par
A suggested schedule for the labs is shown in Table \hyperref[table-suggested-schedule]{.0.1}. Again, the instructor should choose what they feel to be the most relevant activities and problems for a given week, and have the students work those activities and problems first.%
\begin{longtable}{clc}\\
\hrulethick
Week&Labs (Lab Activities)&Supplementary Exercises\\\hrulemedium
1&\hyperref[section-velocity]{Velocity~\ref*{section-velocity}}&\\
&\hyperref[section-secant]{Secant Line to a Curve~\ref*{section-secant}}&\\
&\hyperref[section-difference-quotient]{The Difference Quotient~\ref*{section-difference-quotient}}&\\
&\hyperref[section-limits]{Limits~\ref*{section-limits}}&\\\hrulemedium
2&\hyperref[section-limit-laws]{Limits Laws~\ref*{section-limit-laws}}&\\
&\hyperref[section-indeterminate-limits]{Indeterminate Limits~\ref*{section-indeterminate-limits}}&\\
&\hyperref[section-limits-at-infinity]{Limits at Infinity~\ref*{section-limits-at-infinity}}&\\
&\hyperref[section-limits-at-infinity-tending-to-zero]{Limits at Infinity Tending to Zero~\ref*{section-limits-at-infinity-tending-to-zero}}&\\
&\hyperref[section-ratios-of-infinities]{Ratios of Infinities~\ref*{section-ratios-of-infinities}}&\\
&\hyperref[section-nonexistent-limits]{Non-existent Limits~\ref*{section-nonexistent-limits}}&\\
&\hyperref[section-vertical-asymptotes]{Vertical Asymptotes~\ref*{section-vertical-asymptotes}}&\\
&\hyperref[section-continuity]{Continuity~\ref*{section-continuity}}&\\
&\hyperref[section-discontinuities]{Discontinuities~\ref*{section-discontinuities}}&\\
&\hyperref[section-continuity-on-an-interval]{Continuity on an Interval~\ref*{section-continuity-on-an-interval}}&\\
&\hyperref[section-discontinuous-formulas]{Discontinuous Formulas~\ref*{section-discontinuous-formulas}}&\\
&\hyperref[section-piecewise-defined-functions]{Piecewise-Defined Functions~\ref*{section-piecewise-defined-functions}}&\\\hrulemedium
3&\hyperref[section-instantaneous-velocity]{Instantaneous Velocity~\ref*{section-instantaneous-velocity}}&\\
&\hyperref[section-tangent-lines]{Tangent Lines~\ref*{section-tangent-lines}}&\\
&\hyperref[section-first-derivative]{The First Derivative~\ref*{section-first-derivative}}&\\
&\hyperref[section-derivative-units]{Derivative Units~\ref*{section-derivative-units}}&\\\hrulemedium
4&\hyperref[section-graph-features]{Graph Features~\ref*{section-graph-features}}&\\
&\hyperref[section-graphical-derivatives]{Graphical Derivatives~\ref*{section-graphical-derivatives}}&\\
&\hyperref[section-nondifferentiability]{Nondifferentiability~\ref*{section-nondifferentiability}}&\\
&\hyperref[section-higher-order-derivatives]{Higher Order Derivatives~\ref*{section-higher-order-derivatives}}&\\\hrulemedium
5&\hyperref[section-antiderivatives]{Antiderivatives~\ref*{section-antiderivatives}}&\\
&\hyperref[section-graphical-features-from-derivatives]{Graphical Features from Derivatives~\ref*{section-graphical-features-from-derivatives}}&\\\hrulemedium
6&\hyperref[section-leibniz-notation]{Leibniz Notation~\ref*{section-leibniz-notation}}&\\
&\hyperref[section-derivative-operator]{The Derivative Operator~\ref*{section-derivative-operator}}&\\
&\hyperref[section-power-rule]{The Power Rule~\ref*{section-power-rule}}&\\
&\hyperref[section-constant-factor-rule]{The Constant Factor Rule~\ref*{section-constant-factor-rule}}&\\
&\hyperref[section-constant-divisor-rule]{The Constant Divisor Rule~\ref*{section-constant-divisor-rule}}&\\
&\hyperref[section-sum-and-difference-rules]{The Sum and Difference Rules~\ref*{section-sum-and-difference-rules}}&\\
&\hyperref[section-product-rule]{Product Rule~\ref*{section-product-rule}}&\\
&\hyperref[section-quotient-rule]{Quotient Rule~\ref*{section-quotient-rule}}&\\
&\hyperref[section-simplification]{Simplification~\ref*{section-simplification}}&\\
&\hyperref[section-product-and-quotient-together]{Product and Quotient Rules Together~\ref*{section-product-and-quotient-together}}&\\
&\hyperref[section-derivative-formulas-and-function-behavior]{Derivative Formulas and Function Behavior~\ref*{section-derivative-formulas-and-function-behavior}}&\\\hrulemedium
7&\hyperref[section-introduction-to-the-chain-rule]{Introduction to the Chain Rule~\ref*{section-introduction-to-the-chain-rule}}&\\
&\hyperref[section-order-to-apply-rules]{Order to Apply Rules~\ref*{section-order-to-apply-rules}}&\\
&\hyperref[section-not-simplifying-first]{Not Simplifying First~\ref*{section-not-simplifying-first}}&\\
&\hyperref[section-chain-rule-and-leibniz]{Chain Rule with Leibniz Notation~\ref*{section-chain-rule-and-leibniz}}&\\\hrulemedium
\end{longtable}
\mainmatter
\typeout{************************************************}
\typeout{Chapter 1 Rates Of Change}
\typeout{************************************************}
\chapter[Rates Of Change]{Rates Of Change}\label{chapter-rates-of-change}
\typeout{************************************************}
\typeout{Section 1.1 Velocity}
\typeout{************************************************}
\section[Velocity]{Velocity}\label{section-velocity}
Motion is frequently modeled using calculus. A building block for this application is the concept of \terminology{average velocity}. Average velocity is defined to be net displacement divided by elapsed time.  More precisely,%
\begin{definition}[Average Velocity]\label{definition-average-velocity}
If \(p\) is a position function for something moving along a numbered line, then we define the \terminology{average velocity} over the time interval \(\cinterval{t_0}{t_1}\) to be: \[\text{average velocity}=\frac{\fe{p}{t_1}-\fe{p}{t_0}}{t_1-t_0}\text{.}\]%
\end{definition}
\typeout{************************************************}
\typeout{Exercises}
\typeout{************************************************}
\section*{Exercises}\label{exercises-1}

According to simplified Newtonian physics, if an object is dropped from an elevation of \SI{200}{\meter} and allowed to free fall to the ground, then the elevation of the object (measured in \si{\meter}) is given by the position function \(\fe{p}{t}=200-4.9t^2\) where \(t\) is the amount of time that has passed since the object was dropped (measured in \si{\second}).%
\begin{exerciselist}
\item[1.]\phantomsection\hypertarget{exercise-1}{\null}What, \emph{including unit}, are the values of \(\fe{p}{t}\) three seconds and five seconds into the object's fall? Use these values when working \hyperref[exercise-average-velocity]{Exercise~2}.%
\par\smallskip
\item[2.]\phantomsection\hypertarget{exercise-average-velocity}{\null}Calculate \(\frac{\fe{p}{5\,\text{s}}-\fe{p}{3\,\text{s}}}{{5\,\text{s}}-{3\,\text{s}}}\); \emph{include units while making the calculation}. What does the result tell you in the context of this problem?%
\par\smallskip
\item[3.]\phantomsection\hypertarget{exercise-average-velocity-formula}{\null}Use \hyperref[definition-average-velocity]{Definition~\ref*{definition-average-velocity}} to find a formula for the average velocity of this object over the general time interval \(\cinterval{t_0}{t_1}\). The first couple of lines of this process are shown below. Copy these lines onto your paper and continue the simplification process.\begin{align*}
\frac{\fe{p}{t_1}-\fe{p}{t_0}}{t_1-t_0}&=\frac{\left[200-4.9t_1^2\right]-\left[200-4.9t_0^2\right]}{t_1-t_0}\\
&=\frac{200-4.9t_1^2-200+4.9t_0^2}{t_1-t_0}\\
&=\frac{-4.9t_1^2+4.9t_0^2}{t_1-t_0}\\
&=\cdots
\end{align*}
                Hint: In the next step you should factor \(-4.9\) from the numerator; the remaining factor will factor further.%
\par\smallskip
\item[4.]\phantomsection\hypertarget{exercise-4}{\null}Check the formula you derived in \hyperref[exercise-average-velocity-formula]{Exercise~3} using \(t_0=3\) and \(t_1=5\); that is, compare the value generated by the formula to that you found in \hyperref[exercise-average-velocity]{Exercise~2}.%
\par\smallskip
\item[5.]\phantomsection\hypertarget{exercise-5}{\null}Using the formula found in \hyperref[exercise-average-velocity-formula]{Exercise~3}, replace \(t_0\) with \(3\) but leave \(t_1\) as a variable; simplify the result. Then copy \hyperref[table-velocity]{Table~\ref*{table-velocity}} onto your paper and fill in the missing entries.%
\begin{table}
\centering
\caption{\(y=\frac{\fe{p}{t_1}-\fe{p}{3}}{t_1-3}\)\label{table-velocity}}
\begin{tabular}{lc}\hrulethick
\(t_1\,\text{(s)}\)&\(y\,\left(\sfrac{\text{m}}{\text{s}}\right)\)\\\hrulemedium
2.9&\\
2.99&\\
2.999&\\
3.001&\\
3.01&\\
3.1&\\\hrulethick
\end{tabular}
\end{table}
\par\smallskip
\item[6.]\phantomsection\hypertarget{exercise-6}{\null}As the value of \(t_1\) gets closer to \(3\), the values in the \(y\)-column of \hyperref[table-velocity]{Table~\ref*{table-velocity}} appear to be converging on a single number; what is this number and what do you think it tells you in the context of this problem?%
\par\smallskip
\end{exerciselist}
\typeout{************************************************}
\typeout{Section 1.2 Secant Line to a Curve}
\typeout{************************************************}
\section[Secant Line to a Curve]{Secant Line to a Curve}\label{section-secant}
One of the building blocks in differential calculus is the \emph{secant line to a curve}. It is very easy for a line to be considered a secant line to a curve; the only requirement that must be fulfilled is that the line intersects the curve in at least two points.%
\par
In \hyperref[figure-secant]{Figure~\ref*{figure-secant}}, a secant line to the curve \(y=\fe{f}{x}\) has been drawn through the points \(\point{0}{3}\) and \(\point{4}{5}\). You should verify that the slope of this line is \(-2\).%
\begin{figure}
\centering
{
\begin{tikzpicture}
\begin{axis}[]
    \addplot+[
        domain=-2.3:4.3,
        <->,
    ]{3+2*x-x^2};
    \addplot+[
        domain=-1.945:4.945,
        <->,
    ]{3-2*x};
    \addplot[
        soldot
    ]coordinates{
        (0,3)
        (4,-5)};
\end{axis}
\end{tikzpicture}
}
\caption{\(f\)\label{figure-secant}}
\end{figure}
\par
The formula for \(f\) is \(\fe{f}{x}=3+2x-x^2\). We can use this formula to come up with a generalized formula for the slope of secant lines to this curve. Specifically, the slope of the line connecting the point \(\point{x_0}{\fe{f}{x_0}}\) to the point \(\point{x_1}{\fe{f}{x_1}}\) is derived in \hyperref[example-secant]{Example~\ref*{example-secant}}.%
\begin{example}[Calculating Secant Slope]\label{example-secant}
\begin{align*}
m_{\text{sec}}&=\frac{\fe{f}{x_1}-\fe{f}{x_0}}{x_1-x_0}\\
&=\frac{\left(3+2x_1-x_1^2\right)-\left(3+2x_0-x_0^2\right)}{x_1-x_0}\\
&=\frac{3+2x_1-x_1^2-3-2x_0+x_0^2}{x_1-x_0}\\
&=\frac{\left(2x_1-2x_0\right)-\left(x_1^2-x_0^2\right)}{x_1-x_0}\\
&=\frac{2\left(x_1-x_0\right)-\left(x_1+x_0\right)\left(x_1-x_0\right)}{x_1-x_0}\tikzmark{factor by grouping}\\
&=\frac{\left[2-\left(x_1+x_0\right)\right]\left(x_1-x_0\right)}{x_1-x_0}\\
&=2-x_1-x_0\text{ for }x_1\neq x_0
\end{align*}
        \Annotate[][anchor=west]{0ex}{0.5ex}{1}{0}{factor by grouping}{\parbox{0.25\linewidth}{This factoring technique is called factoring by grouping.}}%
\par
We can check our formula using the line in \hyperref[figure-secant]{Figure~\ref*{figure-secant}}. If we let \(x_0=0\) and \(x_1=4\) then our simplified slope formula gives us:\begin{align*}
2-x_1-x_0&=2-4-0\\
&=-2\quad\checkmark
\end{align*}%
\end{example}
\typeout{************************************************}
\typeout{Exercises}
\typeout{************************************************}
\section*{Exercises}\label{exercises-2}

Let \(\fe{g}{x}=x^2-5\).%
\begin{exerciselist}
\item[1.]\phantomsection\hypertarget{exercise-7}{\null}Following \hyperref[example-secant]{Example~\ref*{example-secant}}, find a formula for the slope of the secant line connecting the points \(\point{x_0}{\fe{g}{x_0}}\) and \(\point{x_1}{\fe{g}{x_1}}\). Please note that factoring by grouping will \emph{not} be necessary when simplifying the expression.%
\par\smallskip
\item[2.]\phantomsection\hypertarget{exercise-8}{\null}Check your slope formula using the two points indicated in \hyperref[figure-secant-exercise]{Figure~\ref*{figure-secant-exercise}}. That is, use the graph to find the slope between the two points and then use your formula to find the slope; make sure that the two values agree!%
\begin{figure}
\centering
{
\begin{tikzpicture}
\begin{axis}[]
    \addplot+[
        domain=-3.45:3.45,
    <->,
    ]{x^2-5};
    \addplot+[
        domain=-6.9025:5.9025,
        <->,
    ]{x+1};
    \addplot[
        soldot
    ]coordinates{
        (-2,-1)
        (3,4)};
\end{axis}
\end{tikzpicture}
}
\caption{\(g\)\label{figure-secant-exercise}}
\end{figure}
\par\smallskip
\end{exerciselist}
\typeout{************************************************}
\typeout{Section 1.3 The Difference Quotient}
\typeout{************************************************}
\section[The Difference Quotient]{The Difference Quotient}\label{section-difference-quotient}
While it's easy to see that the formula \(\frac{\fe{f}{x_1}-\fe{f}{x_1}}{x_1-x_0}\) gives the slope of the line connecting two points on the function \(f\), the resultant expression can at times be awkward to work with. We actually already saw that when we had to use slight-of-hand factoring in \hyperref[example-secant]{Example~\ref*{example-secant}}.%
\par
The algebra associated with secant lines (and average velocities) can sometimes be simplified if we designate the variable \(h\) to be the run between the two points (or the length of the time interval). With this designation we have \(x_1-x_0=h\) which gives us \(x_1=x_0+h\). Making these substitutions we get \hyperref[equation-difference-quotient]{1.3.1}. The expression on the right side of \hyperref[equation-difference-quotient]{1.3.1} is called the \terminology{difference quotient} for \(f\).%
\begin{equation}\frac{\fe{f}{x_1}-\fe{f}{x_1}}{x_1-x_0}=\frac{\fe{f}{x_0+h}-\fe{f}{x_1}}{h}\label{equation-difference-quotient}\end{equation}\par
Let's revisit the function \(\fe{f}{x}=3+2x-x^2\) from \hyperref[example-secant]{Example~\ref*{example-secant}}. The difference quotient for this function is derived in \hyperref[example-difference-quotient]{Example~\ref*{example-difference-quotient}}.%
\begin{example}[Calculating a Difference Quotient]\label{example-difference-quotient}
\begin{align*}
\frac{\fe{f}{x_0+h}-\fe{f}{x_0}}{h}&=\frac{\left[3+2\left(x_0+h\right)-\left(x_0+h\right)^2\right]-\left[3+2x_0-x_0^2\right]}{h}\\
&=\frac{3+2x_0+2h-x_0^2-2x_0h-h^2-3-2x_0+x_0^2}{h}\\
&=\frac{2h-2x_0h-h^2}{h}\\
&=\frac{h\left(2-2x_0-h\right)}{h}\\
&=2-2x_0-h\text{ for }h\neq 0
\end{align*}Please notice that all of the terms without a factor of \(h\) subtracted to zero. Please notice, too, that we avoided all of the tricky factoring that appeared in \hyperref[example-secant]{Example~\ref*{example-secant}}!%
\end{example}
\par
For simplicity's sake, we generally drop the variable subscript when applying the difference quotient. So for future reference we will define the difference quotient as follows:%
\begin{definition}[The Difference Quotient]\label{definition-difference-quotient}
The \terminology{difference quotient} for the function \(y=\fe{f}{x}\) is the expression \(\frac{\fe{f}{x+h}-\fe{f}{x}}{h}\).%
\end{definition}
\typeout{************************************************}
\typeout{Exercises}
\typeout{************************************************}
\section*{Exercises}\label{exercises-3}

\begin{exercisegroup}%
Completely simplify the difference quotient for each of the following functions. Please note that the template for the difference quotient needs to be adapted to the function name and independent variable in each given equation. For example, the difference quotient for the function in \hyperref[exercise-first-difference-quotient]{Exercise~1} is \(\frac{\fe{v}{t+h}-\fe{v}{t}}{h}\).%
\par
Please make sure that you lay out your work in a manner consistent with the way the work is shown in \hyperref[example-difference-quotient]{Example~\ref*{example-difference-quotient}} (excluding the subscripts, of course).%
\begin{exercisegroupbycol}{3}%
\begin{exercisegroupitem}{1. }\phantomsection\hypertarget{exercise-first-difference-quotient}{\null}
\(\fe{v}{t}=2.5t^2-7.5t\)\end{exercisegroupitem}%
\par%
\begin{exercisegroupitem}{2. }\phantomsection\hypertarget{exercise-10}{\null}
\(\fe{g}{x}=3-7x\)\end{exercisegroupitem}%
\par%
\begin{exercisegroupitem}{3. }\phantomsection\hypertarget{exercise-11}{\null}
\(\fe{w}{x}=\frac{3}{x+2}\)\end{exercisegroupitem}%
\par%
\end{exercisegroupbycol}%
\end{exercisegroup}%
\begin{exercisegroup}%
Suppose that an object is tossed into the air in such a way that the elevation of the object (measured in \si{\foot}) is given by the function \(\fe{s}{t}=40+40t-16t^2\) where \(t\) is the amount of time that has passed since the object was tossed (measured in \si{\second}).%
\begin{exercisegroupbycol}{1}%
\begin{exercisegroupitem}{4. }\phantomsection\hypertarget{exercise-12}{\null}
Simplify the difference quotient for \(s\).%
\end{exercisegroupitem}%
\par%
\begin{exercisegroupitem}{5. }\phantomsection\hypertarget{exercise-difference-quotient-average-velocity}{\null}
Ignoring the unit, use the difference quotient to determine the average velocity over the interval \(\cinterval{1.6}{2.8}\). (\emph{Hint:} Use \(t=1.6\) and \(h=1.2\). Make sure that you understand why!)%
\end{exercisegroupitem}%
\par%
\begin{exercisegroupitem}{6. }\phantomsection\hypertarget{exercise-14}{\null}
What, \emph{including unit}, are the values of \(\fe{s}{1.6}\) and \(\fe{s}{2.8}\)? Use these values when working \hyperref[exercise-difference-quotient-verify]{Exercise~7}.%
\end{exercisegroupitem}%
\par%
\begin{exercisegroupitem}{7. }\phantomsection\hypertarget{exercise-difference-quotient-verify}{\null}
Use the expression \(\frac{\fe{s}{2.8}-\fe{s}{1.6}}{2.8-1.6}\) to verify the value you found in \hyperref[exercise-difference-quotient-average-velocity]{Exercise~5}. \emph{Include the unit while making this calculation.}%
\end{exercisegroupitem}%
\par%
\begin{exercisegroupitem}{8. }\phantomsection\hypertarget{exercise-16}{\null}
Ignoring the unit, use the difference quotient to determine the average velocity over the interval \(\cinterval{0.4}{2.4}\).%
\end{exercisegroupitem}%
\par%
\end{exercisegroupbycol}%
\end{exercisegroup}%
\begin{exercisegroup}%
Moose and squirrel were having casual conversation when suddenly, without any apparent provocation, Boris Badenov launched anti-moose missile in their direction. Fortunately, squirrel had ability to fly as well as great knowledge of missile technology, and he was able to disarm missile well before it hit ground.%
\par
The elevation (\si{\foot}) of the tip of the missile \(t\) seconds after it was launched is given by the function \(\fe{h}{t}=-16t^2+294.4t+15\).%
\begin{exercisegroupbycol}{1}%
\begin{exercisegroupitem}{9. }\phantomsection\hypertarget{exercise-17}{\null}
What, including unit, is the value of \(\fe{h}{12}\) and what does the value tell you about the flight of the missile?%
\end{exercisegroupitem}%
\par%
\begin{exercisegroupitem}{10. }\phantomsection\hypertarget{exercise-18}{\null}
What, including unit, is the value of \(\frac{\fe{h}{10\,\text{s}}-\fe{f}{0\,\text{s}}}{10\,\text{s}}\) and what does this value tell you about the flight of the missile?%
\end{exercisegroupitem}%
\par%
\begin{exercisegroupitem}{11. }\phantomsection\hypertarget{exercise-19}{\null}
The velocity (\si{\meter\per\second}) function for the missile is \(\fe{v}{t}=-32t+294.4\). What, including unit, is the value of \(\frac{\fe{v}{10\,\text{s}}-\fe{v}{0\,\text{s}}}{10\,\text{s}}\) and what does this value tell you about the flight of the missile?%
\end{exercisegroupitem}%
\par%
\end{exercisegroupbycol}%
\end{exercisegroup}%
\begin{exercisegroup}%
Timmy lived a long life in the 19th century. When Timmy was seven he found a rock that weighed exactly half a stone. (Timmy lived in jolly old England, don't you know.) That rock sat on Timmy's window sill for the next 80 years and wouldn't you know the weight of that rock did not change even one smidge the entire time. In fact, the weight function for this rock was \(\fe{w}{t}=0.5\) where \(\fe{w}{t}\) was the weight of the rock (stones) and \(t\) was the number of years that had passed since that day Timmy brought the rock home.%
\begin{exercisegroupbycol}{1}%
\begin{exercisegroupitem}{12. }\phantomsection\hypertarget{exercise-20}{\null}
What was the average rate of change in the weight of the rock over the 80 years it sat on Timmy's window sill?%
\end{exercisegroupitem}%
\par%
\begin{exercisegroupitem}{13. }\phantomsection\hypertarget{exercise-21}{\null}
Ignoring the unit, simplify the expression \(\frac{\fe{w}{t_1}-\fe{w}{t_0}}{t_1-t_0}\). Does the result make sense in the context of this problem?%
\end{exercisegroupitem}%
\par%
\begin{exercisegroupitem}{14. }\phantomsection\hypertarget{exercise-22}{\null}
Showing each step in the process and ignoring the unit, simplify the difference quotient for \(w\). Does the result make sense in the context of this problem?%
\end{exercisegroupitem}%
\par%
\end{exercisegroupbycol}%
\end{exercisegroup}%
\begin{exercisegroup}%
Truth be told, there was one day in 1842 when Timmy's mischievous son Nigel took that rock outside and chucked it into the air. The velocity of the rock (\si{\foot\per\second}) was given by \(\fe{v}{t}=60-32t\) where \(t\) was the number of seconds that had passed since Nigel chucked the rock.%
\begin{exercisegroupbycol}{1}%
\begin{exercisegroupitem}{15. }\phantomsection\hypertarget{exercise-23}{\null}
What, \emph{including unit}, are the values of \(\fe{v}{0}\), \(\fe{v}{1}\), and \(\fe{v}{2}\) and what do these values tell you in the context of this problem? Don't just write that the values tell you the velocity at certain times; explain what the velocity values tell you about the motion of the rock.%
\end{exercisegroupitem}%
\par%
\begin{exercisegroupitem}{16. }\phantomsection\hypertarget{exercise-24}{\null}
Ignoring the unit, simplify the difference quotient for \(v\).%
\end{exercisegroupitem}%
\par%
\begin{exercisegroupitem}{17. }\phantomsection\hypertarget{exercise-25}{\null}
 What is the unit for the difference quotient for \(v\)? What does the value of the difference quotient (including unit) tell you in the context of this problem?%
\end{exercisegroupitem}%
\par%
\end{exercisegroupbycol}%
\end{exercisegroup}%
\begin{exercisegroup}%
Suppose that a vat was undergoing a controlled drain and that the amount of fluid left in the vat (\si{\gallon}) was given by the formula \(\fe{V}{t}=100-2t^{3/2}\) where \(t\) is the number of minutes that had passed since the draining process began.%
\begin{exercisegroupbycol}{1}%
\begin{exercisegroupitem}{18. }\phantomsection\hypertarget{exercise-vat-first}{\null}
What, \emph{including unit}, is the value of \(\fe{V}{4}\) and what does that value tell you in the context of this problem?%
\end{exercisegroupitem}%
\par%
\begin{exercisegroupitem}{19. }\phantomsection\hypertarget{exercise-27}{\null}
Ignoring the unit, write down the formula you get for the difference quotient of \(V\) when \(t=4\). Copy \hyperref[table-vat]{Table~\ref*{table-vat}} onto your paper and fill in the missing values. \emph{Look for a pattern in the output and write down enough digits for each entry so that the pattern is clearly illustrated;} the first two entries in the output column have been given to help you understand what is meant by this direction.%
\begin{table}
\centering
\caption{\(y=\frac{\fe{V}{4+h}-\fe{V}{4}}{h}\)\label{table-vat}}
\begin{tabular}{ll}\hrulethick
\multicolumn{1}{c}{\(h\)}&\multicolumn{1}{c}{\(y\)}\\\hrulemedium
\(-0.1\)&\(-5.962\ldots\)\\
\(-0.01\)&\(-5.9962\ldots\)\\
\(-0.001\)&\\
\(\phantom{-}0.001\)&\\
\(\phantom{-}0.01\)&\\
\(\phantom{-}0.1\)&
\end{tabular}
\end{table}
\end{exercisegroupitem}%
\par%
\begin{exercisegroupitem}{20. }\phantomsection\hypertarget{exercise-28}{\null}
What is the unit for the \(y\) values in \hyperref[table-vat]{Table~\ref*{table-vat}}? What do these values (with their unit) tell you in the context of this problem?%
\end{exercisegroupitem}%
\par%
\begin{exercisegroupitem}{21. }\phantomsection\hypertarget{exercise-vat-last}{\null}
As the value of \(h\) gets closer to \(0\), the values in the \(y\) column of \hyperref[table-vat]{Table~\ref*{table-vat}} appear to be converging to a single number; what is this number and what do you think that value (with its unit) tells you in the context of this problem?%
\end{exercisegroupitem}%
\par%
\end{exercisegroupbycol}%
\end{exercisegroup}%
\typeout{************************************************}
\typeout{Chapter 2 Limits and Continuity}
\typeout{************************************************}
\chapter[Limits and Continuity]{Limits and Continuity}\label{chapter-limits}
\typeout{************************************************}
\typeout{Section 2.1 Limits}
\typeout{************************************************}
\section[Limits]{Limits}\label{section-limits}
While working \hyperref[exercise-vat-first]{Exercises~18} through \hyperref[exercise-vat-last]{21} from \hyperref[section-difference-quotient]{Section~\ref*{section-difference-quotient}} you completed \hyperref[table-vat]{Table~\ref*{table-vat}}. In the context of that problem the difference quotient being evaluated returned the average rate of change in the volume of fluid remaining in a vat between times \(t=4\) and \(t=4+h\). As the elapsed time closes in on \(0\) this average rate of change converges to \(-6\). From that we deduce that the rate of change in the volume \(4\) minutes into the draining process must have been \SI{6}{\gallon\per\minute}.%
\par
Please note that we could not deduce the rate of change \(4\) minutes into the process by replacing \(h\) with \(0\); in fact, there are at least two things preventing us from doing so. From a strictly mathematical perspective, we cannot replace \(h\) with \(0\) because that would lead to division by zero in the difference quotient. From a more physical perspective, replacing \(h\) with \(0\) would in essence stop the clock. If time is frozen, so is the amount of fluid in the vat and the entire concept of ``rate of change'' becomes moot.%
\par
It turns out that it is frequently more useful (not to mention interesting) to explore the \emph{trend} in a function as the input variable \emph{approaches} a number rather than the actual value of the function at that number. Mathematically we describe these trends using \terminology{limits}.%
\par
If we call the difference quotient in the heading for \hyperref[table-vat]{Table~\ref*{table-vat}} \(\fe{f}{h}\), then we could describe the trend evidenced in the table by saying ``the limit of \(\fe{f}{h}\) as \(h\) approaches zero is \(-6\).'' Please note that as \(h\) changes value, the value of \(\fe{f}{h}\) changes, not the value of the limit. The limit value is a fixed number to which the value of \(\fe{f}{h}\) converges. Symbolically we write \(\lim\limits_{h\to0}\fe{f}{h}=-6\).%
\par
Most of the time the value of a function at the number \(a\) and the limit of the function as \(x\) approaches \(a\) are in fact the same number. When this occurs we say that the function is \terminology{continuous} at \(a\). However, to help you better understand the concept of limit we need to have you confront situations where the function value and limit value are not equal to one another. Graphs can be useful for helping distinguish function values from limit values, so that is the perspective you are going to use in the first couple of problems in this lab.%
\typeout{************************************************}
\typeout{Exercises}
\typeout{************************************************}
\section*{Exercises}\label{exercises-4}

\begin{exerciselist}
\item[1.]\phantomsection\hypertarget{exercise-30}{\null}Several function values and limit values for the function in \hyperref[figure-first-limit]{Figure~\ref*{figure-first-limit}} are given below. You and your group mates should take turns reading the equations aloud. Make sure that you read the symbols correctly, that's part of what you are learning! Also, discuss why the values are what they are and make sure that you get help from your instructor to clear up any confusion.%
\par
\begin{align*}
\fe{f}{-2}&=6&\text{but}&&\lim_{x\to-2}\fe{f}{x}&=3\\
\fe{f}{-4}&\text{ is undefined}&\text{but}&&\lim_{x\to-4}\fe{f}{x}&=2\\
\fe{f}{1}&=-1&\text{but}&&\lim_{x\to1}\fe{f}{x}&\text{ does not exist}\\
\underbrace{\lim_{x\to1^{-}}\fe{f}{x}}_{\begin{array}{c}\text{the limit of }\fe{f}{x}\\\text{as }x\text{ approaches }1\\\text{from the left}\end{array}}&=-3&\text{but}&&\underbrace{\lim_{x\to1^{+}}\fe{f}{x}}_{\begin{array}{c}\text{the limit of }\fe{f}{x}\\\text{as }x\text{ approaches }1\\\text{from the right}\end{array}}&=-1
\end{align*}%
\begin{figure}
\centering
{
\begin{tikzpicture}
\begin{axis}[]
    \addplot[
       pccplot,
        ->
        ]coordinates{
            (1,-3)
            (-3,5)
            (-6.9,-6.7)};
    \addplot[
        pccplot,
        ->
        ]coordinates{
            (1,-1)
            (6.9,4.9)};
    \addplot[
        soldot,
        ]coordinates{
            (-2,6)
            (1,-1)};
    \addplot[
        holdot,
        ]coordinates{
            (-4,2)
            (-2,3)
            (1,-3)};
\end{axis}
\end{tikzpicture}
}
\caption{\(f\)\label{figure-first-limit}}
\end{figure}
\par\smallskip
\end{exerciselist}
\begin{exercisegroup}%
Copy each of the following expressions onto your paper and either state the value or state that the value is undefined or doesn't exist. Make sure that when discussing the values you use proper terminology. All expressions are in reference to the function \(g\) shown in \hyperref[figure-second-limit]{Figure~\ref*{figure-second-limit}}.%
\begin{figure}
\centering
{
\begin{tikzpicture}
\begin{axis}[xlabel = {$t$},]
    \addplot[
        pccplot,
        ->,
        ]coordinates{
            (-2,5)
            (-6.9,-2.35)};
    \addplot[
        pccplot,
        -,
        ]coordinates{
            (-2,-3.5)
            (3,-1)};
    \addplot[
        pccplot,
        ->,
        domain=3:6.9,
        ]{(x-4)^2-2};
    \addplot[
        soldot,
        ]coordinates{
            (-2,5)
            (5,1)};
    \addplot[
        holdot,
        ]coordinates{
            (-2,-3.5)
            (2,-1.5)
            (5,-1)};
\end{axis}
\end{tikzpicture}
}
\caption{\(g\)\label{figure-second-limit}}
\end{figure}
\begin{exercisegroupbycol}{4}%
\begin{exercisegroupitem}{2. }\phantomsection\hypertarget{exercise-31}{\null}
\(\fe{g}{5}\)%
\end{exercisegroupitem}%
\par%
\begin{exercisegroupitem}{3. }\phantomsection\hypertarget{exercise-32}{\null}
\(\lim\limits_{t\to5}\fe{g}{t}\)%
\end{exercisegroupitem}%
\par%
\begin{exercisegroupitem}{4. }\phantomsection\hypertarget{exercise-33}{\null}
\(\fe{g}{3}\)%
\end{exercisegroupitem}%
\par%
\begin{exercisegroupitem}{5. }\phantomsection\hypertarget{exercise-34}{\null}
\(\lim\limits_{t\to3^{-}}\fe{g}{t}\)%
\end{exercisegroupitem}%
\par%
\begin{exercisegroupitem}{6. }\phantomsection\hypertarget{exercise-35}{\null}
\(\lim\limits_{t\to3^{+}}\fe{g}{t}\)%
\end{exercisegroupitem}%
\par%
\begin{exercisegroupitem}{7. }\phantomsection\hypertarget{exercise-36}{\null}
\(\lim\limits_{t\to3}\fe{g}{t}\)%
\end{exercisegroupitem}%
\par%
\begin{exercisegroupitem}{8. }\phantomsection\hypertarget{exercise-37}{\null}
\(\fe{g}{2}\)%
\end{exercisegroupitem}%
\par%
\begin{exercisegroupitem}{9. }\phantomsection\hypertarget{exercise-38}{\null}
\(\lim\limits_{t\to2}\fe{g}{t}\)%
\end{exercisegroupitem}%
\par%
\begin{exercisegroupitem}{10. }\phantomsection\hypertarget{exercise-39}{\null}
\(\fe{g}{-2}\)%
\end{exercisegroupitem}%
\par%
\begin{exercisegroupitem}{11. }\phantomsection\hypertarget{exercise-40}{\null}
\(\lim\limits_{t\to-2^{-}}\fe{g}{t}\)%
\end{exercisegroupitem}%
\par%
\begin{exercisegroupitem}{12. }\phantomsection\hypertarget{exercise-41}{\null}
\(\lim\limits_{t\to-2^{+}}\fe{g}{t}\)%
\end{exercisegroupitem}%
\par%
\begin{exercisegroupitem}{13. }\phantomsection\hypertarget{exercise-42}{\null}
\(\lim\limits_{t\to-2}\fe{g}{t}\)%
\end{exercisegroupitem}%
\par%
\end{exercisegroupbycol}%
\end{exercisegroup}%
\begin{exercisegroup}%
Values of the function \(f\), where \(\fe{f}{x}=\frac{3x^2-16x+5}{2x^2-13x+15}\) are shown in \hyperref[table-rational-function-values]{Table~\ref*{table-rational-function-values}}. Both of the questions below are in reference to this function.%
\begin{table}
\centering
\caption{\(\fe{f}{x}=\frac{3x^2-16x+5}{2x^2-13x+15}\)\label{table-rational-function-values}}
\begin{tabular}{ll}\hrulethick
\multicolumn{1}{c}{\(x\)}&\multicolumn{1}{c}{\(\fe{f}{x}\)}\\\hrulemedium
\(4.99\)&\(2.0014\)\\
\(4.999\)&\(2.00014\)\\
\(4.9999\)&\(2.000014\)\\
\(5.0001\)&\(1.999986\)\\
\(5.001\)&\(1.99986\)\\
\(5.01\)&\(1.9986\)
\end{tabular}
\end{table}
\begin{exercisegroupbycol}{1}%
\begin{exercisegroupitem}{14. }\phantomsection\hypertarget{exercise-43}{\null}
What is the value of \(\fe{f}{5}\)?%
\end{exercisegroupitem}%
\par%
\begin{exercisegroupitem}{15. }\phantomsection\hypertarget{exercise-44}{\null}
What is the value of \(\lim\limits_{x\to5}\frac{3x^2-16x+5}{2x^2-13x+15}\)?%
\end{exercisegroupitem}%
\par%
\end{exercisegroupbycol}%
\end{exercisegroup}%
\begin{exercisegroup}%
Values of the function \(p\) where \(\fe{p}{t}=\sqrt{t-12}\) are shown in \hyperref[table-square-root-values]{Table~\ref*{table-square-root-values}}. Both of the questions below are in reference to this function.%
\begin{table}
\centering
\caption{\(\fe{p}{t}=\sqrt{t-12}\)\label{table-square-root-values}}
\begin{tabular}{ll}\hrulethick
\multicolumn{1}{c}{\(t\)}&\multicolumn{1}{c}{\(\fe{p}{t}\)}\\\hrulemedium
\(20.9\)&\(2.98\ldots\)\\
\(20.99\)&\(2.998\ldots\)\\
\(20.999\)&\(2.9998\ldots\)\\
\(21.001\)&\(3.0002\ldots\)\\
\(21.01\)&\(3.002\ldots\)\\
\(21.1\)&\(3.02\ldots\)
\end{tabular}
\end{table}
\begin{exercisegroupbycol}{1}%
\begin{exercisegroupitem}{16. }\phantomsection\hypertarget{exercise-45}{\null}
What is the value of \(\fe{p}{21}\)?%
\end{exercisegroupitem}%
\par%
\begin{exercisegroupitem}{17. }\phantomsection\hypertarget{exercise-46}{\null}
What is the value of \(\lim\limits_{t\to21}\sqrt{t-12}\)?%
\end{exercisegroupitem}%
\par%
\end{exercisegroupbycol}%
\end{exercisegroup}%
\begin{exercisegroup}%
Create tables similar to \hyperref[table-rational-function-values]{Tables~\ref*{table-rational-function-values}} and \hyperref[table-square-root-values]{2.1.19} from which you can deduce each of the following limit values. Make sure that you include table numbers, table captions, and meaningful column headings. Make sure that your input values follow patterns similar to those used in \hyperref[table-rational-function-values]{Tables~\ref*{table-rational-function-values}} and \hyperref[table-square-root-values]{2.1.19}. Make sure that you round your output values in such a way that a clear and compelling pattern in the output is clearly demonstrated by your stated values. Make sure that you state the limit value!%
\begin{exercisegroupbycol}{3}%
\begin{exercisegroupitem}{18. }\phantomsection\hypertarget{exercise-47}{\null}
\(\lim\limits_{t\to6}\frac{t^2-10t+24}{t-6}\)%
\end{exercisegroupitem}%
\par%
\begin{exercisegroupitem}{19. }\phantomsection\hypertarget{exercise-48}{\null}
\(\lim\limits_{x\to-1^{+}}\frac{\sin(x+1)}{3x+3}\)%
\end{exercisegroupitem}%
\par%
\begin{exercisegroupitem}{20. }\phantomsection\hypertarget{exercise-49}{\null}
\(\lim\limits_{h\to0^{-}}\frac{h}{4-\sqrt{16-h}}\)%
\end{exercisegroupitem}%
\par%
\end{exercisegroupbycol}%
\end{exercisegroup}%
\typeout{************************************************}
\typeout{Section 2.2 Limits Laws}
\typeout{************************************************}
\section[Limits Laws]{Limits Laws}\label{section-limit-laws}
When proving the value of a limit we frequently rely upon laws that are easy to prove using the technical definitions of limit. These laws can be found in {$\langle\langle$Appendix C$\rangle\rangle$}. The first of these type laws are called replacement laws. Replacement laws allow us to replace limit expressions with the actual values of the limits.%
\typeout{************************************************}
\typeout{Exercises}
\typeout{************************************************}
\section*{Exercises}\label{exercises-5}

\begin{exercisegroup}%
The value of each of the following limits can be established using one of the replacement laws. Copy each limit expression onto your own paper, state the value of the limit (e.g.\@ \(\lim\limits_{x\to9}5=5\)), and state the replacement law (by number) that establishes the value of the limit.%
\begin{exercisegroupbycol}{3}%
\begin{exercisegroupitem}{1. }\phantomsection\hypertarget{exercise-50}{\null}
\(\lim\limits_{t\to\pi}t\)%
\end{exercisegroupitem}%
\par%
\begin{exercisegroupitem}{2. }\phantomsection\hypertarget{exercise-51}{\null}
\(\lim\limits_{x\to14}14\)%
\end{exercisegroupitem}%
\par%
\begin{exercisegroupitem}{3. }\phantomsection\hypertarget{exercise-52}{\null}
\(\lim\limits_{x\to14}x\)%
\end{exercisegroupitem}%
\par%
\end{exercisegroupbycol}%
\end{exercisegroup}%
\begin{example}[Applying Limit Laws]\label{example-apply-limit-laws}
\begin{align*}
\lim_{x\to7}\left(4x^2+3\right)&=\lim_{x\to7}\left(4x^2\right)+\lim_{x\to7}3&&\text{Limit Law A1}\\
&=4\lim_{x\to7}x^2+\lim_{x\to7}3&&\text{Limit Law A3}\\
&=4\left(\lim_{x\to7}x\right)^2+\lim_{x\to7}3&&\text{Limit Law A6}\\
&=4\cdot7^2+3&&\text{Limit Laws R1 and R2}\\
&=199
\end{align*}%
\end{example}
\begin{exercisegroup}%
The algebraic limit laws allow us to replace limit expressions with equivalent limit expressions. When applying limit laws our first goal is to come up with an expression in which every limit in the expression can be replaced with its value based up on one of the replacement laws. This process is shown in example \hyperref[example-apply-limit-laws]{Example~\ref*{example-apply-limit-laws}}. Please note that all replacement laws are saved for the second to last step and that each replacement is explicitly shown. Please note also that each limit law used is referenced by number.%
\par
Use the limit laws to establish the value of each of the following limits. Make sure that you use the step-by-step, vertical format shown in example \hyperref[example-apply-limit-laws]{Example~\ref*{example-apply-limit-laws}}. Make sure that you cite the limit laws used in each step. To help you get started, the steps necessary in \hyperref[exercise-first-apply-limit-laws]{Exercise~4} are outlined as:%
\begin{multicols}{2}
\begin{enumerate}[label=(\alph*)]
\item{}Apply Law A6\item{}Apply Law A1\item{}Apply Law A3\item{}Apply Laws R1 and R2\end{enumerate}
\end{multicols}
\begin{exercisegroupbycol}{3}%
\begin{exercisegroupitem}{4. }\phantomsection\hypertarget{exercise-first-apply-limit-laws}{\null}
\(\lim\limits_{t\to4}\sqrt{6t+1}\)%
\end{exercisegroupitem}%
\par%
\begin{exercisegroupitem}{5. }\phantomsection\hypertarget{exercise-54}{\null}
\(\lim\limits_{y\to7}\frac{y+3}{y-\sqrt{y+9}}\)%
\end{exercisegroupitem}%
\par%
\begin{exercisegroupitem}{6. }\phantomsection\hypertarget{exercise-55}{\null}
\(\lim\limits_{x\to\pi}x\cos(x)\)%
\end{exercisegroupitem}%
\par%
\end{exercisegroupbycol}%
\end{exercisegroup}%
\typeout{************************************************}
\typeout{Section 2.3 Indeterminate Limits}
\typeout{************************************************}
\section[Indeterminate Limits]{Indeterminate Limits}\label{section-indeterminate-limits}
Many limits have the form \(\frac{0}{0}\) which means the expressions in both the numerator and denominator limit to zero (e.g.\@\(\lim\limits_{x\to3}\frac{2x-6}{x-3}\)). The form \(\frac{0}{0}\) is called \terminology{indeterminate} because we do not know the value of the limit (or even if it exists) so long as the limit has that form. When confronted with limits of form \(\frac{0}{0}\) we must first manipulate the expression so that common factors causing the zeros in the numerator and denominator are isolated. Limit law A7 can then be used to justify eliminating the common factors and once they are gone we may proceed with the application of the remaining limit laws. \hyperref[example-first-indeterminate]{Examples~\ref*{example-first-indeterminate}} and \hyperref[example-second-indeterminate]{2.3.2} illustrate this situation.%
\begin{example}\label{example-first-indeterminate}
\begin{align*}
\lim_{x\to3}\frac{x^2-8x+15}{x-3}&=\lim_{x\to3}\frac{(x-5)(x-3)}{x-3}\\
&=\tikzmark{not indeterminate}\lim_{x\to3}(x-5)&&\text{Limit Law A7}\\
&=\lim_{x\to3}x-\lim_{x\to3}5&&\text{Limit Law A2}\\
&=3-5&&\text{Limit Laws R1 and R2}\\
&=-2
\end{align*}\Annotate[][anchor=north east]{0ex}{0}{-1}{-0.5}{not indeterminate}{\parbox{0.3\linewidth}{This limit does not have indeterminate form, so we may apply the remaining limit laws.}}%
\end{example}
\begin{example}\label{example-second-indeterminate}
\begin{align*}
\lim_{\theta\to0}\frac{1-\fe{\cos}{\theta}}{\fe{\sin^2}{\theta}}&=\lim_{\theta\to0}\left(\frac{1-\fe{\cos}{\theta}}{\fe{\sin^2}{\theta}}\cdot\frac{1+\fe{\cos}{\theta}}{1+\fe{\cos}{\theta}}\right)\\
&=\lim_{\theta\to0}\frac{1-\fe{\cos^2}{\theta}}{\fe{\sin^2}{\theta}\cdot\left(1+\fe{\cos}{\theta}\right)}\\
&=\lim_{\theta\to0}\frac{\fe{\sin^2}{\theta}}{\fe{\sin^2}{\theta}\cdot\left(1+\fe{\cos}{\theta}\right)}\\
&=\tikzmark{second not indeterminate}\lim_{\theta\to0}\frac{1}{1+\fe{\cos}{\theta}}&&\text{Limit Law A7}\\
&=\frac{\lim_{\theta\to0}1}{\lim_{\theta\to0}\left(1+\fe{\cos}{\theta}\right)}&&\text{Limit Law A5}\\
&=\frac{\lim_{\theta\to0}1}{\lim_{\theta\to0}1+\lim_{\theta\to0}\fe{\cos}{\theta}}&&\text{Limit Law A1}\\
&=\frac{\lim_{\theta\to0}1}{\lim_{\theta\to0}1+\fe{\cos}{\lim_{\theta\to0}\theta}}&&\text{Limit Law A6}\\
&=\frac{1}{1+\fe{\cos}{0}}&&\text{Limit Laws R1 and R2}\\
&=\frac{1}{2}
\end{align*}\Annotate[][anchor=north east]{0ex}{0}{-1}{-0.5}{second not indeterminate}{\parbox{0.3\linewidth}{This limit does not have indeterminate form, so we may apply the remaining limit laws.}}%
\end{example}
\par
As seen in \hyperref[example-second-indeterminate]{Example~\ref*{example-second-indeterminate}}, trigonometric identities can come into play while trying to eliminate the form \(\frac{0}{0}\). Elementary rules of logarithms can also play a role in this process. Before you begin evaluating limits whose initial form is \(\frac{0}{0}\), you need to make sure that you recall some of these basic rules. That is the purpose of \hyperref[exercise-identities-review]{Exercise~1}.%
\typeout{************************************************}
\typeout{Exercises}
\typeout{************************************************}
\section*{Exercises}\label{exercises-6}

\begin{exerciselist}
\item[1.]\phantomsection\hypertarget{exercise-identities-review}{\null}Complete each of the following identities (over the real numbers). Make sure that you check with your lecture instructor so that you know which of these identities you are expected to memorize.%
\par
The following identities are valid for all values of \(x\) and \(y\).\begin{align*}
1-\fe{\cos^2}{x}&=&\fe{\tan^2}{x}&=\\
\fe{\sin}{2x}&=&\fe{\tan}{2x}&=\\
\fe{\sin}{x+y}&=&\fe{\cos}{x+y}&=\\
\fe{\sin}{\frac{x}{2}}&=&\fe{\cos}{\frac{x}{2}}&=
\end{align*}%
\par
There are three versions of the following identity; write them all.\begin{align*}
\fe{\cos}{2x}&=&\fe{\cos}{2x}&=&\fe{\cos}{2x}&=
\end{align*}%
\par
The following identities are valid for all positive values of \(x\) and \(y\) and all values of \(n\).\begin{align*}
\fe{\ln}{xy}&=&\fe{\ln}{\frac{x}{y}}&=\\
\fe{\ln}{x^n}&=&\fe{\ln}{e^n}&=
\end{align*}%
\par\smallskip
\end{exerciselist}
\begin{exercisegroup}%
Use the limit laws to establish the value of each of the following limits after first manipulating the expression so that it
no longer has form \(\frac{0}{0}\). Make sure that you use the step-by-step, vertical format shown in \hyperref[example-first-indeterminate]{Examples~\ref*{example-first-indeterminate}} and \hyperref[example-second-indeterminate]{Example~\ref*{example-second-indeterminate}}. Make sure that you cite each limit law used.%
\begin{exercisegroupbyrow}{3}%
\begin{exercisegroupitem}{2. }\phantomsection\hypertarget{exercise-57}{\null}
\(\lim\limits_{x\to-4}\frac{x+4}{2x^2+5x-12}\)%
\end{exercisegroupitem}%
\begin{exercisegroupitem}{3. }\phantomsection\hypertarget{exercise-58}{\null}
\(\lim\limits_{x\to0}\frac{\fe{\sin}{2x}}{\fe{\sin}{x}}\)%
\end{exercisegroupitem}%
\begin{exercisegroupitem}{4. }\phantomsection\hypertarget{exercise-59}{\null}
\(\lim\limits_{\beta\to0}\frac{\fe{\sin}{\beta+\pi}}{\fe{\sin}{\beta}}\)%
\end{exercisegroupitem}%
\par%
\begin{exercisegroupitem}{5. }\phantomsection\hypertarget{exercise-60}{\null}
\(\lim\limits_{t\to0}\frac{\fe{\cos}{2t}-1}{\fe{\cos}{t}-1}\)%
\end{exercisegroupitem}%
\begin{exercisegroupitem}{6. }\phantomsection\hypertarget{exercise-61}{\null}
\(\lim\limits_{x\to1}\frac{4\fe{\ln}{x}+2\fe{\ln}{x^3}}{\fe{\ln}{x}-\fe{\ln}{\sqrt{x}}}\)%
\end{exercisegroupitem}%
\begin{exercisegroupitem}{7. }\phantomsection\hypertarget{exercise-62}{\null}
\(\lim\limits_{w\to9}\frac{9-w}{\sqrt{w}-3}\)%
\end{exercisegroupitem}%
\par%
\end{exercisegroupbyrow}%
\end{exercisegroup}%
\typeout{************************************************}
\typeout{Section 2.4 Limits at Infinity}
\typeout{************************************************}
\section[Limits at Infinity]{Limits at Infinity}\label{section-limits-at-infinity}
We are frequently interested in a function's ``end behavior''. That is, what is the behavior of the function as the input variable increases without bound or decreases without bound.%
\par
Many times a function will approach a horizontal asymptote as its end behavior. Assuming that the horizontal asymptote \(y=L\) represents the end behavior of the function \(f\) both as \(x\) increases without bound and as \(x\) decreases into the negative without bound, we write \(\lim\limits_{x\to\infty}\fe{f}{x}=L\) and \(\lim\limits_{x\to\infty}\fe{f}{x}=L\).%
\par
The formalistic way to read \(\lim\limits_{x\to\infty}\fe{f}{x}=L\) is ``the limit of \(\fe{f}{x}\) as \(x\) approaches infinity equals \(L\)''. When read that way, however, the words need to be taken \emph{anything but literally}. In the first place, \(x\) isn't approaching anything! The entire point is that \(x\) is increasing without any bound on how large its value becomes. Secondly, there is no place on the real number line called ``infinity''; infinity is not a number. Hence \(x\) certainly can't be approaching something that isn’t even there!%
\typeout{************************************************}
\typeout{Exercises}
\typeout{************************************************}
\section*{Exercises}\label{exercises-7}

\begin{exerciselist}
\item[1.]\phantomsection\hypertarget{exercise-63}{\null}For the function in {$\langle\langle$Figure 7.1 (Appendix B, page B1)$\rangle\rangle$} we could (correctly) write \(\lim\limits_{x\to\infty}\fe{f_1}{x}=-2\) and \(\lim\limits_{x\to-\infty}\fe{f_1}{x}=-2\). Go ahead and write (and say aloud) the analogous limits for the functions in {$\langle\langle$Figures 7.2-7.5 (pages B1 and B2)$\rangle\rangle$}.%
\par\smallskip
\end{exerciselist}
\begin{table}
\centering
\caption{\(\fe{f}{x}=\frac{3x^2-16x+5}{2x^2-13x+15}\)\label{table-limit-at-minus-infinity}}
\begin{tabular}{rl}\hrulethick
\multicolumn{1}{c}{\(x\)}&\multicolumn{1}{c}{\(\fe{f}{x}\)}\\\hrulemedium
\(-1000\)&\(1.498\ldots\)\\
\(-10{,}000\)&\(1.4998\ldots\)\\
\(-100{,}000\)&\(1.49998\ldots\)\\
\(-1{,}000{,}000\)&\(1.499998\ldots\)
\end{tabular}
\end{table}
\begin{exercisegroup}%
Values of the function \(f\) defined by \(\fe{f}{x}=\frac{3x^2-16x+5}{2x^2-13x+15}\) are shown in \hyperref[table-limit-at-minus-infinity]{Table~\ref*{table-limit-at-minus-infinity}}. Both of the questions below are in reference to this function.%
\begin{exercisegroupbycol}{1}%
\begin{exercisegroupitem}{2. }\phantomsection\hypertarget{exercise-64}{\null}
What is the value of \(\lim\limits_{x\to-\infty}\fe{f}{x}\)?%
\end{exercisegroupitem}%
\par%
\begin{exercisegroupitem}{3. }\phantomsection\hypertarget{exercise-65}{\null}
What is the equation of the horizontal asymptote for the graph of \(y=\fe{f}{x}\)?%
\end{exercisegroupitem}%
\par%
\end{exercisegroupbycol}%
\end{exercisegroup}%
\begin{table}
\centering
\caption{\(\fe{g}{t}=1+\frac{\fe{\sin}{t}}{t}\)\label{table-limit-crossing-asymptote}}
\begin{tabular}{ll}\hrulethick
\multicolumn{1}{c}{\(t\)}&\multicolumn{1}{c}{\(\fe{g}{t}\)}\\\hrulemedium
\(10^3\)&\(1.008\ldots\)\\
\(10^4\)&\(0.9997\ldots\)\\
\(10^5\)&\(1.0000004\ldots\)\\
\(10^6\)&\(0.9999997\ldots\)\\
\(10^7\)&\(1.00000004\ldots\)\\
\(10^8\)&\(1.000000009\ldots\)\\
\(10^9\)&\(1.0000000005\ldots\)\\
\(10^{10}\)&\(0.99999999995\ldots\)
\end{tabular}
\end{table}
\begin{exercisegroup}%
Jorge and Vanessa were in a heated discussion about horizontal asymptotes. Jorge said that functions never cross horizontal asymptotes. Vanessa said Jorge was nuts. Vanessa whipped out her trusty calculator and generated the values in \hyperref[table-limit-crossing-asymptote]{Table~\ref*{table-limit-crossing-asymptote}} to prove her point.%
\begin{exercisegroupbycol}{1}%
\begin{exercisegroupitem}{4. }\phantomsection\hypertarget{exercise-66}{\null}
What is the value of \(\lim\limits_{t\to\infty}\fe{g}{t}\)?%
\end{exercisegroupitem}%
\par%
\begin{exercisegroupitem}{5. }\phantomsection\hypertarget{exercise-67}{\null}
What is the equation of the horizontal asymptote for the graph of \(y=\fe{g}{t}\)?%
\end{exercisegroupitem}%
\par%
\begin{exercisegroupitem}{6. }\phantomsection\hypertarget{exercise-68}{\null}
Just how many times does the curve \(y=\fe{g}{t}\) cross its horizontal asymptote?%
\end{exercisegroupitem}%
\par%
\end{exercisegroupbycol}%
\end{exercisegroup}%
\typeout{************************************************}
\typeout{Section 2.5 Limits at Infinity Tending to Zero}
\typeout{************************************************}
\section[Limits at Infinity Tending to Zero]{Limits at Infinity Tending to Zero}\label{section-limits-at-infinity-tending-to-zero}
When using limit laws to establish limit values as \(x\to\infty\) or \(x\to-\infty\), limit laws A1-A6 and R2 are still in play (when applied in a valid manner), but limit law R1 cannot be applied. (The reason limit law R1 cannot be applied is discussed in detail in \hyperref[exercise-hear-me-first]{Exercises~11} through \hyperref[exercise-hear-me-last]{18} from \hyperref[section-vertical-asymptotes]{Section~\ref*{section-vertical-asymptotes}}.)%
\par
There is a new replacement law that can only be applied when \(x\to\infty\) or \(x\to-\infty\); this is replacement law R3. Replacement law R3 essentially says that if the value of a function is increasing without any bound on how large it becomes or if the function is decreasing without any bound on how large its absolute value becomes, then the value of a constant divided by that function must be approaching zero. An analogy can be found in extremely poor party planning. Let's say that you plan to have a pizza party and you buy five pizzas. Suppose that as the hour of the party approaches more and more guests come in the door\textemdash{}in fact the guests never stop coming! Clearly as the number of guests continues to rise the amount of pizza each guest will receive quickly approaches zero (assuming the pizzas are equally divided among the guests).%
\typeout{************************************************}
\typeout{Exercises}
\typeout{************************************************}
\section*{Exercises}\label{exercises-8}

\begin{exerciselist}
\item[1.]\phantomsection\hypertarget{exercise-69}{\null}Consider the function \(f\) defined by \(\fe{f}{x}=\frac{12}{x}\). Complete \hyperref[table-limit-to-zero]{Table~\ref*{table-limit-to-zero}} without the use of your calculator. What limit value and limit law are being illustrated in the table?%
\begin{table}
\centering
\caption{\(\fe{f}{x}=\frac{12}{x}\)\label{table-limit-to-zero}}
\begin{tabular}{rl}\hrulethick
\multicolumn{1}{c}{\(x\)}&\multicolumn{1}{c}{\(\fe{f}{x}\)}\\\hrulemedium
\(1000\)&\\
\(10{,}000\)&\\
\(100{,}000\)&\\
\(1{,}000{,}000\)&
\end{tabular}
\end{table}
\par\smallskip
\end{exerciselist}
\typeout{************************************************}
\typeout{Section 2.6 Ratios of Infinities}
\typeout{************************************************}
\section[Ratios of Infinities]{Ratios of Infinities}\label{section-ratios-of-infinities}
Many limits have the form \(\frac{\infty}{\infty}\), which we take to mean that the expressions in both the numerator and denominator are increasing or decreasing without bound. When confronted with a limit of type \(\lim\limits_{x\to\infty}\frac{\fe{f}{x}}{\fe{g}{x}}\) or \(\lim\limits_{x\to-\infty}\frac{\fe{f}{x}}{\fe{g}{x}}\) that has the form \(\frac{\infty}{\infty}\), we can frequently resolve the limit if we first divide the dominant factor of the dominant term of the denominator from both the numerator and the denominator. When we do this, we need to completely simplify each of the resultant fractions and make sure that the resultant limit exists before we start to apply limit laws. We then apply the algebraic limit laws until all of the resultant limits can be replaced using limit laws R2 and R3. This process is illustrated in \hyperref[example-ratio-of-infinities]{Example~\ref*{example-ratio-of-infinities}}.%
\begin{example}\label{example-ratio-of-infinities}
\begin{align*}
\lim_{t\to\infty}\frac{3t^2+5t}{3-5t^2}&=\lim_{t\to\infty}\left(\frac{3t^2+5t}{3-5t^2}\cdot\frac{\sfrac{1}{t^2}}{\sfrac{1}{t^2}}\right)\\
&=\lim_{t\to\infty}\frac{3+\sfrac{5}{t}}{\sfrac{3}{t^2}-5}\tikzmark{ratio of infinities}\\
&=\frac{\lim_{t\to\infty}\left(3+\sfrac{5}{t}\right)}{\lim_{t\to\infty}\left(\sfrac{3}{t^2}-5\right)}&&\text{Limit Law A5}\\
&=\frac{\lim_{t\to\infty}3+\lim_{t\to\infty}\frac{5}{t}}{\lim_{t\to\infty}\frac{3}{t^2}-\lim_{t\to\infty}5}&&\text{Limit Laws A1 and A2}\\
&=\frac{3+0}{0-5}\\
&=-\frac{3}{5}
\end{align*}\Annotate[][anchor=west]{0ex}{0.5ex}{3}{0.5}{ratio of infinities}{\parbox{0.4\linewidth}{The ``form'' of the limit is now $\frac{3+0}{0-5}$, so we can begin to apply the limit laws because the limits will all exist.}}%
\end{example}
\typeout{************************************************}
\typeout{Exercises}
\typeout{************************************************}
\section*{Exercises}\label{exercises-9}

\begin{exercisegroup}%
Use the limit laws to establish the value of each limit after dividing the dominant term-factor in the denominator from both the numerator and denominator. Remember to simplify each resultant expression before you begin to apply the limit laws.%
\begin{exercisegroupbycol}{3}%
\begin{exercisegroupitem}{1. }\phantomsection\hypertarget{exercise-70}{\null}
\(\lim\limits_{t\to-\infty}\dfrac{4t^2}{4t^2+t^3}\)%
\end{exercisegroupitem}%
\par%
\begin{exercisegroupitem}{2. }\phantomsection\hypertarget{exercise-71}{\null}
\(\lim\limits_{t\to\infty}\dfrac{6e^t+10e^{2t}}{2e^{2t}}\)%
\end{exercisegroupitem}%
\par%
\begin{exercisegroupitem}{3. }\phantomsection\hypertarget{exercise-72}{\null}
\(\lim\limits_{y\to\infty}\sqrt{\dfrac{4y+5}{5+9y}}\)%
\end{exercisegroupitem}%
\par%
\end{exercisegroupbycol}%
\end{exercisegroup}%
\typeout{************************************************}
\typeout{Section 2.7 Non-existent Limits}
\typeout{************************************************}
\section[Non-existent Limits]{Non-existent Limits}\label{section-nonexistent-limits}
Many limit values do not exist. Sometimes the non-existence is caused by the function value either increasing without bound or decreasing without bound. In these special cases we use the symbols \(\infty\) and \(-\infty\) to communicate the non-existence of the limits. \hyperref[figure-first-nonexistent-limit]{Figures~\ref*{figure-first-nonexistent-limit}}\textendash{}\hyperref[figure-last-nonexistent-limit]{2.7.3} can be used to illustrate some ways in which we communicate the \terminology{non-existence} of these types of limits.%
\begin{itemize}[label=\textbullet]
\item{}In \hyperref[figure-first-nonexistent-limit]{Figure~\ref*{figure-first-nonexistent-limit}} we could (correctly) write \(\lim\limits_{x\to2}\fe{k}{x}=\infty\), \(\lim\limits_{x\to2^{-}}\fe{k}{x}=\infty\), and \(\lim\limits_{x\to2^{+}}\fe{k}{x}=\infty\).\item{}In \hyperref[figure-middle-nonexistent-limit]{Figure~\ref*{figure-middle-nonexistent-limit}} we could (correctly) write \(\lim\limits_{t\to4}\fe{w}{t}=-\infty\), \(\lim\limits_{t\to4^{-}}\fe{w}{t}=\infty\), and \(\lim\limits_{t\to4^{+}}\fe{w}{t}=-\infty\).\item{}In \hyperref[figure-last-nonexistent-limit]{Figure~\ref*{figure-last-nonexistent-limit}} we could (correctly) write \(\lim\limits_{x\to-3^{-}}\fe{T}{x}=\infty\) and \(\lim\limits_{x\to-3^{+}}\fe{T}{x}=-\infty\). There is no shorthand way of communicating the non-existence of the two sided limit \(\lim\limits_{x\to-3}\fe{T}{x}\).\end{itemize}
\begin{figure}
\centering
\pushValignCaptionBottom[b]{minipage}{.30\textwidth}{%
\pgfplotsset{every axis/.append style={width=\linewidth}}%
\centering% horizontal alignment 
{
\begin{tikzpicture}
\begin{axis}[]
    \addplot[pccplot,
             variable=\t,
             domain=0:28.8167615,
             ]
             ({2+1/sqrt(13)*1.1^t}, {-6.5+1/(1/sqrt(13)*1.1^t)^2}); 
    \addplot[pccplot,
             variable=\t,
             domain=0:35.489593,
             ]
             ({2-1/sqrt(13)*1.1^t}, {-6.5+1/(1/sqrt(13)*1.1^t)^2});
    \addplot[asymptote
    ]coordinates{
        (2,-7)
        (2,7)};
\end{axis}
\end{tikzpicture}
}
}% end body 
{\captionof{figure}{\(y=\fe{k}{x}\)\label{figure-first-nonexistent-limit}}
}% caption 
\pushValignCaptionBottom[b]{minipage}{.30\textwidth}{%
\pgfplotsset{every axis/.append style={width=\linewidth}}%
\centering% horizontal alignment 
{
\begin{tikzpicture}
\begin{axis}[,
             xlabel = {$t$},]
    \addplot[pccplot,
             variable=\t,
             domain=0:22.6496693,
             ]
             ({4+1/sqrt(13)*1.1^t}, {6.5-1/(1/sqrt(13)*1.1^t)^2});
    \addplot[pccplot,
             variable=\t,
             domain=0:37.7066604,
             ]
             ({4-1/sqrt(13)*1.1^t}, {6.5-1/(1/sqrt(13)*1.1^t)^2});
    \addplot[asymptote
    ]coordinates{
        (4,-7)
        (4,7)};
\end{axis}
\end{tikzpicture}
}
}% end body 
{\captionof{figure}{\(y=\fe{w}{t}\)\label{figure-middle-nonexistent-limit}}
}% caption 
\pushValignCaptionBottom[b]{minipage}{.30\textwidth}{%
\pgfplotsset{every axis/.append style={width=\linewidth}}%
\centering% horizontal alignment 
{
\begin{tikzpicture}
\begin{axis}[]
    \addplot[pccplot,
             variable=\t,
             domain=0:36.6565789,
             ]
             ({-3+1/sqrt(13)*1.1^t}, {6.5-1/(1/sqrt(13)*1.1^t)^2});
    \addplot[pccplot,
             variable=\t,
             domain=0:26.1799558,
             ]
             ({-3-1/sqrt(13)*1.1^t}, {-6.5+1/(1/sqrt(13)*1.1^t)^2});
    \addplot[asymptote
    ]coordinates{
        (-3,-7)
        (-3,7)};
\end{axis}
\end{tikzpicture}
}
}% end body 
{\captionof{figure}{\(y=\fe{T}{x}\)\label{figure-last-nonexistent-limit}}
}% caption 
\popValignCaptionBottom
\end{figure}
\typeout{************************************************}
\typeout{Exercises}
\typeout{************************************************}
\section*{Exercises}\label{exercises-10}

\begin{exerciselist}
\item[1.]\phantomsection\hypertarget{exercise-73}{\null}In the plane provided, draw the graph of a single function, \(f\), that satisfies each of the following limit statements. Make sure that you draw the necessary asymptotes and that you label each asymptote with its equation.%
\begin{figure}
\centering
\pushValignCaptionBottom[t]{minipage}{.60\textwidth}{%
\pgfplotsset{every axis/.append style={width=\linewidth}}%
% horizontal alignment 
\parbox{\textwidth}{%
% horizontal alignment 
\begin{align*}
\lim\limits_{x\to3^{-}}\fe{f}{x}&=-\infty&\lim\limits_{x\to\infty}\fe{f}{x}&=0\\
\lim\limits_{x\to3^{+}}\fe{f}{x}&=\infty&\lim\limits_{x\to-\infty}\fe{f}{x}&-2
\end{align*}%
}
}% end body 
{}% caption 
\pushValignCaptionBottom[t]{minipage}{.35\textwidth}{%
\pgfplotsset{every axis/.append style={width=\linewidth}}%
\centering% horizontal alignment 
{
\begin{tikzpicture}
\begin{axis}[]
\end{axis}
\end{tikzpicture}
}
}% end body 
{}% caption 
\popValignCaptionBottom
\end{figure}
\par\smallskip
\end{exerciselist}
\typeout{************************************************}
\typeout{Section 2.8 Vertical Asymptotes}
\typeout{************************************************}
\section[Vertical Asymptotes]{Vertical Asymptotes}\label{section-vertical-asymptotes}
Whenever \(\lim\limits_{x\to a}\fe{f}{x}\neq0\) but \(\lim\limits_{x\to a}\fe{g}{x}=0\), then \(\lim\limits_{x\to a}\frac{\fe{f}{x}}{\fe{g}{x}}\) \emph{does not exist} because from either side of \(a\) the value of \(\frac{\fe{f}{x}}{\fe{g}{x}}\) has an absolute value that will become arbitrarily large. In these situations the line \(x=a\) is a vertical asymptote for the graph of \(y=\frac{\fe{f}{x}}{\fe{g}{x}}\). For example, the line \(x=2\) is a vertical asymptote for the function \(h\) defined by \(\fe{h}{x}=\frac{x+5}{2-x}\). We say that \(\lim\limits_{x\to 2}\frac{x+5}{2-x}\) has the form ``not-zero over zero''. (Specifically, the form of \(\lim\limits_{x\to 2}\frac{x+5}{2-x}\) is \(\frac{7}{0}\).) Every limit with form ``not-zero over zero'' \emph{does not exist}. However, we frequently can communicate the non-existence of the limit using an infinity symbol. In the case of \(\fe{h}{x}=\frac{x+5}{2-x}\) it's pretty easy to see that \(\fe{h}{1.99}\) is a positive number whereas \(\fe{h}{2.01}\) is a negative number. Consequently, we can infer that \(\lim\limits_{x\to 2^{-}}\fe{h}{x}=\infty\) and \(\lim\limits_{x\to 2^{+}}\fe{h}{x}=-\infty\). Remember, these equations are communicating that the limits \emph{do not exist} as well as the reason for their non-existence. There is no short-hand way to communicate the non-existence of the two-sided limit \(\lim\limits_{x\to 2}\fe{h}{x}\).%
\typeout{************************************************}
\typeout{Exercises}
\typeout{************************************************}
\section*{Exercises}\label{exercises-11}

\begin{exercisegroup}%
Suppose that \(\fe{g}{t}=\frac{t+4}{t+3}\).%
\begin{exercisegroupbycol}{1}%
\begin{exercisegroupitem}{1. }\phantomsection\hypertarget{exercise-74}{\null}
What is the vertical asymptote on the graph of \(y=\fe{g}{t}\)?%
\end{exercisegroupitem}%
\par%
\begin{exercisegroupitem}{2. }\phantomsection\hypertarget{exercise-75}{\null}
Write an equality about \(\lim\limits_{t\to-3^{-}}\fe{g}{t}\).%
\end{exercisegroupitem}%
\par%
\begin{exercisegroupitem}{3. }\phantomsection\hypertarget{exercise-76}{\null}
Write an equality about \(\lim\limits_{t\to-3^{+}}\fe{g}{t}\).%
\end{exercisegroupitem}%
\par%
\begin{exercisegroupitem}{4. }\phantomsection\hypertarget{exercise-77}{\null}
Is it possible to write an equality about \(\lim\limits_{t\to-3}\fe{g}{t}\)? If so, write it.%
\end{exercisegroupitem}%
\par%
\begin{exercisegroupitem}{5. }\phantomsection\hypertarget{exercise-78}{\null}
Which of the following limits exist? \(\lim\limits_{t\to-3^{-}}\fe{g}{t}\)? \(\lim\limits_{t\to-3^{+}}\fe{g}{t}\)? \(\lim\limits_{t\to-3}\fe{g}{t}\)?%
\end{exercisegroupitem}%
\par%
\end{exercisegroupbycol}%
\end{exercisegroup}%
\begin{exercisegroup}%
Suppose that \(\fe{z}{x}=\frac{7-3x^2}{\left(x-2\right)^2}\).%
\begin{exercisegroupbycol}{1}%
\begin{exercisegroupitem}{6. }\phantomsection\hypertarget{exercise-79}{\null}
What is the vertical asymptote on the graph of \(y=\fe{z}{x}\)?%
\end{exercisegroupitem}%
\par%
\begin{exercisegroupitem}{7. }\phantomsection\hypertarget{exercise-80}{\null}
Is it possible to write an equality about \(\lim\limits_{x\to2}\fe{z}{x}\)? If so, write it.%
\end{exercisegroupitem}%
\par%
\begin{exercisegroupitem}{8. }\phantomsection\hypertarget{exercise-81}{\null}
What is the horizontal asymptote on the graph of \(y=\fe{z}{x}\)?%
\end{exercisegroupitem}%
\par%
\begin{exercisegroupitem}{9. }\phantomsection\hypertarget{exercise-82}{\null}
Which of the following limits exist? \(\lim\limits_{x\to2^{-}}\fe{z}{x}\)? \(\lim\limits_{x\to2^{+}}\fe{z}{x}\)? \(\lim\limits_{x\to2}\fe{z}{x}\)?%
\end{exercisegroupitem}%
\par%
\end{exercisegroupbycol}%
\end{exercisegroup}%
\begin{exerciselist}
\item[10.]\phantomsection\hypertarget{exercise-83}{\null}Consider the function \(f\) defined by \(\fe{f}{x}=\frac{x+7}{x-8}\). Complete \hyperref[table-vertical-asymptote]{Table~\ref*{table-vertical-asymptote}} without the use of your calculator.%
\par
Use this as an opportunity to discuss why limits of form ``not-zero over zero'' are ``infinite limits''. What limit equation is being illustrated in the table?%
\begin{table}
\centering
\caption{\(\fe{f}{x}=\frac{x+7}{x-8}\)\label{table-vertical-asymptote}}
\begin{tabular}{llll}\hrulethick
\multicolumn{1}{c}{\(x\)}&\multicolumn{1}{c}{\(x+7\)}&\multicolumn{1}{c}{\(x-8\)}&\multicolumn{1}{c}{\(\fe{f}{x}\)}\\\hrulemedium
\(8.1\)&\(15.1\)&\(0.1\)&\\
\(8.01\)&\(15.01\)&\(0.01\)&\\
\(8.001\)&\(15.001\)&\(0.001\)&\\
\(8.0001\)&\(15.0001\)&\(0.0001\)&
\end{tabular}
\end{table}
\par\smallskip
\begin{exercisegroup}%
Hear me, and hear me loud\dots{}\(\infty\) \emph{does not exist}. This, in part, is why we cannot apply Limit Law R1 to an expression like \(\lim\limits_{x\to\infty}x\) to see that \(\lim\limits_{x\to\infty}x=\infty\). When we write, say, \(\lim\limits_{x\to7}x=7\), we are replacing the limit, we are replacing the limit expression \emph{with its value}\textemdash{}that’s what the replacement laws are all about! When we write \(\lim\limits_{x\to\infty}x=\infty\), we are not replacing the limit expression with a value! We are explicitly saying that the limit has no value (i.e.\@ does not exist) as well as saying the reason the limit does not exist. The limit laws (R1-R3 and A1-A6) can only be applied when all of the limits in the equation exist. With this in mind, discuss and decide whether each of the following equations are \emph{true} or \emph{false}.%
\begin{exercisegroupbycol}{1}%
\begin{exercisegroupitem}{11. }\phantomsection\hypertarget{exercise-hear-me-first}{\null}
True or False? \(\lim\limits_{x\to0}\left(\frac{e^x}{e^x}\right)=\frac{\lim\limits_{x\to0}e^x}{\lim\limits_{x\to0}e^x}\)%
\end{exercisegroupitem}%
\par%
\begin{exercisegroupitem}{12. }\phantomsection\hypertarget{exercise-85}{\null}
True or False? \(\lim\limits_{x\to1}\frac{e^x}{\fe{\ln}{x}}=\frac{\lim\limits_{x\to1}e^x}{\lim\limits_{x\to1}\fe{\ln}{x}}\)%
\end{exercisegroupitem}%
\par%
\begin{exercisegroupitem}{13. }\phantomsection\hypertarget{exercise-86}{\null}
True or False? \(\lim\limits_{x\to0^{+}}\left(2\fe{\ln}{x}\right)=2\lim\limits_{x\to0^{+}}\fe{\ln}{x}\)%
\end{exercisegroupitem}%
\par%
\begin{exercisegroupitem}{14. }\phantomsection\hypertarget{exercise-87}{\null}
True or False? \(\lim\limits_{x\to\infty}\left(e^x-\fe{\ln}{x}\right)=\lim\limits_{x\to0^{+}}e^x-\lim\limits_{x\to0^{+}}\fe{\ln}{x}\)%
\end{exercisegroupitem}%
\par%
\begin{exercisegroupitem}{15. }\phantomsection\hypertarget{exercise-88}{\null}
True or False? \(\lim\limits_{x\to-\infty}\left(\frac{e^{-x}}{e^{-x}}\right)=\frac{\lim\limits_{x\to-\infty}e^{-x}}{\lim\limits_{x\to-\infty}e^{-x}}\)%
\end{exercisegroupitem}%
\par%
\begin{exercisegroupitem}{16. }\phantomsection\hypertarget{exercise-89}{\null}
True or False? \(\lim\limits_{x\to1}\left(\frac{\fe{\ln}{x}}{e^x}\right)=\frac{\lim\limits_{x\to1}\fe{\ln}{x}}{\lim\limits_{x\to1}e^x}\)%
\end{exercisegroupitem}%
\par%
\begin{exercisegroupitem}{17. }\phantomsection\hypertarget{exercise-90}{\null}
True or False? \(\lim\limits_{\theta\to\infty}\frac{\fe{\sin}{\theta}}{\fe{\sin}{\theta}}=\frac{\lim\limits_{\theta\to\infty}\fe{\sin}{\theta}}{\lim\limits_{\theta\to\infty}\fe{\sin}{\theta}}\)%
\end{exercisegroupitem}%
\par%
\begin{exercisegroupitem}{18. }\phantomsection\hypertarget{exercise-hear-me-last}{\null}
True or False? \(\lim\limits_{x\to-\infty}e^{\sfrac{1}{x}}=e^{\lim_{x\to-\infty}\sfrac{1}{x}}\)%
\end{exercisegroupitem}%
\par%
\end{exercisegroupbycol}%
\end{exercisegroup}%
\item[19.]\phantomsection\hypertarget{exercise-92}{\null}Mindy tried to evaluate \(\lim\limits_{x\to6^{+}}\frac{4x-24}{x^2-12x+36}\) using the limit laws. Things went horribly wrong for Mindy (her work is shown below). Identify what is wrong in Mindy's work and discuss what a more reasonable approach might have been.%
\par
\begin{align*}
\lim_{x\to6^{+}}\frac{4x-24}{x^2-12x+36}&=\lim_{x\to6^{+}}\frac{4(x-6)}{(x-6)^2}\\
&=\lim_{x\to6^{+}}\frac{4}{x-6}\\
&\tikzmark{Mindys work}=\frac{\lim\limits_{x\to6^{+}}4}{\lim\limits_{x\to6^{+}}(x-6)}&&\text{Limit Law A5}\\
&=\frac{\lim\limits_{x\to6^{+}}4}{\lim\limits_{x\to6^{+}}x-\lim\limits_{x\to6^{+}}6}&&\text{Limit Law A2}\\
&=\frac{4}{6-6}&&\text{Limit Laws R1 and R2}\\
&=\frac{4}{0}
\end{align*}\Annotate[draw=none][anchor=east]{0ex}{0.5ex}{-1}{0}{Mindys work}{\parbox{0.3\linewidth}{This ``solution'' is not correct! Do not emulate Mindy's work!}}%
\par\smallskip
\end{exerciselist}
\typeout{************************************************}
\typeout{Section 2.9 Continuity}
\typeout{************************************************}
\section[Continuity]{Continuity}\label{section-continuity}
Many statements we make about functions are only true over intervals where the function is \terminology{continuous}. When we say a function is continuous over an interval, we basically mean that there are no breaks in the function over that interval; that is, there are no vertical asymptotes, holes, jumps, or gaps along that interval.%
\begin{definition}[Continuity]\label{definition-continuity}
The function \(f\) is \terminology{continuous at the number \(a\)} if and only if \(\lim\limits_{x\to a}\fe{f}{x}=\fe{f}{a}\).%
\par
There are three ways that the defining property can fail to be satisfied at a given value of \(a\). To facilitate exploration of these three manner of failure, we can break the defining property into a spectrum of three properties.%
\begin{multicols}{3}
\begin{enumerate}
\item{}\(\fe{f}{a}\) must be defined\item{}\(\lim\limits_{x\to a}\fe{f}{x}\) must exist\item{}\(\lim\limits_{x\to a}\fe{f}{x}\) must equal \(\fe{f}{a}\)\end{enumerate}
\end{multicols}
\par
Please note that if either property 1 or property 2 fails to be satisfied at a given value of \(a\), then property 3 also fails to be satisfied at \(a\).%
\end{definition}
\typeout{************************************************}
\typeout{Exercises}
\typeout{************************************************}
\section*{Exercises}\label{exercises-12}

\begin{figure}
\centering
{
\begin{tikzpicture}
\begin{axis}[xlabel = {$t$},]
    \addplot[pccplot,
             domain=-6.7:-4,
             <-,]
             {1+6/5*(x+4)};
    \addplot[pccplot,
             domain=-4:-2,
             -]
             {4};
             ]
    \addplot[pccplot,
             domain=-2:3,
             -,]
             {-x+2};
             ]
    \addplot[pccplot,
             domain=0:22.4536998,
             variable=t,
             <-,]
             ({5-2/8.5*1.1^t},{1/(1/8.5*1.1^t)-2});
    \addplot[pccplot,
             domain=0:15.38769119,
             variable=t,
             ]
             ({5+1/sqrt(6.5)*1.1^t},{1/(1/sqrt(6.5)*1.1^t)^2});
    \addplot[asymptote
    ]coordinates{
        (5,-7)
        (5,7)};
    \addplot[soldot
    ]coordinates{
        (-4,4)
        (2,-3)};
    \addplot[holdot
    ]coordinates{
        (-4,1)
        (-1,3)
        (2,0)};
\end{axis}
\end{tikzpicture}
}
\caption{\(y=\fe{h}{x}\)\label{figure-discontinuities}}
\end{figure}
\begin{exerciselist}
\item[1.]\phantomsection\hypertarget{exercise-93}{\null}State the values of \(t\) at which the function shown in \hyperref[figure-discontinuities]{Figure~\ref*{figure-discontinuities}} is discontinuous. For each instance of discontinuity, state (by number) all of the sub-properties in \hyperref[definition-continuity]{Definition~\ref*{definition-continuity}} that fail to be satisfied.%
\par\smallskip
\end{exerciselist}
\typeout{************************************************}
\typeout{Section 2.10 Discontinuities}
\typeout{************************************************}
\section[Discontinuities]{Discontinuities}\label{section-discontinuities}
When a function has a discontinuity at \(a\), the function is sometimes continuous from only the right or only the left at \(a\). (Please note that when we say ``the function is continuous at \(a\)'' we mean that the function is continuous from \emph{both} the right and left at \(a\).)%
\begin{definition}[One-sided Continuity]\label{definition-4}
The function \(f\) is continuous from the left at \(a\) if and only if \(\lim\limits_{x\to a^{-}}\fe{f}{x}=\fe{f}{a}\).%
\par
The function \(f\) is continuous from the right at \(a\) if and only if \(\lim\limits_{x\to a^{+}}\fe{f}{x}=\fe{f}{a}\).%
\end{definition}
\par
Some discontinuities are classified as \terminology{removable discontinuities}. Specifically, discontinuities that are holes or skips (holes with a secondary point) are called removable.%
\begin{definition}[Removeable Discontinuity]\label{definition-5}
We say that \(f\) has a removable discontinuity at \(a\) if \(f\) is discontinuous at \(a\) but \(\lim\limits_{x\to a}\fe{f}{x}\) exists.%
\end{definition}
\typeout{************************************************}
\typeout{Exercises}
\typeout{************************************************}
\section*{Exercises}\label{exercises-13}

\begin{exerciselist}
\item[1.]\phantomsection\hypertarget{exercise-94}{\null}Referring to the function \(h\) shown in \hyperref[figure-discontinuities]{Figure~\ref*{figure-discontinuities}}, state the values of \(t\) where the function is continuous from the right but not the left. Then state the values of \(t\) where the function is continuous from the left but not the right.%
\par\smallskip
\item[2.]\phantomsection\hypertarget{exercise-95}{\null}Referring again to the function \(h\) shown in \hyperref[figure-discontinuities]{Figure~\ref*{figure-discontinuities}}, state the values of \(t\) where the function has removable discontinuities.%
\par\smallskip
\end{exerciselist}
\typeout{************************************************}
\typeout{Section 2.11 Continuity on an Interval}
\typeout{************************************************}
\section[Continuity on an Interval]{Continuity on an Interval}\label{section-continuity-on-an-interval}
Now that we have a definition for continuity at a number, we can go ahead and define what we mean when we say a function is continuous over an interval.%
\begin{definition}[Continuity on an Interval]\label{definition-continuity-on-an-interval}
The function \(f\) is \terminology{continuous} over an open interval if and only if it is continuous at each and every number on that interval.%
\par
The function \(f\) is \terminology{continuous} over a closed interval \(\cinterval{a}{b}\) if and only if it is continuous on \(\ointerval{a}{b}\), continuous from the right at \(a\), and continuous from the left at \(b\). Similar definitions apply to half-open intervals.%
\end{definition}
\typeout{************************************************}
\typeout{Exercises}
\typeout{************************************************}
\section*{Exercises}\label{exercises-14}

\begin{exerciselist}
\item[1.]\phantomsection\hypertarget{exercise-96}{\null}Write a definition for continuity on the half-open interval \(\ocinterval{a}{b}\).%
\par\smallskip
\end{exerciselist}
\begin{exercisegroup}%
Referring to the function in \hyperref[figure-discontinuities]{Figure~\ref*{figure-discontinuities}}, decide whether each of the following statements are true or false.%
\begin{exercisegroupbycol}{2}%
\begin{exercisegroupitem}{2. }\phantomsection\hypertarget{exercise-97}{\null}
\(h\) is continuous on \(\cointerval{-4}{-1}\).%
\end{exercisegroupitem}%
\par%
\begin{exercisegroupitem}{3. }\phantomsection\hypertarget{exercise-98}{\null}
\(h\) is continuous on \(\ointerval{-4}{-1}\).%
\end{exercisegroupitem}%
\par%
\begin{exercisegroupitem}{4. }\phantomsection\hypertarget{exercise-99}{\null}
\(h\) is continuous on \(\ocinterval{-4}{-1}\).%
\end{exercisegroupitem}%
\par%
\begin{exercisegroupitem}{5. }\phantomsection\hypertarget{exercise-100}{\null}
\(h\) is continuous on \(\ocinterval{-1}{2}\).%
\end{exercisegroupitem}%
\par%
\begin{exercisegroupitem}{6. }\phantomsection\hypertarget{exercise-101}{\null}
\(h\) is continuous on \(\ointerval{-1}{2}\).%
\end{exercisegroupitem}%
\par%
\begin{exercisegroupitem}{7. }\phantomsection\hypertarget{exercise-102}{\null}
\(h\) is continuous on \(\ointerval{-\infty}{-4}\).%
\end{exercisegroupitem}%
\par%
\begin{exercisegroupitem}{8. }\phantomsection\hypertarget{exercise-103}{\null}
\(h\) is continuous on \(\ocinterval{-\infty}{-4}\).%
\end{exercisegroupitem}%
\par%
\end{exercisegroupbycol}%
\end{exercisegroup}%
\begin{exercisegroup}%
Several functions are described below. Your task is to draw each function on its provided axis system. Do not introduce any unnecessary discontinuities or intercepts that are not directly implied by the stated properties. Make sure that you draw all implied asymptotes and label them with their equations.%
\begin{exercisegroupbycol}{1}%
\begin{exercisegroupitem}{9. }\phantomsection\hypertarget{exercise-104}{\null}
Sketch a graph of a function that satisfies all of the following properties%
\begin{figure}
\centering
\pushValignCaptionBottom[t]{minipage}{.60\textwidth}{%
\pgfplotsset{every axis/.append style={width=\linewidth}}%
% horizontal alignment 
\parbox{\textwidth}{%
% horizontal alignment 
\begin{align*}
\lim\limits_{x\to4^{-}}\fe{f}{x}&=2&\fe{f}{0}&=4\\
\lim\limits_{x\to4^{+}}\fe{f}{x}&=5&\fe{f}{4}&=5\\
\lim\limits_{x\to-\infty}\fe{f}{x}=\lim\limits_{x\to\infty}\fe{f}{x}&=4
\end{align*}%
}
}% end body 
{}% caption 
\pushValignCaptionBottom[t]{minipage}{.35\textwidth}{%
\pgfplotsset{every axis/.append style={width=\linewidth}}%
\centering% horizontal alignment 
{
\begin{tikzpicture}
\begin{axis}[]
\end{axis}
\end{tikzpicture}
}
}% end body 
{}% caption 
\popValignCaptionBottom
\end{figure}
\end{exercisegroupitem}%
\par%
\begin{exercisegroupitem}{10. }\phantomsection\hypertarget{exercise-105}{\null}
Sketch a graph of a function that satisfies all of the following properties%
\begin{figure}
\centering
\pushValignCaptionBottom[t]{minipage}{.60\textwidth}{%
\pgfplotsset{every axis/.append style={width=\linewidth}}%
% horizontal alignment 
\parbox{\textwidth}{%
% horizontal alignment 
\begin{align*}
\lim\limits_{x\to-2}\fe{g}{x}&=\infty&\fe{g}{0}&=4\\
\lim\limits_{x\to-\infty}\fe{g}{x}&=\infty&\fe{g}{3}&=-2\\
&&\fe{g}{6}&=0
\end{align*}
                            \(g\) is continuous and has constant slope on \(\ointerval{0}{\infty}\).%
}
}% end body 
{}% caption 
\pushValignCaptionBottom[t]{minipage}{.35\textwidth}{%
\pgfplotsset{every axis/.append style={width=\linewidth}}%
\centering% horizontal alignment 
{
\begin{tikzpicture}
\begin{axis}[]
\end{axis}
\end{tikzpicture}
}
}% end body 
{}% caption 
\popValignCaptionBottom
\end{figure}
\end{exercisegroupitem}%
\par%
\begin{exercisegroupitem}{11. }\phantomsection\hypertarget{exercise-106}{\null}
Sketch a graph of a function that satisfies all of the following properties%
\begin{figure}
\centering
\pushValignCaptionBottom[t]{minipage}{.60\textwidth}{%
\pgfplotsset{every axis/.append style={width=\linewidth}}%
% horizontal alignment 
\parbox{\textwidth}{%
% horizontal alignment 
The only discontinuities on \(m\) occur at \(-4\) and \(3\). \(m\) has no \(x\)-intercepts. \(m\) has constant slope of \(-2\) over \(\ointerval{-\infty}{-4}\). \(m\) is continuous over \(\cointerval{-4}{3}\).\begin{align*}
\fe{m}{-6}&=5&\lim\limits_{x\to3}\fe{m}{x}&=-\infty\\
\lim\limits_{x\to-4^{+}}\fe{m}{x}&=-2&\lim\limits_{x\to\infty}\fe{m}{x}&=-\infty
\end{align*}%
}
}% end body 
{}% caption 
\pushValignCaptionBottom[t]{minipage}{.35\textwidth}{%
\pgfplotsset{every axis/.append style={width=\linewidth}}%
\centering% horizontal alignment 
{
\begin{tikzpicture}
\begin{axis}[]
\end{axis}
\end{tikzpicture}
}
}% end body 
{}% caption 
\popValignCaptionBottom
\end{figure}
\end{exercisegroupitem}%
\par%
\end{exercisegroupbycol}%
\end{exercisegroup}%
\typeout{************************************************}
\typeout{Section 2.12 Discontinuous Formulas}
\typeout{************************************************}
\section[Discontinuous Formulas]{Discontinuous Formulas}\label{section-discontinuous-formulas}
Discontinuities are a little more challenging to identify when working with formulas than when working with graphs. One reason for the added difficulty is that when working with a function formula you have to dig into your memory bank and retrieve fundamental properties about certain types of functions.%
\typeout{************************************************}
\typeout{Exercises}
\typeout{************************************************}
\section*{Exercises}\label{exercises-15}

\begin{exerciselist}
\item[1.]\phantomsection\hypertarget{exercise-107}{\null}What would cause a discontinuity on a rational function (a polynomial divided by another polynomial)?%
\par\smallskip
\item[2.]\phantomsection\hypertarget{exercise-108}{\null}If \(y\) is a function of \(u\), defined by \(y=\fe{\ln}{u}\), what is always true about the \terminology{argument} of the function, \(u\), over intervals where the function is continuous?%
\par\smallskip
\item[3.]\phantomsection\hypertarget{exercise-109}{\null}Name three values of \(\theta\) where the function \(\fe{\tan}{\theta}\) is discontinuous.%
\par\smallskip
\item[4.]\phantomsection\hypertarget{exercise-110}{\null}What is the domain of the function \(k\), where \(\fe{k}{t}=\sqrt{t-4}\)?%
\par\smallskip
\item[5.]\phantomsection\hypertarget{exercise-111}{\null}What is the domain of the function \(g\), where \(\fe{g}{t}=\sqrt[3]{t-4}\)?%
\par\smallskip
\end{exerciselist}
\typeout{************************************************}
\typeout{Section 2.13 Piecewise-Defined Functions}
\typeout{************************************************}
\section[Piecewise-Defined Functions]{Piecewise-Defined Functions}\label{section-piecewise-defined-functions}
Piecewise-defined functions are functions where the formula used depends upon the value of the input. When looking for discontinuities on piecewise-defined functions, you need to investigate the behavior at values where the formula changes as well as values where the issues discussed in \hyperref[section-discontinuous-formulas]{Section~\ref*{section-discontinuous-formulas}} might pop up.%
\typeout{************************************************}
\typeout{Exercises}
\typeout{************************************************}
\section*{Exercises}\label{exercises-16}

\begin{exercisegroup}%
This question is all about the function \(f\) defined by \[\fe{f}{x}=\begin{cases}\frac{4}{5-x}&x\lt1\\\frac{x-3}{x-3}&1\lt x\lt4\\2x+1&4\leq x\leq7\\\frac{15}{8-x}&x\gt7\text{.}\end{cases}\] Answer \hyperref[exercise-first-piecewise]{Exercise~1} at each of the values \(1,3,4,5,7\), and \(8\). At the values where the answer to \hyperref[exercise-first-piecewise]{Exercise~1} is yes, go ahead and answer \hyperref[exercise-second-piecewise]{Exercises~2}\textendash{}\hyperref[exercise-last-piecewise]{4}; skip \hyperref[exercise-second-piecewise]{Exercises~2}\textendash{}\hyperref[exercise-last-piecewise]{4} at the values where the answer to \hyperref[exercise-first-piecewise]{Exercise~1} is no.%
\begin{exercisegroupbycol}{1}%
\begin{exercisegroupitem}{1. }\phantomsection\hypertarget{exercise-first-piecewise}{\null}
Is \(f\) discontinuous at the given value?%
\end{exercisegroupitem}%
\par%
\begin{exercisegroupitem}{2. }\phantomsection\hypertarget{exercise-second-piecewise}{\null}
Is \(f\) continuous only from the left at the given value?%
\end{exercisegroupitem}%
\par%
\begin{exercisegroupitem}{3. }\phantomsection\hypertarget{exercise-114}{\null}
Is \(f\) continuous only from the right at the given value?%
\end{exercisegroupitem}%
\par%
\begin{exercisegroupitem}{4. }\phantomsection\hypertarget{exercise-last-piecewise}{\null}
Is the discontinuity removable?%
\end{exercisegroupitem}%
\par%
\end{exercisegroupbycol}%
\end{exercisegroup}%
\begin{exercisegroup}%
Consider the function \(g\) defined by \[\fe{g}{x}=\begin{cases}\frac{C}{x-17}&x\lt10\\C+3x&x=10\\2C-4&x\gt10\text{.}\end{cases}\] The symbol \(C\) represents the same real number in all three of the piecewise formulas.%
\begin{exercisegroupbycol}{1}%
\begin{exercisegroupitem}{5. }\phantomsection\hypertarget{exercise-116}{\null}
Find the value for \(C\) that makes the function continuous on \(\ocinterval{-\infty}{10}\). Make sure that your reasoning is clear.%
\end{exercisegroupitem}%
\par%
\begin{exercisegroupitem}{6. }\phantomsection\hypertarget{exercise-117}{\null}
Is it possible to find a value for \(C\) that makes the function continuous over \(\ointerval{-\infty}{\infty}\)? Explain.%
\end{exercisegroupitem}%
\par%
\end{exercisegroupbycol}%
\end{exercisegroup}%
\begin{exercisegroup}%
Consider the function \(f\) defined by \[\fe{f}{x}=\begin{cases}\frac{5}{x-10}&x\leq5\\\frac{5}{5x-30}&5\lt x\lt7\\\frac{x-2}{12-x}&x>7\text{.}\end{cases}\] State the values of \(x\) where each of the following occur. If a stated property doesn't occur, make sure that you state that (as opposed to simply not responding to the question). No explanation necessary.%
\begin{exercisegroupbycol}{1}%
\begin{exercisegroupitem}{7. }\phantomsection\hypertarget{exercise-118}{\null}
At what values of \(x\) is \(f\) discontinuous?%
\end{exercisegroupitem}%
\par%
\begin{exercisegroupitem}{8. }\phantomsection\hypertarget{exercise-119}{\null}
At what values of \(x\) is \(f\) continuous from the left, but not from the right?%
\end{exercisegroupitem}%
\par%
\begin{exercisegroupitem}{9. }\phantomsection\hypertarget{exercise-120}{\null}
At what values of \(x\) is \(f\) continuous from the right, but not from the left?%
\end{exercisegroupitem}%
\par%
\begin{exercisegroupitem}{10. }\phantomsection\hypertarget{exercise-121}{\null}
At what values of \(x\) does \(f\) have removable discontinuities?%
\end{exercisegroupitem}%
\par%
\end{exercisegroupbycol}%
\end{exercisegroup}%
\typeout{************************************************}
\typeout{Chapter 3 Introduction to the First Derivative}
\typeout{************************************************}
\chapter[Introduction to the First Derivative]{Introduction to the First Derivative}\label{chapter-introduction-first-derivative}
\typeout{************************************************}
\typeout{Section 3.1 Instantaneous Velocity}
\typeout{************************************************}
\section[Instantaneous Velocity]{Instantaneous Velocity}\label{section-instantaneous-velocity}
Most of the focus in \hyperref[chapter-rates-of-change]{Chapter~\ref*{chapter-rates-of-change}} was on average rates of change. While the idea of rates of change at one specific instant was alluded to, we couldn't explore that idea formally because we hadn't yet talked about limits. Now that we have covered average rates of change and limits we can put those two ideas together to discuss rates of change at specific instances in time.%
\par
Suppose that an object is tossed into the air in such a way that the elevation of the object (measured in \si{\foot}) is given by the function \(s\) defined by \(\fe{s}{t}=40+40t-16t^2\) where \(t\) is the amount of time that has passed since the object was tossed (measured in \si{\second}). Let's determine the velocity of the object \(2\) seconds into this flight.%
\par
Recall that the difference quotient \(\frac{\fe{s}{2+h}-\fe{s}{2}}{h}\) gives us the average velocity for the object between the times \(t=2\) and \(t=2+h\). So long as \(h\) is positive, we can think of \(h\) as the length of the time interval. To infer the velocity exactly \(2\) seconds into the flight we need the time interval as close to \(0\) as possible; this is done using the appropriate limit in \hyperref[example-instantaneous-velocity]{Example~\ref*{example-instantaneous-velocity}}.%
\begin{example}\label{example-instantaneous-velocity}
\begin{align*}
\fe{v}{2}&=\lim_{h\to0}\frac{\fe{s}{2+h}-\fe{s}{2}}{h}\\
&=\lim_{h\to0}\frac{\left[40+40\left(2+h\right)-16\left(2+h\right)^2\right]-\left[40+40(2)-16(2)^2\right]}{h}\\
&=\lim_{h\to0}\frac{40+80+40h-64-64h-16h^2-56}{h}\\
&=\lim_{h\to0}\frac{-24h-16h^2}{h}\\
&=\lim_{h\to0}\frac{h\left(-24-16h\right)}{h}\\
&=\lim_{h\to0}\left(-24-16h\right)\\
&=-24-16\cdot0\\
&=-24
\end{align*}%
\par
From this we can infer that the velocity of the object \(2\) seconds into its flight is \SI{-24}{\foot\per\second}. From that we know that the object is falling at a speed of \SI{24}{\foot\per\second}.%
\end{example}
\typeout{************************************************}
\typeout{Exercises}
\typeout{************************************************}
\section*{Exercises}\label{exercises-17}

\begin{exercisegroup}%
Suppose that an object is tossed into the air so that its elevation (measured in \si{\meter}) is given by the function \(p\) defined by \(\fe{p}{t}=300+10t-4.9t^2\) where \(t\) is the amount of time that has passed since the object was tossed (measured in \si{\second}).%
\begin{exercisegroupbycol}{1}%
\begin{exercisegroupitem}{1. }\phantomsection\hypertarget{exercise-instantaneous-velocity}{\null}
Evaluate \(\lim\limits_{h\to0}\frac{\fe{p}{4+h}-\fe{p}{4}}{h}\) showing each step in the simplification process (as illustrated in \hyperref[example-instantaneous-velocity]{Example~\ref*{example-instantaneous-velocity}}).%
\end{exercisegroupitem}%
\par%
\begin{exercisegroupitem}{2. }\phantomsection\hypertarget{exercise-instantaneous-velocity-second}{\null}
What is the unit for the value calculated in \hyperref[exercise-instantaneous-velocity]{Exercise~1} and what does the value (including unit) tell you abou the motion of the object?%
\end{exercisegroupitem}%
\par%
\begin{exercisegroupitem}{3. }\phantomsection\hypertarget{exercise-124}{\null}
Copy \hyperref[table-instantaneous-velocity]{Table~\ref*{table-instantaneous-velocity}} onto your paper and compute and record the missing values. Do these values support your answer to \hyperref[exercise-instantaneous-velocity-second]{Exercise~2}?%
\begin{table}
\centering
\caption{Average Velocities\label{table-instantaneous-velocity}}
\begin{tabular}{ll}\hrulethick
\multicolumn{1}{c}{\(t_1\)}&\multicolumn{1}{c}{\(\frac{\fe{p}{t_1}-\fe{p}{4}}{t_1-4}\)}\\\hrulemedium
\(3.9\)&\\
\(3.99\)&\\
\(3.999\)&\\
\(4.001\)&\\
\(4.01\)&\\
\(4.1\)&
\end{tabular}
\end{table}
\end{exercisegroupitem}%
\par%
\end{exercisegroupbycol}%
\end{exercisegroup}%
\typeout{************************************************}
\typeout{Section 3.2 Tangent Lines}
\typeout{************************************************}
\section[Tangent Lines]{Tangent Lines}\label{section-tangent-lines}
In previous activities we saw that if \(p\) is a position function, then the difference quotient for \(p\) can be used to calculate average velocities and the expression \(\frac{\fe{p}{t_0+h}-\fe{p}{t_0}}{h}\) calculates the instantaneous velocity at time \(t_0\).%
\par
Graphically, the difference quotient of a function \(f\) can be used to calculate the slope of secant lines to \(f\). What happens when we take the run of the secant line to zero? Basically, we are connecting two points on the line that are really, really, (\emph{really}), close to one another. As mentioned above, sending \(h\) to zero turns an average velocity into an instantaneous velocity. Graphically, sending \(h\) to zero turns a secant line into a \terminology{tangent line}.%
\begin{example}\label{example-tangent-line}
The tangent line to the function \(f\) defined by \(\fe{f}{x}=\sqrt{x}\) at \(x=4\) is shown in \hyperref[figure-tangent-line]{Figure~\ref*{figure-tangent-line}}. Here is a calculation of the slope of this line.%
\begin{figure}
\centering
\pushValignCaptionBottom[t]{minipage}{.50\textwidth}{%
\pgfplotsset{every axis/.append style={width=\linewidth}}%
% horizontal alignment 
\parbox{\textwidth}{%
% horizontal alignment 
\begin{align*}
m_{\text{tan}}&=\lim_{h\to0}\frac{\fe{f}{4+h}-\fe{f}{4}}{h}\\
&=\lim_{h\to0}\frac{\sqrt{4+h}-\sqrt{4}}{h}\\
&=\lim_{h\to0}\left(\frac{\sqrt{4+h}-2}{h}\cdot\frac{\sqrt{4+h}+2}{\sqrt{4+h}+2}\right)\\
&=\lim_{h\to0}\frac{4+h-4}{h\left(\sqrt{4+h}+2\right)}\\
&=\lim_{h\to0}\frac{h}{h\left(\sqrt{4+h}+2\right)}\\
&=\lim_{h\to0}\frac{1}{1\left(\sqrt{4+h}+2\right)}\\
&=\frac{1}{1\left(\sqrt{4+0}+2\right)}\\
&=\frac{1}{4}
\end{align*}%
}
}% end body 
{}% caption 
\pushValignCaptionBottom[b]{minipage}{.50\textwidth}{%
\pgfplotsset{every axis/.append style={width=\linewidth}}%
\centering% horizontal alignment 
{
\begin{tikzpicture}
\begin{axis}[xmin=-2,
             xmax=11,
             ymin=-2,
             ymax=11,
             xtick = {-2,0,...,10},
             ytick = {-2,0,...,10},
             minor xtick = {-2,-1,...,11},
             minor ytick = {-2,-1,...,-11},
             legend entries = {$y=\fe{f}{x}$,{tangent line}},]
    \addplot+[variable=t,
        domain=0:3.25,
        ->,
    ]({t^2},{t});
    \addplot+[
        domain=-1.8:10.8,
        <->,
    ]{1+x/4};
    \addplot[
        soldot
    ]coordinates{
        (0,0)
        (4,2)};
\end{axis}
\end{tikzpicture}
}
}% end body 
{\captionof{figure}{\(y=\fe{f}{x}\) and its tangent line at \(x=4\)\label{figure-tangent-line}}
}% caption 
\popValignCaptionBottom
\end{figure}
\par
You should verify that the slope of the tangent line shown in \hyperref[figure-tangent-line]{Figure~\ref*{figure-tangent-line}} is indeed \(\frac{1}{4}\). You should also verify that the equation of the tangent line is \(y=\frac{1}{4}x+1\).%
\end{example}
\typeout{************************************************}
\typeout{Exercises}
\typeout{************************************************}
\section*{Exercises}\label{exercises-18}

\begin{exercisegroup}%
Consider the function \(g\) given by \(\fe{g}{x}=5-\sqrt{4-x}\) graphed in \hyperref[figure-tangent-line-exercise]{Figure~\ref*{figure-tangent-line-exercise}}.%
\begin{figure}
\centering
{
\begin{tikzpicture}
\begin{axis}[xmin=-7,
             xmax=7,
             ymin=-2,
             ymax=12,
             xtick = {-6,-4,...,6},
             ytick = {-2,0,...,10},
             minor xtick = {-7,-6,...,7},
             minor ytick = {-2,-1,...,12},
             legend entries = {$y=\fe{g}{x}$,{tangent line}}]
    \addplot+[variable=t,
        domain=5:1.8,
        ->,
    ]({4-(5-t)^2},{t});
    \addplot+[
        domain=-6.8:6.8,
        <->,
    ]{2.5+x/2};
    \addplot[
        soldot
    ]coordinates{
        (4,5)
        (3,4)};
\end{axis}
\end{tikzpicture}
}
\caption{\(y=\fe{g}{x}\) and its tangent line at \(x=3\)\label{figure-tangent-line-exercise}}
\end{figure}
\begin{exercisegroupbycol}{1}%
\begin{exercisegroupitem}{1. }\phantomsection\hypertarget{exercise-tangent-line}{\null}
 Find the slope of the tangent line shown in \hyperref[figure-tangent-line-exercise]{Figure~\ref*{figure-tangent-line-exercise}} using \[m_{\text{tan}}=\lim_{h\to0}\frac{\fe{g}{3+h}-\fe{g}{3}}{h}\text{.}\] Show work consistent with that illustrated in \hyperref[example-tangent-line]{Example~\ref*{example-tangent-line}}.%
\end{exercisegroupitem}%
\par%
\begin{exercisegroupitem}{2. }\phantomsection\hypertarget{exercise-126}{\null}
Use the line in \hyperref[figure-tangent-line-exercise]{Figure~\ref*{figure-tangent-line-exercise}} to verify your answer to \hyperref[exercise-tangent-line]{Exercise~1}.%
\end{exercisegroupitem}%
\par%
\begin{exercisegroupitem}{3. }\phantomsection\hypertarget{exercise-127}{\null}
State the equation of the tangent line to \(g\) at \(x=3\).%
\end{exercisegroupitem}%
\par%
\end{exercisegroupbycol}%
\end{exercisegroup}%
\typeout{************************************************}
\typeout{Section 3.3 The First Derivative}
\typeout{************************************************}
\section[The First Derivative]{The First Derivative}\label{section-first-derivative}
So far we've seen two applications of expressions of the form \(\lim\limits_{h\to0}\frac{\fe{f}{a+h}-\fe{f}{a}}{h}\). It turns out that this expression is so important in mathematics, the sciences, economics, and many other fields that it deserves a name in and of its own right. We call the expression \terminology{the first derivative of \(f\) at \(a\)}.%
\par
So far we've always fixed the value of \(a\) before making the calculation. There's no reason why we couldn't use a variable for \(a\), make the calculation, and then replace the variable with specific values; in fact, it seems like this might be a better plan all around. This leads us to a definition of \terminology{the first derivative function}.%
\begin{definition}[The First Derivative Function]\label{definition-first-derivative}
If \(f\) is a function of \(x\), then we define \terminology{the first derivative function}, \(\fd{f}\), as \[\fe{\fd{f}}{x}=\lim_{h\to0}\frac{\fe{f}{x+h}-\fe{f}{x}}{h}\text{.}\] The symbols \(\fd{f}\) are read alound as ``\(f\) prime of \(x\)'' or ``\(f\) prime at \(x\)''.%
\end{definition}
\par
As we've already seen, \(\fe{\fd{f}}{a}\) gives us the slope of the tangent line to \(f\) at \(a\).%
\par
We've also seen that if \(s\) is a position function, then \(\fe{\fd{s}}{a}\) gives us the instantaneous velocity at \(a\). It's not too much of a stretch to infer that the velocity function for \(s\) would be \(\fe{v}{t}=\fe{\fd{s}}{t}\).%
\par
Let's find a first derivative.%
\begin{example}\label{example-first-derivative}
Let \(\fe{f}{x}=\frac{3}{2-x}\). We can find \(\fe{\fd{f}}{x}\) as follows.%
\begin{figure}
\centering
\pushValignCaptionBottom[t]{minipage}{.60\textwidth}{%
\pgfplotsset{every axis/.append style={width=\linewidth}}%
% horizontal alignment 
\parbox{\textwidth}{%
% horizontal alignment 
\begin{align*}
\fe{\fd{f}}{x}&=\lim_{h\to0}\frac{\fe{f}{x+h}-\fe{f}{x}}{h}\\
&=\lim_{h\to0}\frac{\frac{3}{2-(x+h)}-\frac{3}{2-x}}{h}\\
&=\lim_{h\to0}\frac{\frac{3}{2-x-h}-\frac{3}{2-x}}{h}\cdot\frac{(2-x-h)(2-x)}{(2-x-h)(2-x)}\\
&=\lim_{h\to0}\frac{3(2-x)-3(2-x-h)}{h(2-x-h)(2-x)}\\
&=\lim_{h\to0}\frac{6-3x-6+3x+3h}{h(2-x-h)(2-x)}\\
&=\lim_{h\to0}\frac{3h}{h(2-x-h)(2-x)}\\
&=\lim_{h\to0}\frac{3}{(2-x-h)(2-x)}\\
&=\frac{3}{(2-x-0)(2-x)}\\
&=\frac{3}{(2-x)^2}
\end{align*}%
}
}% end body 
{}% caption 
\pushValignCaptionBottom[b]{minipage}{.40\textwidth}{%
\pgfplotsset{every axis/.append style={width=\linewidth}}%
\centering% horizontal alignment 
{
\begin{tikzpicture}
\begin{axis}[]
    \addplot[pccplot,
        domain=0:24.1588,
        <->,
    ]({2+0.45*1.1^x},{3/(-0.45*1.1^x)});
    \addplot[pccplot,
        domain=0:30.8317,
        <->,
    ]({2-0.45*1.1^x},{3/(0.45*1.1^x)});
    \addplot[
        soldot
    ]coordinates{
        (1,3)
        (5,-1)};
    \addplot[asymptote
    ]coordinates{
        (2,-7)
        (2,7)};
\end{axis}
\end{tikzpicture}
}
}% end body 
{\captionof{figure}{\(y=\fe{f}{x}=\frac{3}{2-x}\)\label{figure-first-derivative}}
}% caption 
\popValignCaptionBottom
\end{figure}
\end{example}
\typeout{************************************************}
\typeout{Exercises}
\typeout{************************************************}
\section*{Exercises}\label{exercises-19}

\begin{exercisegroup}%
A graph of the function \(f\) defined by \(\fe{f}{x}=\frac{3}{2-x}\) is shown in \hyperref[figure-first-derivative]{Figure~\ref*{figure-first-derivative}} and the formula for \(\fe{\fd{f}}{x}\) is derived in \hyperref[example-first-derivative]{Example~\ref*{example-first-derivative}}.%
\begin{exercisegroupbycol}{1}%
\begin{exercisegroupitem}{1. }\phantomsection\hypertarget{exercise-128}{\null}
Use the formula \(\fe{\fd{f}}{x}=\frac{3}{(2-x)^2}\) to calculate \(\fe{\fd{f}}{1}\) and \(\fe{\fd{f}}{5}\).%
\end{exercisegroupitem}%
\par%
\begin{exercisegroupitem}{2. }\phantomsection\hypertarget{exercise-129}{\null}
Draw onto \hyperref[figure-first-derivative]{Figure~\ref*{figure-first-derivative}} a line through the point \(\point{1}{3}\) with a slope of \(\fe{\fd{f}}{1}\). Also draw a line though the point \(\point{5}{1}\) with a slope of \(\fe{\fd{f}}{5}\). What are the names for the two lines you just drew? What are their equations?%
\end{exercisegroupitem}%
\par%
\begin{exercisegroupitem}{3. }\phantomsection\hypertarget{exercise-130}{\null}
Showing work consistent with that shown in \hyperref[example-first-derivative]{Example~\ref*{example-first-derivative}}, find the formula for \(\fe{\fd{g}}{x}\) where \(\fe{g}{x}=\frac{5}{2x+1}\).%
\end{exercisegroupitem}%
\par%
\end{exercisegroupbycol}%
\end{exercisegroup}%
\begin{exercisegroup}%
Suppose that the elevation of an object (measured in \si{\foot}) is given by \(\fe{s}{t}=-16t^2+112t+5\) where \(t\) is the amount of time that has passed since the object was launched into the air (measured in \si{\second}).%
\begin{exercisegroupbycol}{1}%
\begin{exercisegroupitem}{4. }\phantomsection\hypertarget{exercise-131}{\null}
Use \hyperref[instantaneous-velocity-equation]{3.3.1} to find the formula for the velocity function associated with this motion. The first two lines of your presentation should be an exact copy of \hyperref[instantaneous-velocity-equation]{3.3.1}.%
\par
\begin{align}
\fe{v}{t}&=\fe{\fd{s}}{t}\label{instantaneous-velocity-equation}\\
&=\lim_{h\to0}\frac{\fe{s}{t+h}-\fe{s}{t}}{h}\label{mrow-97}
\end{align}%
\end{exercisegroupitem}%
\par%
\begin{exercisegroupitem}{5. }\phantomsection\hypertarget{exercise-132}{\null}
Find the values of \(\fe{v}{2}\) and \(\fe{v}{5}\). What is the unit on each of these values? What do the values tell you about the motion of the object? Don't just say ``the velocity''\textemdash{}describe what is actually happening to the object \(2\) seconds and \(5\) seconds into its travel.%
\end{exercisegroupitem}%
\par%
\begin{exercisegroupitem}{6. }\phantomsection\hypertarget{exercise-133}{\null}
 Use the velocity function to determine when the object reaches its maximum elevation. (Think about what must be true about the velocity at that instant.) Also, what is the common mathematical term for the point on the parabola \(y=\fe{s}{t}\) that occurs at that value of \(t\)?%
\end{exercisegroupitem}%
\par%
\begin{exercisegroupitem}{7. }\phantomsection\hypertarget{exercise-134}{\null}
Use \hyperref[instantaneous-acceleration-equation]{3.3.3} to find the formula for \(\fe{\fd{v}}{t}\). The first line of your presentation should be an exact copy of \hyperref[instantaneous-acceleration-equation]{3.3.3}.%
\par
\begin{align}
\fe{\fd{v}}{t}&=\lim_{h\to0}\frac{\fe{s}{t+h}-\fe{s}{t}}{h}\label{instantaneous-acceleration-equation}
\end{align}%
\end{exercisegroupitem}%
\par%
\begin{exercisegroupitem}{8. }\phantomsection\hypertarget{exercise-135}{\null}
What is the common name for the function \(\fe{\fd{v}}{t}\)? Is its formula consistent with what you know about objects in freefall on Earth?%
\end{exercisegroupitem}%
\par%
\end{exercisegroupbycol}%
\end{exercisegroup}%
\begin{exerciselist}
\item[9.]\phantomsection\hypertarget{exercise-136}{\null}What is the constant slope of the function \(w\) defined by \(\fe{w}{x}=12\)? Verify this by using \hyperref[definition-first-derivative]{Definition~\ref*{definition-first-derivative}} to find the formula for \(\fe{\fd{w}}{x}\).%
\par\smallskip
\end{exerciselist}
\typeout{************************************************}
\typeout{Section 3.4 Derivative Units}
\typeout{************************************************}
\section[Derivative Units]{Derivative Units}\label{section-derivative-units}
We can think about the instantaneous velocity as being \emph{the instantaneous rate of change} in position. In general, whenever you see the phrase ``rate of change'' you can assume that the rate of change at one instant is being discussed. When we want to discuss average rates of change over a time interval we always say ``average rate of change''.%
\par
In general, if \(f\) is any function, then \emph{\(\fe{\fd{f}}{a}\) tells us the rate of change in \(f\) at \(a\)}. Additionally, if \(f\) is an applied function with an input unit of \(i_{\text{unit}}\) and an output unit of \(f_{\text{unit}}\), then the unit on \(\fe{\fd{f}}{a}\) is \(\frac{f_{\text{unit}}}{i_{\text{unit}}}\). Please note that this unit loses all meaning if it is simplified in any way. Consequently, \emph{we do not simplify derivative units in any way, shape, or form}.%
\par
For example, if \(\fe{v}{t}\) is the velocity of your car (measured in \si{\mile\per\hour}) where \(t\) is the amount of time that has passed since you hit the road (measured in minutes), then the unit on \(\fe{v}{t}\) is \(\frac{\text{mi}/\text{h}}{\text{min}}\).%
\typeout{************************************************}
\typeout{Exercises}
\typeout{************************************************}
\section*{Exercises}\label{exercises-20}

\begin{exercisegroup}%
Determine the unit for the first derivative function for each of the following functions. Remember, \emph{we do not simplify derivative units in any way, shape, or form}.%
\begin{exercisegroupbycol}{1}%
\begin{exercisegroupitem}{1. }\phantomsection\hypertarget{exercise-137}{\null}
\(\fe{V}{r}\) is the volume of a sphere (measured in \si{\milli\liter}) with radius \(r\) (measured in \si{\milli\meter}).%
\end{exercisegroupitem}%
\par%
\begin{exercisegroupitem}{2. }\phantomsection\hypertarget{exercise-138}{\null}
\(\fe{A}{x}\) is the area of a square (measured in \si{\foot\tothe{2}}) with sides of length \(x\) (measured in \si{\foot}).%
\end{exercisegroupitem}%
\par%
\begin{exercisegroupitem}{3. }\phantomsection\hypertarget{exercise-bathtub}{\null}
\(\fe{V}{t}\) is the volume of water in a bathtub (measured in \si{\gallon}) where \(t\) is the amount of time that has elapsed since the bathtub began to drain (measured in minutes).%
\end{exercisegroupitem}%
\par%
\begin{exercisegroupitem}{4. }\phantomsection\hypertarget{exercise-bathtub-rate}{\null}
\(\fe{R}{t}\) is the rate at which a bathtub is draining (measured in \si{\gallon\per\minute}) where \(t\) is the amount of time that has elapsed since the bathtub began to drain (measured in minutes).%
\end{exercisegroupitem}%
\par%
\end{exercisegroupbycol}%
\end{exercisegroup}%
\begin{exerciselist}
\item[5.]\phantomsection\hypertarget{exercise-141}{\null}Akbar was given a formula for the function described in \hyperref[exercise-bathtub]{Exercise~3}. Akbar did some calculations and decided that the value of \(\fe{\fd{V}}{20}\) was (without unit) \(1.5\). Nguyen took one look at Akbar's value and said ``that's wrong'' What is it about Akbar's value that caused Nguyen to dismiss it as wrong?%
\par\smallskip
\item[6.]\phantomsection\hypertarget{exercise-142}{\null}After a while Nguyen convinced Akbar that he was wrong, so Akbar set about doing the calculation over again. This time Akbar came up with a value of \num{-12528}. Nguyen took one look at Akbar's value and declared ``still wrong''. What's the problem now?%
\par\smallskip
\end{exerciselist}
\begin{exercisegroup}%
Consider the function described in \hyperref[exercise-bathtub-rate]{Exercise~4}.%
\begin{exercisegroupbycol}{1}%
\begin{exercisegroupitem}{7. }\phantomsection\hypertarget{exercise-143}{\null}
What would it mean if the value of \(\fe{R}{t}\) was zero for all \(t\gt0\)?%
\end{exercisegroupitem}%
\par%
\begin{exercisegroupitem}{8. }\phantomsection\hypertarget{exercise-144}{\null}
What would it mean if the value of \(\fe{\fd{R}}{t}\) was zero for all \(0\lt t\lt2.25\)?%
\end{exercisegroupitem}%
\par%
\end{exercisegroupbycol}%
\end{exercisegroup}%
\typeout{************************************************}
\typeout{Chapter 4 Functions, Derivatives, and Antiderivatives}
\typeout{************************************************}
\chapter[Functions, Derivatives, and Antiderivatives]{Functions, Derivatives, and Antiderivatives}\label{chapter-functions-derivatives-antiderivatives}
\typeout{************************************************}
\typeout{Section 4.1 Graph Features}
\typeout{************************************************}
\section[Graph Features]{Graph Features}\label{section-graph-features}
Functions, derivatives, and antiderivatives have many entangled properties. For example, over intervals where the first derivative of a function is always positive, we know that the function itself is always increasing. (Do you understand why?)%
\par
Many of these relationships can be expressed graphically. Consequently, it is imperative that you fully understand the meaning of some commonly used graphical expressions. These expressions are loosely defined in \hyperref[definition-graph-features]{Definition~\ref*{definition-graph-features}}.%
\begin{definition}[Some Common Graphical Phrases]\label{definition-graph-features}
These are loose, informal definitions. Technical, precise definitions can be found elsewhere.%
\begin{multicols}{2}
\begin{description}
\item[The function is positive.]{}This means that the vertical-coordinate of the point on the function is positive. As such, a function is positive whenever it lies above the horizontal axis.\item[The function is increasing.]{}This means that the vertical-coordinate of the function consistently increases as you move along the curve from left to right. Linear functions with positive slope are always increasing.\item[The function is concave up.]{}A function is concave up at \(a\) if the tangent line to the function at \(a\) lies below the curve. An upward opening parabola is everywhere concave up.\item[The function is negative.]{}This means that the vertical-coordinate of the point on the function is negative. As such, a function is negative whenever it lies below the horizontal axis.\item[The function is decreasing.]{}This means that the vertical-coordinate of the function consistently decreases as you move along the curve from left to right. Linear functions with negative slope are always decreasing.\item[The function is concave down.]{}A function is concave down at \(a\) if the tangent line to the function at \(a\) lies above the curve. A downward opening parabola is everywhere concave down.\end{description}
\end{multicols}
\end{definition}
\typeout{************************************************}
\typeout{Exercises}
\typeout{************************************************}
\section*{Exercises}\label{exercises-21}

\begin{exercisegroup}%
Answer each of the following questions in reference to the function shown in \hyperref[figure-graph-features]{Figure~\ref*{figure-graph-features}}. Each answer is an interval (or intervals) along the \(x\)-axis. Use interval notation when expressing your answers. Make each interval as wide as possible; that is, do not break an interval into pieces if the interval does not need to be broken up. Assume that the slope of the function is constant on \(\ointerval{-\infty}{-5}\), \(\ointerval{3}{4}\), and \(\ointerval{4}{\infty}\).%
\begin{figure}
\centering
{
\begin{tikzpicture}
\begin{axis}[]
    \addplot[pccplot,
        domain=-6.7:-5,
        <-,
    ]{3};
    \addplot[pccplot,
        domain=-5:-1,
        -,
    ]{(x+3)^2-1};
    \addplot[pccplot,
        domain=-1:3,
        -,
    ]{3-cos(90*x)};
    \addplot[pccplot,
        domain=3:4,
        -,
    ]{-2*(x-3)-2};
    \addplot[pccplot,
        domain=4:6.7,
        ->,
    ]{-2/3*(x-4)+6};
    \addplot[
        soldot
    ]coordinates{
        (3,3)
        (4,6)};
    \addplot[
        holdot
    ]coordinates{
        (3,-2)
        (4,-4)};
\end{axis}
\end{tikzpicture}
}
\caption{\(y=\fe{f}{x}\)\label{figure-graph-features}}
\end{figure}
\begin{exercisegroupbycol}{1}%
\begin{exercisegroupitem}{1. }\phantomsection\hypertarget{exercise-145}{\null}
Over what intervals is the function positive?%
\end{exercisegroupitem}%
\par%
\begin{exercisegroupitem}{2. }\phantomsection\hypertarget{exercise-146}{\null}
Over what intervals is the function negative?%
\end{exercisegroupitem}%
\par%
\begin{exercisegroupitem}{3. }\phantomsection\hypertarget{exercise-147}{\null}
Over what intervals is the function increasing?%
\end{exercisegroupitem}%
\par%
\begin{exercisegroupitem}{4. }\phantomsection\hypertarget{exercise-148}{\null}
Over what intervals is the function decreasing?%
\end{exercisegroupitem}%
\par%
\begin{exercisegroupitem}{5. }\phantomsection\hypertarget{exercise-149}{\null}
Over what intervals is the function concave up?%
\end{exercisegroupitem}%
\par%
\begin{exercisegroupitem}{6. }\phantomsection\hypertarget{exercise-150}{\null}
Over what intervals is the function concave down?%
\end{exercisegroupitem}%
\par%
\begin{exercisegroupitem}{7. }\phantomsection\hypertarget{exercise-151}{\null}
Over what intervals is the function linear?%
\end{exercisegroupitem}%
\par%
\begin{exercisegroupitem}{8. }\phantomsection\hypertarget{exercise-152}{\null}
Over what intervals is the function constant?%
\end{exercisegroupitem}%
\par%
\end{exercisegroupbycol}%
\end{exercisegroup}%
\typeout{************************************************}
\typeout{Section 4.2 Graphical Derivatives}
\typeout{************************************************}
\section[Graphical Derivatives]{Graphical Derivatives}\label{section-graphical-derivatives}
Much information about a function's first derivative can be gleaned simply by looking at a graph of the function. In fact, a person with good visual skills can ``see'' the graph of the derivative while looking at the graph of the function. This activity focuses on helping you develop that skill.%
\typeout{************************************************}
\typeout{Exercises}
\typeout{************************************************}
\section*{Exercises}\label{exercises-22}

\begin{exercisegroup}%
A parabolic function is shown in \hyperref[figure-graphical-derivative]{Figure~\ref*{figure-graphical-derivative}}. The next several questions are in reference to that function.%
\begin{figure}
\centering
\pushValignCaptionBottom[b]{minipage}{.30\textwidth}{%
\pgfplotsset{every axis/.append style={width=\linewidth}}%
\centering% horizontal alignment 
{
\begin{tikzpicture}
\begin{axis}[xmin=-8,xmax=8,ymin=-16,ymax=8,xtick={-6,-4,...,6}, minor xtick={-7,-6,...,7},ytick={-14,-12,...,6}, minor ytick={-15,-14,...,7},y post scale=1.5]
    \addplot+[
        domain=-5.5:7.5,
        <->,
    ]{-0.5*x^2+x+5.5};
    \addplot[
        soldot
    ]coordinates{
        (-5,-12)
        (-3,-2)
        (-1,4)
        (1,6)
        (3,4)
        (5,-2)
        (7,-12)};
\end{axis}
\end{tikzpicture}
}
}% end body 
{\captionof{figure}{\(y=\fe{g}{x}\)\label{figure-graphical-derivative}}
}% caption 
\pushValignCaptionBottom[b]{minipage}{.30\textwidth}{%
\pgfplotsset{every axis/.append style={width=\linewidth}}%
\centering% horizontal alignment 
{
\begin{tikzpicture}
\begin{axis}[xmin=-8,xmax=8,ymin=-16,ymax=8,xtick={-6,-4,...,6}, minor xtick={-7,-6,...,7},ytick={-14,-12,...,6}, minor ytick={-15,-14,...,7},y post scale=1.5]
    \addplot+[
        domain=-5.5:7.5,
        <->,
    ]{-0.5*x^2+x+5.5};
\end{axis}
\end{tikzpicture}
}
}% end body 
{\captionof{figure}{\(y=\fe{g}{x}\)\label{figure-graphical-derivative-plot}}
}% caption 
\pushValignCaptionBottom[b]{minipage}{.30\textwidth}{%
\pgfplotsset{every axis/.append style={width=\linewidth}}%
\centering% horizontal alignment 
\begin{tabular}{cc}\hrulethick
\(x\)&\(y\)\\\hrulemedium
\(-5\)&\(6\)\\
\(-3\)&\\
\(-1\)&\(2\)\\
\(1\)&\\
\(3\)&\(-2\)\\
\(5\)&\(-4\)\\
\(7\)&
\end{tabular}
}% end body 
{\captionof{table}{\(y=\fe{\fd{g}}{x}\)\label{table-graphical-derivative}}
}% caption 
\popValignCaptionBottom
\end{figure}
\begin{exercisegroupbycol}{1}%
\begin{exercisegroupitem}{1. }\phantomsection\hypertarget{exercise-153}{\null}
Several values of the function \(\fd{g}\) are given in \hyperref[table-graphical-derivative]{Table~\ref*{table-graphical-derivative}}. For each given value draw a nice long line segment at the corresponding point on \(g\) whose slope is equal to the value of \(g\). If we think of these line segments as actual lines, what do we call the lines?%
\end{exercisegroupitem}%
\par%
\begin{exercisegroupitem}{2. }\phantomsection\hypertarget{exercise-154}{\null}
What is the value of \(\fd{g}\) at \(1\)? How do you know? Go ahead and enter that value into \hyperref[table-graphical-derivative]{Table~\ref*{table-graphical-derivative}}.%
\end{exercisegroupitem}%
\par%
\begin{exercisegroupitem}{3. }\phantomsection\hypertarget{exercise-155}{\null}
The function \(g\) is symmetric across the line \(x=1\); that is, if we move equal distance to the left and right from this line the corresponding \(y\)-coordinates on \(g\) are always equal. Notice that the slopes of the tangent lines are ``equal but opposite'' at points that are equally removed from the axis of symmetry; this is reflected in the values of \(\fe{\fd{g}}{1}\) and \(\fe{\fd{g}}{3}\). Use the idea of ``equal but opposite slope equidistance from the axis of symmetry'' to complete \hyperref[table-graphical-derivative]{Table~\ref*{table-graphical-derivative}}.%
\end{exercisegroupitem}%
\par%
\begin{exercisegroupitem}{4. }\phantomsection\hypertarget{exercise-156}{\null}
Plot the points from \hyperref[table-graphical-derivative]{Table~\ref*{table-graphical-derivative}} onto \hyperref[figure-graphical-derivative-plot]{Figure~\ref*{figure-graphical-derivative-plot}} and connect the dots. Determine the formula for the resultant linear function.%
\end{exercisegroupitem}%
\par%
\begin{exercisegroupitem}{5. }\phantomsection\hypertarget{exercise-157}{\null}
The formula for \(g\) is \(-0.5x^2+x+5.5\). Use \hyperref[definition-first-derivative]{Definition~\ref*{definition-first-derivative}} to determine the formula for \(\fe{\fd{g}}{x}\).%
\end{exercisegroupitem}%
\par%
\begin{exercisegroupitem}{6. }\phantomsection\hypertarget{exercise-158}{\null}
The line you drew onto \hyperref[figure-graphical-derivative-plot]{Figure~\ref*{figure-graphical-derivative-plot}} is \emph{not} a tangent line to \(g\). Just what exactly is this line?%
\end{exercisegroupitem}%
\par%
\end{exercisegroupbycol}%
\end{exercisegroup}%
\begin{exercisegroup}%
A function \(f\) is shown in \hyperref[figure-cubic]{Figure~\ref*{figure-cubic}} and the corresponding first derivative function \(\fd{f}\) is shown in \hyperref[figure-cubic-derivative]{Figure~\ref*{figure-cubic-derivative}}. Answer each of the following questions referencing these two functions.%
\begin{figure}
\centering
\pushValignCaptionBottom[b]{minipage}{.40\textwidth}{%
\pgfplotsset{every axis/.append style={width=\linewidth}}%
\centering% horizontal alignment 
{
\begin{tikzpicture}
\begin{axis}[samples=50]
    \addplot+[
        domain=-4.2:4,
        <->,
    ]{x^3/3-4*x+1};
    \addplot[
        soldot
    ]coordinates{
        (-2,6.33333)
        (0,1)
        (3,-2)};
\end{axis}
\end{tikzpicture}
}
}% end body 
{\captionof{figure}{\(y=\fe{f}{x}\)\label{figure-cubic}}
}% caption 
\pushValignCaptionBottom[b]{minipage}{.40\textwidth}{%
\pgfplotsset{every axis/.append style={width=\linewidth}}%
\centering% horizontal alignment 
{
\begin{tikzpicture}
\begin{axis}[]
    \addplot+[
        domain=-3.3:3.3,
        <->,
    ]{x^2-4};
    \addplot[
        soldot
    ]coordinates{
        (-2,0)
        (0,-4)
        (3,5)};
\end{axis}
\end{tikzpicture}
}
}% end body 
{\captionof{figure}{\(y=\fe{\fd{f}}{x}\)\label{figure-cubic-derivative}}
}% caption 
\popValignCaptionBottom
\end{figure}
\begin{exercisegroupbycol}{1}%
\begin{exercisegroupitem}{7. }\phantomsection\hypertarget{exercise-159}{\null}
Draw the tangent line to \(f\) at the three points indicated in \hyperref[figure-cubic]{Figure~\ref*{figure-cubic}} after first using the graph of \(\fd{f}\) to determine the exact slope of the respective tangent lines.%
\end{exercisegroupitem}%
\par%
\begin{exercisegroupitem}{8. }\phantomsection\hypertarget{exercise-160}{\null}
Write a sentence relating the slope of the tangent line to \(f\) with the corresponding \(y\)-coordinate on \(\fd{f}\).%
\end{exercisegroupitem}%
\par%
\begin{exercisegroupitem}{9. }\phantomsection\hypertarget{exercise-161}{\null}
Copy each of the following sentences onto your paper and supply the words or phrases that correctly complete each sentence.%
\begin{itemize}[label=\textbullet]
\item{}Over the interval where \(\fd{f}\) is negative, \(f\) is \dots{}\item{}Over the interval where \(\fd{f}\) is positive, \(f\) is \dots{}\item{}Over the interval where \(\fd{f}\) is increasing, \(f\) is \dots{}\item{}Over the interval where \(\fd{f}\) is decreasing, \(f\) is \dots{}\end{itemize}
\end{exercisegroupitem}%
\par%
\end{exercisegroupbycol}%
\end{exercisegroup}%
\begin{exerciselist}
\item[10.]\phantomsection\hypertarget{exercise-162}{\null}In each of \hyperref[figure-sine]{Figures~\ref*{figure-sine}} and \hyperref[figure-rational]{4.2.17} a function (the solid curve) is given; both of these functions are symmetric about the \(y\)-axis. The first derivative of each function (the dash-dotted curves) have been drawn over the interval \(\ointerval{0}{7}\). Use the given portion of the first derivative together with the symmetry of the function to help you draw each first derivative over the interval \(\ointerval{-7}{0}\).%
\begin{figure}
\centering
\pushValignCaptionBottom[b]{minipage}{.40\textwidth}{%
\pgfplotsset{every axis/.append style={width=\linewidth}}%
\centering% horizontal alignment 
{
\begin{tikzpicture}
\begin{axis}[legend style={at={(1,1)},anchor=south east},legend entries={$y=\fe{g}{x}$,$y=\fe{\fd{g}}{x}$},samples=50,]
    \addplot+[
        domain=-6.8:6.8,
        <->,
    ]{4-2*cos(x*60)};
    \addplot+[
        domain=0:6.8,
        ->,
    ]{2*sin(x*60)};
\end{axis}
\end{tikzpicture}
}
}% end body 
{\captionof{figure}{\(y=\fe{g}{x}\) and \(y=\fe{\fd{g}}{x}\)\label{figure-sine}}
}% caption 
\pushValignCaptionBottom[b]{minipage}{.40\textwidth}{%
\pgfplotsset{every axis/.append style={width=\linewidth}}%
\centering% horizontal alignment 
{
\begin{tikzpicture}
\begin{axis}[legend style={at={(1,1)},anchor=south east},legend entries={$y=\fe{g}{x}$,$y=\fe{\fd{g}}{x}$,$y=\frac{1}{2}$}]
    \addplot+[
        variable=t,
        domain=0:32.75,
        <->,
    ]({-0.2998*1.1^t},{(0.2998*1.1^t)/2+1/(6*(0.2998*1.1^t)^3)+1/6});
    \addplot+[
        variable=t,
        domain=0:33.957,
        <->,
    ]({0.26726*1.1^t},{1/2-1/(2*(0.26726*1.1^t)^2)});
    \addplot+[asymptote,
        domain=-7:7,
        <->,
    ]{1/2};
    \addplot+[
        variable=t,
        domain=0:32.75,
        <->,
    ]({0.2998*1.1^t},{(0.2998*1.1^t)/2+1/(6*(0.2998*1.1^t)^3)+1/6});
\end{axis}
\end{tikzpicture}
}
}% end body 
{\captionof{figure}{\(y=\fe{k}{x}\) and \(y=\fe{\fd{k}}{x}\)\label{figure-rational}}
}% caption 
\popValignCaptionBottom
\end{figure}
\par\smallskip
\begin{exercisegroup}%
A graph of the function \(f\) given by \(\fe{f}{x}=\frac{1}{x}\) is show in \hyperref[figure-one-over-x]{Figure~\ref*{figure-one-over-x}}.%
\begin{figure}
\centering
\pushValignCaptionBottom[b]{minipage}{.40\textwidth}{%
\pgfplotsset{every axis/.append style={width=\linewidth}}%
\centering% horizontal alignment 
{
\begin{tikzpicture}
\begin{axis}[]
    \addplot[pccplot,
        variable=t,
        domain=0:39.278,
        <->,
    ]({-1/6.5*1.1^t},{1/(-1/6.5*1.1^t)});
    \addplot[pccplot,
        variable=t,
        domain=0:39.278,
        <->,
    ]({1/6.5*1.1^t},{1/(1/6.5*1.1^t)});
\end{axis}
\end{tikzpicture}
}
}% end body 
{\captionof{figure}{\(y=\fe{f}{x}=\frac{1}{x}\)\label{figure-one-over-x}}
}% caption 
\pushValignCaptionBottom[b]{minipage}{.40\textwidth}{%
\pgfplotsset{every axis/.append style={width=\linewidth}}%
\centering% horizontal alignment 
{
\begin{tikzpicture}
\begin{axis}[]
\end{axis}
\end{tikzpicture}
}
}% end body 
{\captionof{figure}{\(y=\fe{\fd{f}}{x}\)\label{figure-derivative-of-one-over-x}}
}% caption 
\popValignCaptionBottom
\end{figure}
\begin{exercisegroupbycol}{1}%
\begin{exercisegroupitem}{11. }\phantomsection\hypertarget{exercise-negative-derivative-of-one-over-x}{\null}
Except at \(0\), there is something that is always true about the value of \(\fd{f}\); what is the common trait?%
\end{exercisegroupitem}%
\par%
\begin{exercisegroupitem}{12. }\phantomsection\hypertarget{exercise-derivative-of-one-over-x}{\null}
Use \hyperref[definition-first-derivative]{Definition~\ref*{definition-first-derivative}} to find the formula for \(\fe{\fd{f}}{x}\).%
\end{exercisegroupitem}%
\par%
\begin{exercisegroupitem}{13. }\phantomsection\hypertarget{exercise-165}{\null}
Does the formula for \(\fe{\fd{f}}{x}\) support your answer to \hyperref[exercise-derivative-of-one-over-x]{Exercise~12}?%
\end{exercisegroupitem}%
\par%
\begin{exercisegroupitem}{14. }\phantomsection\hypertarget{exercise-asymptotes-one-over-x}{\null}
Use the formula for \(\fe{\fd{f}}{x}\) to determine the horizontal and vertical asymptotes to the graph of \(y=\fe{\fd{f}}{x}\).%
\end{exercisegroupitem}%
\par%
\begin{exercisegroupitem}{15. }\phantomsection\hypertarget{exercise-167}{\null}
 Keeping it simple, draw onto \hyperref[figure-derivative-of-one-over-x]{Figure~\ref*{figure-derivative-of-one-over-x}} a curve with the asymptotes found in \hyperref[exercise-asymptotes-one-over-x]{Exercise~14} and the property determined in \hyperref[exercise-negative-derivative-of-one-over-x]{Exercise~11}. Does the curve you drew have the properties you would expect to see in the first derivative of \(f\)? For example, \(f\) is concave down over \(\ointerval{-\infty}{0}\), and concave up over \(\ointerval{0}{\infty}\); what are the corresponding differences in the behavior of \(\fd{f}\) over those two intervals?%
\end{exercisegroupitem}%
\par%
\end{exercisegroupbycol}%
\end{exercisegroup}%
\item[16.]\phantomsection\hypertarget{exercise-168}{\null}A graph of the function \(g\) is shown in \hyperref[figure-arctangent]{Figure~\ref*{figure-arctangent}}. The absolute minimum value ever obtained by \(\fd{g}\) is \(3\). With that in mind, draw \(\fd{g}\) onto \hyperref[figure-derivative-of-arctangent]{Figure~\ref*{figure-derivative-of-arctangent}}. Make sure that you draw and label any and all necessary asymptotes. Make sure that your graph of \(\fd{g}\) adequately reflects the symmetry in the graph of \(g\).%
\begin{figure}
\centering
\pushValignCaptionBottom[b]{minipage}{.40\textwidth}{%
\pgfplotsset{every axis/.append style={width=\linewidth}}%
\centering% horizontal alignment 
{
\begin{tikzpicture}
\begin{axis}[]
    \addplot+[smooth,
        domain=-6.8:6.8,
        <->,
    ]{2-atan((x-1)*sqrt(3))/30};
    \addplot[asymptote,
        domain=-7:7,
        <->,
    ]{5};
    \addplot[asymptote,
        domain=-7:7,
        <->,
    ]{-1};
\end{axis}
\end{tikzpicture}
}
}% end body 
{\captionof{figure}{\(y=\fe{g}{x}\)\label{figure-arctangent}}
}% caption 
\pushValignCaptionBottom[b]{minipage}{.40\textwidth}{%
\pgfplotsset{every axis/.append style={width=\linewidth}}%
\centering% horizontal alignment 
{
\begin{tikzpicture}
\begin{axis}[]
\end{axis}
\end{tikzpicture}
}
}% end body 
{\captionof{figure}{\(y=\fe{\fd{g}}{x}\)\label{figure-derivative-of-arctangent}}
}% caption 
\popValignCaptionBottom
\end{figure}
\par\smallskip
\end{exerciselist}
\begin{exercisegroup}%
A function, \(w\), is shown in \hyperref[figure-inflection-points]{Figure~\ref*{figure-inflection-points}}. A larger version of \hyperref[figure-inflection-points]{Figure~\ref*{figure-inflection-points}} is available in {$\langle\langle$Appendix B (page B3)$\rangle\rangle$}. Answer each of the following questions in reference to this function.%
\begin{figure}
\centering
\pushValignCaptionBottom[b]{minipage}{.40\textwidth}{%
\pgfplotsset{every axis/.append style={width=\linewidth}}%
\centering% horizontal alignment 
{
\begin{tikzpicture}
\begin{axis}[xmin=-1,xmax=10,ymin=-1,ymax=10,xtick={-1,0,...,10}, minor xtick={-1,-0.5,...,10},ytick={0,1,...,10}, minor ytick={-1,-0.5,...,10},samples=100]
    \addplot+[
        domain=-0.1:8.25,
        <->,
    ]{x^5/60-5/16*x^4+19/9*x^3-13/2*x^2+28/3*x};
\end{axis}
\end{tikzpicture}
}
}% end body 
{\captionof{figure}{\(y=\fe{w}{x}\)\label{figure-inflection-points}}
}% caption 
\pushValignCaptionBottom[b]{minipage}{.40\textwidth}{%
\pgfplotsset{every axis/.append style={width=\linewidth}}%
\centering% horizontal alignment 
\begin{tabular}{cc}\hrulethick
Interval&Properties\\\hrulemedium
\(\ointerval{-\infty}{2}\)&\\
\(\ointerval{2}{3.25}\)&\\
\(\ointerval{3.25}{4}\)&\\
\(\ointerval{4}{6}\)&\\
\(\ointerval{6}{7}\)&\\
\(\ointerval{7}{\infty}\)&
\end{tabular}
}% end body 
{\captionof{table}{Properties of \(\fd{w}\) (positive, negative, increasing, or decreasing)\label{table-inflection-points}}
}% caption 
\popValignCaptionBottom
\end{figure}
\begin{exercisegroupbycol}{1}%
\begin{exercisegroupitem}{17. }\phantomsection\hypertarget{exercise-169}{\null}
 An \terminology{inflection point} is a point where the function is continuous and the concavity of the function changes. The inflection points on \(w\) occur at \(2\), \(3.25\), and \(6\). With that in mind, over each interval stated in \hyperref[table-inflection-points]{Table~\ref*{table-inflection-points}} exactly two of the properties from its caption apply to \(\fd{w}\). Complete \hyperref[table-inflection-points]{Table~\ref*{table-inflection-points}} with the appropriate pairs of properties.%
\end{exercisegroupitem}%
\par%
\begin{exercisegroupitem}{18. }\phantomsection\hypertarget{exercise-170}{\null}
In \hyperref[table-values-for-w-prime]{Table~\ref*{table-values-for-w-prime}}, three possible values are given for \(\fd{w}\) at several values of \(x\). In each case, one of the values is correct. Use tangent lines to \(w\) to determine each of the correct values. (This is where you probably want to use the graph on {$\langle\langle$page B3$\rangle\rangle$}.)%
\begin{figure}
\centering
\pushValignCaptionBottom[b]{minipage}{.40\textwidth}{%
\pgfplotsset{every axis/.append style={width=\linewidth}}%
\centering% horizontal alignment 
\begin{tabular}{cc}\hrulethick
\(x\)&Proposed values\\\hrulemedium
\(0\)&\(\sfrac{2}{3}\) or \(\sfrac{8}{3}\) or \(\sfrac{28}{3}\)\\
\(1\)&\(\sfrac{1}{2}\) or \(\sfrac{3}{2}\) or \(\sfrac{5}{2}\)\\
\(3\)&\(\sfrac{1}{3}\) or \(1\) or \(3\)\\
\(5\)&\(-\sfrac{1}{2}\) or \(-1\) or \(-\sfrac{3}{2}\)\\
\(6\)&\(-\sfrac{4}{3}\) or \(-\sfrac{8}{3}\) or \(-4\)\\
\(8\)&\(1\) or \(6\) or \(12\)
\end{tabular}
}% end body 
{\captionof{table}{Choose the correct values for \(\fd{w}\)\label{table-values-for-w-prime}}
}% caption 
\pushValignCaptionBottom[b]{minipage}{.40\textwidth}{%
\pgfplotsset{every axis/.append style={width=\linewidth}}%
\centering% horizontal alignment 
{
\begin{tikzpicture}
\begin{axis}[xmin=-1,xmax=10,ymin=-5,ymax=5,xtick={-1,0,...,10}, minor xtick={-1,-0.5,...,10},ytick={-5,-4,...,5}, minor ytick={-5,-5.5,...,5},]
\end{axis}
\end{tikzpicture}
}
}% end body 
{\captionof{figure}{\(y=\fe{\fd{w}}{x}\)\label{figure-w-prime}}
}% caption 
\popValignCaptionBottom
\end{figure}
\end{exercisegroupitem}%
\par%
\begin{exercisegroupitem}{19. }\phantomsection\hypertarget{exercise-171}{\null}
The value of \(\fd{w}\) is the same at \(2\), \(4\), and \(7\). What is this common value?%
\end{exercisegroupitem}%
\par%
\begin{exercisegroupitem}{20. }\phantomsection\hypertarget{exercise-172}{\null}
Put it all together and draw \(\fd{w}\) onto \hyperref[figure-w-prime]{Figure~\ref*{figure-w-prime}}.%
\end{exercisegroupitem}%
\par%
\end{exercisegroupbycol}%
\end{exercisegroup}%
\typeout{************************************************}
\typeout{Section 4.3 Nondifferentiability}
\typeout{************************************************}
\section[Nondifferentiability]{Nondifferentiability}\label{section-nondifferentiability}
A function is said to be nondifferentiable at any value its first derivative is undefined.  There are three graphical behaviors that lead to non-differentiability.%
\begin{itemize}[label=\textbullet]
\item{}\(f\) is nondifferentiable at \(a\) if \(f\) is discontinuous at \(a\).\item{}\(f\) is nondifferentiable at \(a\) if the slope of \(f\) is different from the left and right at \(a\).\item{}\(f\) is nondifferentiable at \(a\) if \(f\) has a vertical tangent line at \(a\).\end{itemize}
\typeout{************************************************}
\typeout{Exercises}
\typeout{************************************************}
\section*{Exercises}\label{exercises-23}

\begin{exercisegroup}%
Consider the function \(k\) shown in \hyperref[figure-nondifferentiable]{Figure~\ref*{figure-nondifferentiable}}.%
\begin{figure}
\centering
\pushValignCaptionBottom[b]{minipage}{.40\textwidth}{%
\pgfplotsset{every axis/.append style={width=\linewidth}}%
\centering% horizontal alignment 
{
\begin{tikzpicture}
\begin{axis}[]
    \addplot[pccplot,
        domain=-6.8:-3,
        <-,
    ]{-2*(abs(x+4))^(1/3)*(x+4)/abs(x+4)+1};
    \addplot[pccplot,
        domain=-3:-1,
        -,
    ]{2*x+5};
    \addplot[pccplot,
        domain=-1:4,
        -,
    ]{2*x-3};
    \addplot[pccplot,
        domain=4:6.8,
        ->,
    ]{-4*(x-4)+5};
    \addplot[
        soldot
    ]coordinates{
        (-1,3)};
    \addplot[
        holdot
    ]coordinates{
        (-1,-5)};
\end{axis}
\end{tikzpicture}
}
}% end body 
{\captionof{figure}{\(y=\fe{k}{x}\)\label{figure-nondifferentiable}}
}% caption 
\pushValignCaptionBottom[b]{minipage}{.40\textwidth}{%
\pgfplotsset{every axis/.append style={width=\linewidth}}%
\centering% horizontal alignment 
{
\begin{tikzpicture}
\begin{axis}[]
\end{axis}
\end{tikzpicture}
}
}% end body 
{\captionof{figure}{\(y=\fe{\fd{k}}{x}\)\label{figure-nondifferentiable-derivative}}
}% caption 
\popValignCaptionBottom
\end{figure}
\begin{exercisegroupbycol}{1}%
\begin{exercisegroupitem}{1. }\phantomsection\hypertarget{exercise-173}{\null}
There are four values where \(k\) is nondifferentiable; what are these values?%
\end{exercisegroupitem}%
\par%
\begin{exercisegroupitem}{2. }\phantomsection\hypertarget{exercise-174}{\null}
Draw \(\fd{k}\) onto \hyperref[figure-nondifferentiable-derivative]{Figure~\ref*{figure-nondifferentiable-derivative}}.%
\end{exercisegroupitem}%
\par%
\end{exercisegroupbycol}%
\end{exercisegroup}%
\begin{exercisegroup}%
Consider the function \(g\) shown in \hyperref[figure-bounce]{Figure~\ref*{figure-bounce}}.%
\begin{figure}
\centering
\pushValignCaptionBottom[b]{minipage}{.40\textwidth}{%
\pgfplotsset{every axis/.append style={width=\linewidth}}%
\centering% horizontal alignment 
{
\begin{tikzpicture}
\begin{axis}[samples = 289]
    \addplot+[
        domain=-6.8:6.8,
    ]{5*abs(sin(36*x))-2};
\end{axis}
\end{tikzpicture}
}
}% end body 
{\captionof{figure}{\(y=\fe{g}{x}\)\label{figure-bounce}}
}% caption 
\pushValignCaptionBottom[b]{minipage}{.40\textwidth}{%
\pgfplotsset{every axis/.append style={width=\linewidth}}%
\centering% horizontal alignment 
{
\begin{tikzpicture}
\begin{axis}[]
    \addplot+[
        domain=-5:-2.5,
        -,
    ]{3.1415926*abs(cos(36*x))};
    \addplot[holdot] coordinates {(-5,3.1415926)};
\end{axis}
\end{tikzpicture}
}
}% end body 
{\captionof{figure}{\(y=\fe{\fd{g}}{x}\)\label{figure-bounce-derivative}}
}% caption 
\popValignCaptionBottom
\end{figure}
\begin{exercisegroupbycol}{1}%
\begin{exercisegroupitem}{3. }\phantomsection\hypertarget{exercise-175}{\null}
\(\fd{g}\) has been drawn onto \hyperref[figure-bounce-derivative]{Figure~\ref*{figure-bounce-derivative}}  over the interval \(\ointerval{-5}{-2.5}\). Use the piecewise symmetry and periodic behavior of \(g\) to help you draw the remainder of \(\fd{g}\) over \(\ointerval{-7}{7}\).%
\end{exercisegroupitem}%
\par%
\begin{exercisegroupitem}{4. }\phantomsection\hypertarget{exercise-176}{\null}
What six-syllable word applies to \(g\) at \(-5\), \(0\), and \(5\)?%
\end{exercisegroupitem}%
\par%
\begin{exercisegroupitem}{5. }\phantomsection\hypertarget{exercise-177}{\null}
What five-syllable and six-syllable words apply to \(\fd{g}\) at \(-5\), \(0\), and \(5\)?%
\end{exercisegroupitem}%
\par%
\end{exercisegroupbycol}%
\end{exercisegroup}%
\typeout{************************************************}
\typeout{Section 4.4 Higher Order Derivatives}
\typeout{************************************************}
\section[Higher Order Derivatives]{Higher Order Derivatives}\label{section-higher-order-derivatives}
Seeing as the first derivative of \(f\) is a function in its own right, \(\fd{f}\) must have its own first derivative.  The first derivative of \(\fd{f}\) is the \terminology{second derivative} of \(f\) and is symbolized as \(\sd{f}\) (\(f\) double-prime).  Likewise, \(\td{f}\) (\(f\) triple-prime) is the first derivative of \(\sd{f}\), the second derivative of \(\fd{f}\), and the \terminology{third derivative} of \(f\).%
\par
All of the graphical relationships you’ve established between \(f\) and \(\fd{f}\) work their way down the derivative chain; this is illustrated in \hyperref[table-f-prime-and-f]{Table~\ref*{table-f-prime-and-f}}.%
\begin{table}
\centering
\caption{\(\fd{f}\) and \(f\), \(\sd{f}\) and \(\fd{f}\), \(\td{f}\) and \(\sd{f}\)\label{table-f-prime-and-f}}
\begin{tabular}{ll}\hrulethick
When \(\fd{f}\) is \dots{}&\(f\) is \dots{}\\
When \(\sd{f}\) is \dots{}&\(\fd{f}\) is \dots{}\\
When \(\td{f}\) is \dots{}&\(\sd{f}\) is \dots{}\\\hrulemedium
Positive&Increasing\\
Negative&Decreasing\\
Constantly Zero&Constant\\
Increasing&Concave Up\\
Decreasing&Concave Down\\
Constant Nonzero&Linear
\end{tabular}
\end{table}
\typeout{************************************************}
\typeout{Exercises}
\typeout{************************************************}
\section*{Exercises}\label{exercises-24}

\begin{exerciselist}
\item[1.]\phantomsection\hypertarget{exercise-178}{\null}Extrapolating from \hyperref[table-f-prime-and-f]{Table~\ref*{table-f-prime-and-f}}, what must be true about \(f\) over intervals where \(\sd{f}\) is, respectively: positive, negative, and constantly zero?%
\par\smallskip
\item[2.]\phantomsection\hypertarget{exercise-179}{\null}A function, \(g\), and its first three derivatives are shown in \hyperref[figure-derivatives-one]{Figures~\ref*{figure-derivatives-one}}\textendash{}\hyperref[figure-derivatives-four]{4.4.7}, although not in that order.  Determine which curve is which function (\(g,\fd{g},\sd{g},\td{g}\)).%
\begin{figure}
\centering
\pushValignCaptionBottom[b]{minipage}{.23\textwidth}{%
\pgfplotsset{every axis/.append style={width=\linewidth}}%
\centering% horizontal alignment 
{
\begin{tikzpicture}
\begin{axis}[xmin=-0.5,xmax=4,ymin=-4,ymax=4,xtick={1,2,...,3}, minor xtick={-0.5,0,...,3}, ytick={-4,-2,...,4}, minor ytick={-4,-3,...,4}]
    \addplot+[
        domain=-0.5:4,
        -,
    ]{(x^6-18*x^4+48*x^2)/64*3*exp(-(x/2)^2)};
\end{axis}
\end{tikzpicture}
}
}% end body 
{\captionof{figure}{\label{figure-derivatives-one}}
}% caption 
\pushValignCaptionBottom[b]{minipage}{.23\textwidth}{%
\pgfplotsset{every axis/.append style={width=\linewidth}}%
\centering% horizontal alignment 
{
\begin{tikzpicture}
\begin{axis}[xmin=-0.5,xmax=4,ymin=-4,ymax=4,xtick={1,2,...,3}, minor xtick={-0.5,0,...,3}, ytick={-4,-2,...,4}, minor ytick={-4,-3,...,4}]
    \addplot+[
        domain=-0.5:4,
        -,
    ]{(x/2)^4*3*exp(-(x/2)^2)-1};
\end{axis}
\end{tikzpicture}
}
}% end body 
{\captionof{figure}{\label{figure-derivatives-two}}
}% caption 
\pushValignCaptionBottom[b]{minipage}{.23\textwidth}{%
\pgfplotsset{every axis/.append style={width=\linewidth}}%
\centering% horizontal alignment 
{
\begin{tikzpicture}
\begin{axis}[xmin=-0.5,xmax=4,ymin=-4,ymax=4,xtick={1,2,...,3}, minor xtick={-0.5,0,...,3}, ytick={-4,-2,...,4}, minor ytick={-4,-3,...,4}]
    \addplot+[samples=50,
        domain=-0.5:4,
        -,
    ]{(-x^7+30*x^5-192*x^3+192*x)/128*3*exp(-(x/2)^2)};
\end{axis}
\end{tikzpicture}
}
}% end body 
{\captionof{figure}{\label{figure-derivatives-three}}
}% caption 
\pushValignCaptionBottom[b]{minipage}{.23\textwidth}{%
\pgfplotsset{every axis/.append style={width=\linewidth}}%
\centering% horizontal alignment 
{
\begin{tikzpicture}
\begin{axis}[xmin=-0.5,xmax=4,ymin=-4,ymax=4,xtick={1,2,...,3}, minor xtick={-0.5,0,...,3}, ytick={-4,-2,...,4}, minor ytick={-4,-3,...,4}]
    \addplot+[
        domain=-0.5:4,
        -,
    ]{(-x^5+8*x^3)/32*3*exp(-(x/2)^2)};
\end{axis}
\end{tikzpicture}
}
}% end body 
{\captionof{figure}{\label{figure-derivatives-four}}
}% caption 
\popValignCaptionBottom
\end{figure}
\par\smallskip
\begin{exercisegroup}%
Three containers are shown in \hyperref[figure-prism-uniform]{Figures~\ref*{figure-prism-uniform}}\textendash{}\hyperref[figure-prism-bottom-heavy]{4.4.10}. Each of the following questions are in reference to these containers.%
\begin{figure}
\centering
\pushValignCaptionBottom[b]{minipage}{.33333\textwidth}{%
\pgfplotsset{every axis/.append style={width=\linewidth}}%
\centering% horizontal alignment 
{
\begin{tikzpicture}
    \pgfmathsetmacro{\cubex}{1}
    \pgfmathsetmacro{\cubey}{1.6}
    \pgfmathsetmacro{\cubez}{0.6}
    \shade[draw,fill=blue, opacity=0.5] (0,0,-\cubez) -- ++(-\cubex,0,0) -- ++(0,-\cubey,0) -- ++(\cubex,0,0) -- cycle;
    \shade[draw,fill=blue, opacity=0.5] (-\cubex,0,0) -- ++(0,0,-\cubez) -- ++(0,-\cubey,0) -- ++(0,0,\cubez) -- cycle;
    \shade[draw,fill=blue, opacity=0.5] (0,-\cubey,0) -- ++(-\cubex,0,0) -- ++(0,0,-\cubez) -- ++(\cubex,0,0) -- cycle;
    \shade[draw,fill=blue, opacity=0.5] (0,0,0) -- ++(-\cubex,0,0) -- ++(0,-\cubey,0) -- ++(\cubex,0,0) -- cycle;
    \shade[draw,fill=blue, opacity=0.5] (0,0,0) -- ++(0,0,-\cubez) -- ++(0,-\cubey,0) -- ++(0,0,\cubez) -- cycle;
    \shade[draw,fill=blue, opacity=0.5] (0,0,0) -- ++(-\cubex,0,0) -- ++(0,0,-\cubez) -- ++(\cubex,0,0) -- cycle;
\end{tikzpicture}
}
}% end body 
{\captionof{figure}{\label{figure-prism-uniform}}
}% caption 
\pushValignCaptionBottom[b]{minipage}{.33333\textwidth}{%
\pgfplotsset{every axis/.append style={width=\linewidth}}%
\centering% horizontal alignment 
{
\begin{tikzpicture}
    \pgfmathsetmacro{\cubex}{1}
    \pgfmathsetmacro{\cubey}{1.6}
    \pgfmathsetmacro{\cubez}{0.6}
    \pgfmathsetmacro{\smallx}{1/3}
    \shade[draw,fill=blue, opacity=0.5] (-\cubex,0,0) -- ++(0,0,-\cubez) -- ++(\smallx,-\cubey,0) -- ++(0,0,\cubez) -- cycle;
    \shade[draw,fill=blue, opacity=0.5] (-\smallx,-\cubey,0) -- ++(-\smallx,0,0) -- ++(0,0,-\cubez) -- ++(\smallx,0,0) -- cycle;
    \shade[draw,fill=blue, opacity=0.5] (0,0,-\cubez) -- ++(-\cubex,0,0) -- ++(\smallx,-\cubey,0) -- ++(\smallx,0,0) -- cycle;
    \shade[draw,fill=blue, opacity=0.5] (0,0,0) -- ++(0,0,-\cubez) -- ++(-\smallx,-\cubey,0) -- ++(0,0,\cubez) -- cycle;
    \shade[draw,fill=blue, opacity=0.5] (0,0,0) -- ++(-\cubex,0,0) -- ++(0,0,-\cubez) -- ++(\cubex,0,0) -- cycle;
    \shade[draw,fill=blue, opacity=0.5] (0,0,0) -- ++(-\cubex,0,0) -- ++(\smallx,-\cubey,0) -- ++(\smallx,0,0) -- cycle;
\end{tikzpicture}
}
}% end body 
{\captionof{figure}{\label{figure-prism-top-heavy}}
}% caption 
\pushValignCaptionBottom[b]{minipage}{.33333\textwidth}{%
\pgfplotsset{every axis/.append style={width=\linewidth}}%
\centering% horizontal alignment 
{
\begin{tikzpicture}
    \pgfmathsetmacro{\cubex}{1}
    \pgfmathsetmacro{\cubey}{-1.6}
    \pgfmathsetmacro{\cubez}{0.6}
    \pgfmathsetmacro{\smallx}{1/3}
    \shade[draw,fill=blue, opacity=0.5] (0,0,0) -- ++(-\cubex,0,0) -- ++(0,0,-\cubez) -- ++(\cubex,0,0) -- cycle;
    \shade[draw,fill=blue, opacity=0.5] (0,0,-\cubez) -- ++(-\cubex,0,0) -- ++(\smallx,-\cubey,0) -- ++(\smallx,0,0) -- cycle;
    \shade[draw,fill=blue, opacity=0.5] (-\cubex,0,0) -- ++(0,0,-\cubez) -- ++(\smallx,-\cubey,0) -- ++(0,0,\cubez) -- cycle;
    \shade[draw,fill=blue, opacity=0.5] (-\smallx,-\cubey,0) -- ++(-\smallx,0,0) -- ++(0,0,-\cubez) -- ++(\smallx,0,0) -- cycle;
    \shade[draw,fill=blue, opacity=0.5] (0,0,0) -- ++(0,0,-\cubez) -- ++(-\smallx,-\cubey,0) -- ++(0,0,\cubez) -- cycle;
    \shade[draw,fill=blue, opacity=0.5] (0,0,0) -- ++(-\cubex,0,0) -- ++(\smallx,-\cubey,0) -- ++(\smallx,0,0) -- cycle;
\end{tikzpicture}
}
}% end body 
{\captionof{figure}{\label{figure-prism-bottom-heavy}}
}% caption 
\popValignCaptionBottom
\end{figure}
\begin{exercisegroupbycol}{1}%
\begin{exercisegroupitem}{3. }\phantomsection\hypertarget{exercise-180}{\null}
Suppose that water is being poured into each  of the containers at a constant rate.  Let \(h_a\), \(h_b\), and \(h_c\) be the heights (measured in \si{\centi\meter}) of the liquid in containers \hyperref[figure-prism-uniform]{4.4.8}\textendash{}\hyperref[figure-prism-bottom-heavy]{4.4.10} respectively, \(t\) seconds after the water began to fill the containers.  What would you expect the sign to be on the second derivative functions \(\sd{h_a}\), \(\sd{h_b}\), and \(\sd{h_c}\) while the containers are being filled?  (Hint:  Think about the shape of the curves \(y=\fe{\sd{h_a}}{t}\), \(y=\fe{\sd{h_b}}{t}\), and \(y=\fe{\sd{h_c}}{t}\).)%
\end{exercisegroupitem}%
\par%
\begin{exercisegroupitem}{4. }\phantomsection\hypertarget{exercise-181}{\null}
Suppose that water is being drained from each  of the containers at a constant rate.  Let \(h_a\), \(h_b\), and \(h_c\) be the heights (measured in \si{\centi\meter}) of the liquid remaining in the containers \(t\) seconds after the water began to drain.  What would you expect the sign to be on the second derivative functions \(\sd{h_a}\), \(\sd{h_b}\), and \(\sd{h_c}\)  while the containers are being drained?%
\end{exercisegroupitem}%
\par%
\end{exercisegroupbycol}%
\end{exercisegroup}%
\item[5.]\phantomsection\hypertarget{exercise-182}{\null}During the recession of 2008\textendash{}2009, the total number of employed Americans decreased every month.  One month a talking head on the television made the observation that ``at least the second derivative was positive this month.''  Why was it a good thing that the second derivative was positive?%
\par\smallskip
\item[6.]\phantomsection\hypertarget{exercise-183}{\null}During the early 1980s the problem was inflation. Every month the average price for a gallon of milk was higher than the month before.  Was it a good thing when the second derivative of this function was positive?  Explain.%
\par\smallskip
\end{exerciselist}
\typeout{************************************************}
\typeout{Section 4.5 Antiderivatives}
\typeout{************************************************}
\section[Antiderivatives]{Antiderivatives}\label{section-antiderivatives}
The derivative continuum can be expressed backwards as well as forwards.  When you move from function to function in the reverse direction the resultant functions are called \terminology{antiderivatives} and the process is called \terminology{antidifferentiation}.  These relationships are shown in \hyperref[figure-derivative-chain]{Figures~\ref*{figure-derivative-chain}} and \hyperref[figure-antiderivative-chain]{4.5.2}.%
\begin{figure}
\centering
\[f\xrightarrow{\text{differentiate}}\fd{f}\xrightarrow{\text{differentiate}}\sd{f}\xrightarrow{\text{differentiate}}\td{f}\xrightarrow{\text{differentiate}}\cdots\]%
\caption{Differentiating\label{figure-derivative-chain}}
\end{figure}
\begin{figure}
\centering
\[\cdots\td{f}\xrightarrow{\text{antidifferentiate}}\sd{f}\xrightarrow{\text{antidifferentiate}}\fd{f}\xrightarrow{\text{antidifferentiate}}f\xrightarrow{\text{antidifferentiate}}F\]%
\caption{Antiifferentiating\label{figure-antiderivative-chain}}
\end{figure}
\par
There are (at least) two important differences between the differentiation chain and the antidifferentiation chain (besides their reversed order).%
\begin{itemize}[label=\textbullet]
\item{}When you differentiate, the resultant function is unique. When you antidifferentiate, you do not get a unique function\textemdash{}you get a family of functions; specifically, you get a set of parallel curves.\item{}We introduce a new function in the antidifferentiation chain. We say that \(F\) is an antiderivative of \(f\).  This is where we stop in that direction; we do not have a variable name for an antiderivative of \(F\).\end{itemize}
\par
Since \(F\) is considered an antiderivative of \(f\), it must be the case that \(f\) is the first derivative of \(F\). Hence we can add \(F\) to our derivative chain resulting in \hyperref[figure-longer-derivative-chain]{Figure~\ref*{figure-longer-derivative-chain}}.%
\begin{figure}
\centering
\[F\xrightarrow{\text{differentiate}}f\xrightarrow{\text{differentiate}}\fd{f}\xrightarrow{\text{differentiate}}\sd{f}\xrightarrow{\text{differentiate}}\td{f}\xrightarrow{\text{differentiate}}\cdots\]%
\caption{Differentiating\label{figure-longer-derivative-chain}}
\end{figure}
\typeout{************************************************}
\typeout{Exercises}
\typeout{************************************************}
\section*{Exercises}\label{exercises-25}

\begin{exercisegroup}%
Each of the linear functions in \hyperref[figure-parallel-lines]{Figure~\ref*{figure-parallel-lines}} have the same first derivative function.%
\begin{figure}
\centering
{
\begin{tikzpicture}
\begin{axis}[]
    \addplot[pccplot,domain=-6.8:-4.6]{2*(x+8)};
    \addplot[pccplot,domain=-6.8:-2.6]{2*(x+6)};
    \addplot[pccplot,domain=-6.8:-0.6]{2*(x+4)};
    \addplot[pccplot,domain=-5.4:1.4]{2*(x+2)};
    \addplot[pccplot,domain=-3.4:3.4]{2*(x)};
    \addplot[pccplot,domain=-1.4:5.4]{2*(x-2)};
    \addplot[pccplot,domain=0.6:6.8]{2*(x-4)};
    \addplot[pccplot,domain=2.6:6.8]{2*(x-6)};
    \addplot[pccplot,domain=4.6:6.8]{2*(x-8)};
\end{axis}
\end{tikzpicture}
}
\caption{\label{figure-parallel-lines}}
\end{figure}
\begin{exercisegroupbycol}{1}%
\begin{exercisegroupitem}{1. }\phantomsection\hypertarget{exercise-184}{\null}
Draw this common first derivative function onto \hyperref[figure-parallel-lines]{Figure~\ref*{figure-parallel-lines}} and label it \(g\).%
\end{exercisegroupitem}%
\par%
\begin{exercisegroupitem}{2. }\phantomsection\hypertarget{exercise-185}{\null}
Each of the given lines \hyperref[figure-parallel-lines]{Figure~\ref*{figure-parallel-lines}} is called what in relation to \(g\)?%
\end{exercisegroupitem}%
\par%
\end{exercisegroupbycol}%
\end{exercisegroup}%
\begin{exercisegroup}%
The function \(f\) is shown in \hyperref[figure-plot-antiderivative-one]{Figure~\ref*{figure-plot-antiderivative-one}}. Reference this function in the following questions.%
\begin{figure}
\centering
\pushValignCaptionBottom[b]{minipage}{.40\textwidth}{%
\pgfplotsset{every axis/.append style={width=\linewidth}}%
\centering% horizontal alignment 
{
\begin{tikzpicture}
\begin{axis}[]
    \addplot[pccplot,domain=-6.8:-3,<-]{2};
    \addplot[pccplot,domain=-3:-1,-]{-2*(x+2)};
    \addplot[pccplot,domain=-1:6.8,->]{1};
    \addplot[holdot]coordinates {(-1,-2) (-1,1)};
\end{axis}
\end{tikzpicture}
}
}% end body 
{\captionof{figure}{\(y=\fe{f}{x}\)\label{figure-plot-antiderivative-one}}
}% caption 
\pushValignCaptionBottom[b]{minipage}{.40\textwidth}{%
\pgfplotsset{every axis/.append style={width=\linewidth}}%
\centering% horizontal alignment 
{
\begin{tikzpicture}
\begin{axis}[]
\end{axis}
\end{tikzpicture}
}
}% end body 
{\captionof{figure}{\label{figure-plot-antiderivative-two}}
}% caption 
\popValignCaptionBottom
\end{figure}
\begin{figure}
\centering
\pushValignCaptionBottom[b]{minipage}{.40\textwidth}{%
\pgfplotsset{every axis/.append style={width=\linewidth}}%
\centering% horizontal alignment 
{
\begin{tikzpicture}
\begin{axis}[]
\end{axis}
\end{tikzpicture}
}
}% end body 
{\captionof{figure}{\label{figure-plot-antiderivative-three}}
}% caption 
\pushValignCaptionBottom[b]{minipage}{.40\textwidth}{%
\pgfplotsset{every axis/.append style={width=\linewidth}}%
\centering% horizontal alignment 
{
\begin{tikzpicture}
\begin{axis}[]
\end{axis}
\end{tikzpicture}
}
}% end body 
{\captionof{figure}{\label{figure-plot-antiderivative-four}}
}% caption 
\popValignCaptionBottom
\end{figure}
\begin{exercisegroupbycol}{1}%
\begin{exercisegroupitem}{3. }\phantomsection\hypertarget{exercise-186}{\null}
At what values of \(x\) is \(f\) nondifferentiable?%
\end{exercisegroupitem}%
\par%
\begin{exercisegroupitem}{4. }\phantomsection\hypertarget{exercise-187}{\null}
At what values of \(x\) are antiderivatives of \(f\) nondifferentiable?%
\end{exercisegroupitem}%
\par%
\begin{exercisegroupitem}{5. }\phantomsection\hypertarget{exercise-188}{\null}
Draw onto \hyperref[figure-plot-antiderivative-two]{Figure~\ref*{figure-plot-antiderivative-two}} the continuous antiderivative of \(f\) that passes through the point \(\point{3}{1}\).  Please note that every antiderivative of \(f\) increases exactly one unit over the interval \(\ointerval{-3}{-2}\).%
\end{exercisegroupitem}%
\par%
\begin{exercisegroupitem}{6. }\phantomsection\hypertarget{exercise-189}{\null}
Because \(f\) is not continuous, there are other antiderivatives of \(f\) that pass through the point \(\point{3}{1}\). Specifically, antiderivatives of \(f\) may or may not be continuous at \(1\).  Draw onto \hyperref[figure-plot-antiderivative-three]{Figures~\ref*{figure-plot-antiderivative-three}} and \hyperref[figure-plot-antiderivative-four]{4.5.10} different antiderivatives of \(f\) that pass through the point \(\point{3}{1}\).%
\end{exercisegroupitem}%
\par%
\end{exercisegroupbycol}%
\end{exercisegroup}%
\begin{exerciselist}
\item[7.]\phantomsection\hypertarget{exercise-190}{\null}The function \(y=\fe{\sin}{x}\) is an example of a periodic function.  Specifically, the function has a period of \(2\pi\) because over any interval of length \(2\pi\) the behavior of the function is exactly the same as it was the previous interval of length \(2\pi\).  A little more precisely, \(\fe{\sin}{x+2\pi}=\fe{\sin}{x}\) regardless of the value of \(x\).%
\par
Jasmine was thinking and told her lab assistant that derivatives and antiderivatives of periodic functions must also be periodic.  Jasmine's lab assistant told her that she was half right.  Which half did Jasmine have correct?  Also, draw a function that illustrates that the other half of Jasmine's statement is not correct.%
\par\smallskip
\end{exerciselist}
\begin{exercisegroup}%
Consider the function \(g\) shown in \hyperref[figure-line-segments]{Figure~\ref*{figure-line-segments}}.%
\begin{figure}
\centering
\pushValignCaptionBottom[b]{minipage}{.40\textwidth}{%
\pgfplotsset{every axis/.append style={width=\linewidth}}%
\centering% horizontal alignment 
{
\begin{tikzpicture}
\begin{axis}[]
    \addplot[pccplot,domain=-6:-3,-]{2/3*(x+6)+2};
    \addplot[pccplot,domain=-3:1,-]{-3};
    \addplot[pccplot,domain=1:4,-]{2/3*(x-1)+2};
    \addplot[pccplot,domain=4:6,-]{-3};
    \addplot[holdot] coordinates {(-6, 2) (-3, 4) (-3, -3) (1, -3) (1,2) (4,4) (4,-3) (6,-3)};
\end{axis}
\end{tikzpicture}
}
}% end body 
{\captionof{figure}{\(y=\fe{g}{x}\)\label{figure-line-segments}}
}% caption 
\pushValignCaptionBottom[b]{minipage}{.40\textwidth}{%
\pgfplotsset{every axis/.append style={width=\linewidth}}%
\centering% horizontal alignment 
{
\begin{tikzpicture}
\begin{axis}[]
\end{axis}
\end{tikzpicture}
}
}% end body 
{\captionof{figure}{\(y=\fe{G}{x}\)\label{figure-line-segments-antiderivative}}
}% caption 
\popValignCaptionBottom
\end{figure}
\begin{exercisegroupbycol}{1}%
\begin{exercisegroupitem}{8. }\phantomsection\hypertarget{exercise-191}{\null}
Let \(G\) be an antiderivative of \(g\).  Suppose that \(G\) is continuous on \(\cinterval{-6}{6}\), \(\fe{G}{6}=3\), and that the greatest value \(G\) ever achieves is \(6\).  Draw \(G\) onto \hyperref[figure-line-segments-antiderivative]{Figure~\ref*{figure-line-segments-antiderivative}}.%
\end{exercisegroupitem}%
\par%
\begin{exercisegroupitem}{9. }\phantomsection\hypertarget{exercise-192}{\null}
At what values of \(t\) is \(G\) nondifferentiable?%
\end{exercisegroupitem}%
\par%
\begin{exercisegroupitem}{10. }\phantomsection\hypertarget{exercise-193}{\null}
At what values of \(t\) is \(g\) nondifferentiable?%
\end{exercisegroupitem}%
\par%
\end{exercisegroupbycol}%
\end{exercisegroup}%
\begin{exercisegroup}%
Answer the following question in reference to a continuous function \(g\) whose first derivative is shown in \hyperref[figure-discontinuous-derivative]{Figure~\ref*{figure-discontinuous-derivative}}.  You do not need to state how you made your determination; just state the interval(s) or values of \(x\) that satisfy the stated property.%
\par
Note: the correct answer to one or more of these questions may be ``There is no way of knowing.''%
\begin{figure}
\centering
{
\begin{tikzpicture}
\begin{axis}[]
    \addplot[pccplot,domain=-6.8:-3,<-]{(x+5)^2+1};
    \addplot[pccplot,domain=-3:2,-]{-3/5*(x+3)+5};
    \addplot[pccplot,domain=2:4,-]{-3};
    \addplot[pccplot,domain=4:6.8,->]{1/3*(x-4)^3-2};
    \addplot[holdot] coordinates {(2,2) (2,-3) (4,-3) (4,-2)};
\end{axis}
\end{tikzpicture}
}
\caption{\(y=\fe{g}{x}\)\label{figure-discontinuous-derivative}}
\end{figure}
\begin{exercisegroupbycol}{1}%
\begin{exercisegroupitem}{11. }\phantomsection\hypertarget{exercise-194}{\null}
Over what intervals is \(\sd{g}\) positive and increasing?%
\end{exercisegroupitem}%
\par%
\begin{exercisegroupitem}{12. }\phantomsection\hypertarget{exercise-195}{\null}
At which values of \(x\) is \(g\) nondifferentiable?%
\end{exercisegroupitem}%
\par%
\begin{exercisegroupitem}{13. }\phantomsection\hypertarget{exercise-196}{\null}
Over which intervals is \(g\) not negative?%
\end{exercisegroupitem}%
\par%
\begin{exercisegroupitem}{14. }\phantomsection\hypertarget{exercise-197}{\null}
At which values of \(x\) is every antiderivative of \(g\) nondifferentiable?%
\end{exercisegroupitem}%
\par%
\begin{exercisegroupitem}{15. }\phantomsection\hypertarget{exercise-198}{\null}
Over which intervals is the value of \(\sd{g}\) constant?%
\end{exercisegroupitem}%
\par%
\begin{exercisegroupitem}{16. }\phantomsection\hypertarget{exercise-199}{\null}
Over which intervals is \(g\) linear?%
\end{exercisegroupitem}%
\par%
\begin{exercisegroupitem}{17. }\phantomsection\hypertarget{exercise-200}{\null}
Over which intervals are antiderivatives of \(g\) linear?%
\end{exercisegroupitem}%
\par%
\begin{exercisegroupitem}{18. }\phantomsection\hypertarget{exercise-201}{\null}
Over which intervals is \(\td{g}\) not negative?%
\end{exercisegroupitem}%
\par%
\end{exercisegroupbycol}%
\end{exercisegroup}%
\typeout{************************************************}
\typeout{Section 4.6 Graphical Features from Derivatives}
\typeout{************************************************}
\section[Graphical Features from Derivatives]{Graphical Features from Derivatives}\label{section-graphical-features-from-derivatives}
When given a function formula, we  often find the first and second derivative formulas to determine behaviors of the given function.  In a later lab we will use the first and second derivative formulas to help us graph a function given the formula for the function.  One thing we do with the derivative formulas is determine where they are positive, negative, zero, and undefined. This helps us determine where the given function is increasing, decreasing, concave up, concave down, and linear.%
\typeout{************************************************}
\typeout{Exercises}
\typeout{************************************************}
\section*{Exercises}\label{exercises-26}

\begin{exercisegroup}%
In the next few exercises, you will construct a function and then answer questions about it.%
\begin{figure}
\centering
\pushValignCaptionBottom[b]{minipage}{.60\textwidth}{%
\pgfplotsset{every axis/.append style={width=\linewidth}}%
\centering% horizontal alignment 
\begin{tabular}{ccc}\hrulethick
Interval&\(\fd{f}\)&\(\sd{f}\)\\\hrulemedium
\(\ointerval{-\infty}{-1}\)&Positive&Negative\\
\(\ointerval{-1}{\infty}\)&Positive&Positive
\end{tabular}
}% end body 
{\captionof{table}{Signs on \(\fd{f}\) and \(\sd{f}\)\label{table-based-on-derivatives}}
}% caption 
\pushValignCaptionBottom[b]{minipage}{.40\textwidth}{%
\pgfplotsset{every axis/.append style={width=\linewidth}}%
\centering% horizontal alignment 
{
\begin{tikzpicture}
\begin{axis}[]
\end{axis}
\end{tikzpicture}
}
}% end body 
{\captionof{figure}{\(y=\fe{f}{x}\)\label{figure-based-on-derivatives}}
}% caption 
\popValignCaptionBottom
\end{figure}
\begin{exercisegroupbycol}{1}%
\begin{exercisegroupitem}{1. }\phantomsection\hypertarget{exercise-based-on-derivatives}{\null}
Draw onto \hyperref[figure-based-on-derivatives]{Figure~\ref*{figure-based-on-derivatives}} a \emph{continuous} function \(f\) that has a horizontal tangent line at the point \(\point{1}{2}\) along with the properties stated in \hyperref[table-based-on-derivatives]{Table~\ref*{table-based-on-derivatives}}.%
\end{exercisegroupitem}%
\par%
\begin{exercisegroupitem}{2. }\phantomsection\hypertarget{exercise-203}{\null}
Given the conditions stated in \hyperref[exercise-based-on-derivatives]{Exercise~1}, does it have to be the case that \(\fe{\sd{f}}{-1}=0\)?%
\end{exercisegroupitem}%
\par%
\begin{exercisegroupitem}{3. }\phantomsection\hypertarget{exercise-204}{\null}
Is \(f\) increasing at \(-1\)?  How do you know?%
\end{exercisegroupitem}%
\par%
\begin{exercisegroupitem}{4. }\phantomsection\hypertarget{exercise-205}{\null}
Can a continuous, everywhere differentiable function satisfy the properties stated in \hyperref[table-based-on-derivatives]{Table~\ref*{table-based-on-derivatives}} and not have a slope of zero at \(1\)?  Draw a picture that supports your answer.%
\end{exercisegroupitem}%
\par%
\end{exercisegroupbycol}%
\end{exercisegroup}%
\begin{exercisegroup}%
In the next few exercises, you will construct a function and then answer questions about it.%
\begin{figure}
\centering
\pushValignCaptionBottom[b]{minipage}{.60\textwidth}{%
\pgfplotsset{every axis/.append style={width=\linewidth}}%
\centering% horizontal alignment 
\begin{tabular}{ccc}\hrulethick
Interval&\(\fd{g}\)&\(\sd{g}\)\\\hrulemedium
\(\ointerval{-\infty}{-1}\)&Positive&Positive\\
\(\ointerval{-1}{\infty}\)&Positive&Negative
\end{tabular}
}% end body 
{\captionof{table}{Signs on \(\fd{g}\) and \(\sd{g}\)\label{table-based-on-derivatives-two}}
}% caption 
\pushValignCaptionBottom[b]{minipage}{.40\textwidth}{%
\pgfplotsset{every axis/.append style={width=\linewidth}}%
\centering% horizontal alignment 
{
\begin{tikzpicture}
\begin{axis}[]
\end{axis} 
\end{tikzpicture}
}
}% end body 
{\captionof{figure}{\(y=\fe{g}{x}\)\label{figure-based-on-derivatives-two}}
}% caption 
\popValignCaptionBottom
\end{figure}
\begin{exercisegroupbycol}{1}%
\begin{exercisegroupitem}{5. }\phantomsection\hypertarget{exercise-based-on-derivatives-two}{\null}
Draw onto \hyperref[figure-based-on-derivatives-two]{Figure~\ref*{figure-based-on-derivatives-two}} a continuous function \(g\) that has a vertical tangent line at the point \(\point{-1}{2}\) along with the properties stated in \hyperref[table-based-on-derivatives-two]{Table~\ref*{table-based-on-derivatives-two}}.%
\end{exercisegroupitem}%
\par%
\begin{exercisegroupitem}{6. }\phantomsection\hypertarget{exercise-207}{\null}
Given the conditions stated in \hyperref[exercise-based-on-derivatives-two]{Exercise~5}, does it have to be the case that isundefined?%
\end{exercisegroupitem}%
\par%
\begin{exercisegroupitem}{7. }\phantomsection\hypertarget{exercise-208}{\null}
Is \(g\) increasing at \(-1\)?  How do you know?%
\end{exercisegroupitem}%
\par%
\begin{exercisegroupitem}{8. }\phantomsection\hypertarget{exercise-209}{\null}
Can a continuous, everywhere differentiable function satisfy the properties stated in \hyperref[table-based-on-derivatives-two]{Table~\ref*{table-based-on-derivatives-two}} and not have a vertical tangent line at \(1\)?  Draw a picture that supports your answer.%
\end{exercisegroupitem}%
\par%
\end{exercisegroupbycol}%
\end{exercisegroup}%
\begin{exercisegroup}%
In the next few exercises, you will construct a function and then answer questions about it.%
\begin{figure}
\centering
\pushValignCaptionBottom[b]{minipage}{.60\textwidth}{%
\pgfplotsset{every axis/.append style={width=\linewidth}}%
\centering% horizontal alignment 
\begin{tabular}{ccc}\hrulethick
Interval&\(\fd{k}\)&\(\sd{k}\)\\\hrulemedium
\(\ointerval{-\infty}{-1}\)&Negative&Positive\\
\(\ointerval{-1}{\infty}\)&Positive&Negative
\end{tabular}
}% end body 
{\captionof{table}{Signs on \(\fd{k}\) and \(\sd{k}\)\label{table-based-on-derivatives-three}}
}% caption 
\pushValignCaptionBottom[b]{minipage}{.40\textwidth}{%
\pgfplotsset{every axis/.append style={width=\linewidth}}%
\centering% horizontal alignment 
{
\begin{tikzpicture}
\begin{axis}[]
\end{axis} 
\end{tikzpicture}
}
}% end body 
{\captionof{figure}{\(y=\fe{k}{x}\)\label{figure-based-on-derivatives-three}}
}% caption 
\popValignCaptionBottom
\end{figure}
\begin{exercisegroupbycol}{1}%
\begin{exercisegroupitem}{9. }\phantomsection\hypertarget{exercise-based-on-derivatives-three}{\null}
Draw onto \hyperref[figure-based-on-derivatives-three]{Figure~\ref*{figure-based-on-derivatives-three}} a continuous function \(k\) that passes through the point \(\point{-1}{2}\) and also satisfies the properties stated in \hyperref[table-based-on-derivatives-three]{Table~\ref*{table-based-on-derivatives-three}}.%
\end{exercisegroupitem}%
\par%
\begin{exercisegroupitem}{10. }\phantomsection\hypertarget{exercise-211}{\null}
At what values of \(x\) is \(k\) nondifferentiable?%
\end{exercisegroupitem}%
\par%
\end{exercisegroupbycol}%
\end{exercisegroup}%
\begin{exercisegroup}%
You should know by now that the first derivative  continually increases over intervals where the second derivative is constantly positive and that the first derivative continually decreases over intervals where the second derivative is constantly negative.  A person might infer from this that a function changes more and more quickly over intervals where the second derivative is constantly positive and that a function changes more and more slowly over intervals where the second derivative is constantly negative.   We are going to explore that idea in this problem.%
\begin{exercisegroupbycol}{1}%
\begin{exercisegroupitem}{11. }\phantomsection\hypertarget{exercise-ice-cube}{\null}
Suppose that \(\fe{V}{t}\) is the volume of water in an ice cube (\si{\milli\liter}) where \(t\) is the amount of time that has passed since noon (measure in minutes).  Suppose that \(\fe{\fd{V}}{6}=0\,\frac{\text{ml}}{\text{min}}\) and that \(\fe{\sd{V}}{t}\) has a constant value of \(-0.3\,\frac{\sfrac{\text{ml}}{\text{min}}}{\text{min}}\) over the interval \(\cinterval{6}{11}\).  What is the value of \(\fe{\fd{V}}{11}\)?  When is \(V\) changing more quickly, at 12:06 pm or at 12:11 pm?%
\end{exercisegroupitem}%
\par%
\begin{exercisegroupitem}{12. }\phantomsection\hypertarget{exercise-213}{\null}
Referring to the function \(V\) in \hyperref[exercise-ice-cube]{Exercise~11}, what would the shape of \(V\) be over the interval \(\cinterval{6}{11}\)?  (Choose from options \hyperref[figure-curve-one]{(a)}\textendash{}\hyperref[figure-curve-four]{(d)}.)%
\begin{figure}
\centering
\pushValignCaptionBottom[b]{subfigure}{.12\textwidth}{%
\pgfplotsset{every axis/.append style={width=\linewidth}}%
\centering% horizontal alignment 
{
\begin{tikzpicture}
\begin{axis}[axis x line=none, axis y line=none,xmin=0,xmax=1,ymin=,ymax=]
    \addplot[domain=0:1,-]{sqrt((1-((x-1)/1.1)^2))};
\end{axis}
\end{tikzpicture}
}
}% end body 
{\caption{\label{figure-curve-one}}
}% caption 
\pushValignCaptionBottom[b]{subfigure}{.12\textwidth}{%
\pgfplotsset{every axis/.append style={width=\linewidth}}%
\centering% horizontal alignment 
{
\begin{tikzpicture}
\begin{axis}[axis x line=none, axis y line=none,xmin=0,xmax=1,ymin=,ymax=]
    \addplot[domain=0:1,-]{(x/1.1)^2};
\end{axis}
\end{tikzpicture}
}
}% end body 
{\caption{\label{figure-curve-two}}
}% caption 
\pushValignCaptionBottom[b]{subfigure}{.12\textwidth}{%
\pgfplotsset{every axis/.append style={width=\linewidth}}%
\centering% horizontal alignment 
{
\begin{tikzpicture}
\begin{axis}[axis x line=none, axis y line=none,xmin=0,xmax=1,ymin=,ymax=]
    \addplot[domain=0:1,-]{((x-1)/1.1)^2};
\end{axis}
\end{tikzpicture}
}
}% end body 
{\caption{\label{figure-curve-three}}
}% caption 
\pushValignCaptionBottom[b]{subfigure}{.12\textwidth}{%
\pgfplotsset{every axis/.append style={width=\linewidth}}%
\centering% horizontal alignment 
{
\begin{tikzpicture}
\begin{axis}[axis x line=none, axis y line=none,xmin=0,xmax=1,ymin=,ymax=]
    \addplot[domain=0:1,-]{1-(x/1.1)^2};
\end{axis}
\end{tikzpicture}
}
}% end body 
{\caption{\label{figure-curve-four}}
}% caption 
\popValignCaptionBottom
\caption{\label{sidebyside-curves}}
\end{figure}
\end{exercisegroupitem}%
\par%
\begin{exercisegroupitem}{13. }\phantomsection\hypertarget{exercise-214}{\null}
Again referring to options \hyperref[figure-curve-one]{(a)}\textendash{}\hyperref[figure-curve-four]{(d)}, which functions are changing more and more rapidly from left to right and which functions are changing more and more slowly from left to right? Which functions have positive second derivatives and which functions have negative second derivatives?  Do the functions with positive second derivative values both change more and more quickly from left to right?%
\end{exercisegroupitem}%
\par%
\begin{exercisegroupitem}{14. }\phantomsection\hypertarget{exercise-215}{\null}
Consider the signs on both the first and second derivatives in options \hyperref[figure-curve-one]{(a)}\textendash{}\hyperref[figure-curve-four]{(d)}.  Is there something that the two functions that change more and more quickly have in common that is different in the functions that change more and more slowly?%
\end{exercisegroupitem}%
\par%
\end{exercisegroupbycol}%
\end{exercisegroup}%
\begin{exercisegroup}%
Resolve each of the following disputes.%
\begin{exercisegroupbycol}{1}%
\begin{exercisegroupitem}{15. }\phantomsection\hypertarget{exercise-216}{\null}
One day Sara and Jermaine were working on an assignment.  One question asked them to draw a function over the domain \(\ointerval{-2}{\infty}\) with the properties that the function is always increasing and always concave down.  Sara insisted that the curve must have a vertical asymptote at \(-2\) and Jermaine insisted that the function must have a horizontal asymptote somewhere.  Were either of these students correct?%
\end{exercisegroupitem}%
\par%
\begin{exercisegroupitem}{16. }\phantomsection\hypertarget{exercise-217}{\null}
The next question Sara and Jermaine encountered described the same function with the added condition that the function is never positive.  Sara and Jermaine made the same contentions about asymptotes.  Is one of them now correct?%
\end{exercisegroupitem}%
\par%
\begin{exercisegroupitem}{17. }\phantomsection\hypertarget{exercise-218}{\null}
At another table Pedro and Yoshi were asked to draw a continuous curve that, among other properties, was never concave up.  Pedro said ``OK, so the curve is always concave down'' to which Yoshi replied ``Pedro, you need to open your  mind to other possibilities.''  Who's right?%
\end{exercisegroupitem}%
\par%
\begin{exercisegroupitem}{18. }\phantomsection\hypertarget{exercise-219}{\null}
In the next problem Pedro and Yoshi were asked to draw a function that is everywhere continuous and that is concave down at every value of \(x\) \emph{except} \(3\).  Yoshi declared ``impossible'' and Pedro responded ``have some faith, Yosh-man.''  Pedro then began to draw. Is it possible that Pedro came up with such a function?%
\end{exercisegroupitem}%
\par%
\end{exercisegroupbycol}%
\end{exercisegroup}%
\begin{exercisegroup}%
Determine the correct answer to each of the following questions.  Pictures of the situation may help you determine the correct answers.%
\begin{exercisegroupbycol}{1}%
\begin{exercisegroupitem}{19. }\phantomsection\hypertarget{exercise-220}{\null}
Which of the following propositions is true? If a given proposition is not true, draw a graph that illustrates its untruth.%
\begin{enumerate}[label=(\alph*)]
\item{}If the graph of \(f\) has a vertical asymptote, then the graph of \(\fd{f}\) must also have a vertical asymptote.\item{}If the graph of \(\fd{f}\) has a vertical asymptote, then the graph of \(f\) must also have a vertical asymptote.\end{enumerate}
\end{exercisegroupitem}%
\par%
\begin{exercisegroupitem}{20. }\phantomsection\hypertarget{exercise-221}{\null}
Suppose that the function \(f\) is everywhere continuous and concave down.  Suppose further that \(\fe{f}{7}=5\) and \(\fe{\fd{f}}{7}=3\).  Which of the following is true?%
\begin{enumerate}[label=(\alph*)]
\item{}\(\fe{f}{9}\lt11\)\item{}\(\fe{f}{9}=11\)\item{}\(\fe{f}{9}\gt11\)\item{}There is not enough information to determine the relationship between \(\fe{f}{9}\) and \(11\).\end{enumerate}
\end{exercisegroupitem}%
\par%
\end{exercisegroupbycol}%
\end{exercisegroup}%
\typeout{************************************************}
\typeout{Chapter 5 Derivative Formulas}
\typeout{************************************************}
\chapter[Derivative Formulas]{Derivative Formulas}\label{chapter-derivative-formulas}
\typeout{************************************************}
\typeout{Section 5.1 Leibniz Notation}
\typeout{************************************************}
\section[Leibniz Notation]{Leibniz Notation}\label{section-leibniz-notation}
While the primary focus of this lab is to help you develop shortcut skills for finding derivative formulas, there are inevitable notational issues that must be addressed.  It turns out that the latter issue is the one we are going to address first.%
\begin{definition}[Leibniz Notation]\label{definition-9}
If \(y=\fe{f}{x}\), we say that the derivative of \(y\) with respect to \(x\) is equal to \(\fe{\fd{f}}{x}\). Symbolically, we write: \(\lz{y}{x}=\fe{\fd{f}}{x}\).%
\par
While the symbol \(\lz{y}{x}\) certainly looks like a fraction, it is \emph{not} a fraction.  The symbol is \terminology{Leibniz notation} for the first derivative of \(y\) with respect to \(x\).  The short way of reading the symbol aloud is ``d y d x''%
.

            \par
If \(z=\fe{g}{t}\), we say that the the derivative of \(z\) with respect to \(t\) is equal to \(\fe{\fd{g}}{t}\).  Symbolically we write: \(\lz{z}{t}=\fe{\fd{g}}{t}\). (Read aloud as ``d z d t equals g prime of t''.)%
\end{definition}
\typeout{************************************************}
\typeout{Exercises}
\typeout{************************************************}
\section*{Exercises}\label{exercises-27}

\begin{exercisegroup}%
Take the derivative of both sides of each equation with respect to the independent variable as indicated in the function notation.  Write and say the derivative using Leibniz notation on the left side of the equal sign and function notation on the right side of the equal sign.  Make sure that every one in your group says at least one of the derivative equations aloud using both the formal reading and informal reading of the Leibniz notation.%
\begin{exercisegroupbycol}{3}%
\begin{exercisegroupitem}{1. }\phantomsection\hypertarget{exercise-222}{\null}
\(y=\fe{k}{t}\)%
\end{exercisegroupitem}%
\par%
\begin{exercisegroupitem}{2. }\phantomsection\hypertarget{exercise-223}{\null}
\(V=\fe{f}{r}\)%
\end{exercisegroupitem}%
\par%
\begin{exercisegroupitem}{3. }\phantomsection\hypertarget{exercise-224}{\null}
\(T=\fe{g}{P}\)%
\end{exercisegroupitem}%
\par%
\end{exercisegroupbycol}%
\end{exercisegroup}%
\typeout{************************************************}
\typeout{Section 5.2 The Derivative Operator}
\typeout{************************************************}
\section[The Derivative Operator]{The Derivative Operator}\label{section-derivative-operator}
\(\lz{y}{x}\) is the name of a derivative in the same way that \(\fe{\fd{f}}{x}\) is the name of a derivative.  We need a different symbol that tells us to \emph{take} the derivative of a given expression (in the same way that we have symbols that tell us to take a square root, sine, or  logarithm of an expression).%
\par
The symbol \(\lzo{x}\) is used to tell us to \emph{take the derivative with respect to \(x\) of something}. \emph{The symbol itself is an incomplete phrase} in the same way that the symbol \(\sqrt{\phantom{x}}\) is an incomplete phrase; in both cases we need to indicate the object to be manipulated\textemdash{}what number or formula are we taking the square root of? What number or formula  are we differentiating?%
\par
One thing you can do to help you remember the difference between the symbols \(\lz{y}{x}\) and \(\lzo{x}\) is to get in the habit of always writing grouping symbols after \(\lzo{x}\).  In this way the symbols \(\lzoo{x}{\fe{\sin}{x}}\) mean ``the derivative with respect to \(x\) of the sine of \(x\)''.  Similarly, the symbols \(\lzoo{t}{t^2}\) mean ``the derivative with respect to \(t\) of \(t\)-squared''.%
\typeout{************************************************}
\typeout{Exercises}
\typeout{************************************************}
\section*{Exercises}\label{exercises-28}

\begin{exercisegroup}%
Write the Leibniz notation for each of the following expressions.%
\begin{exercisegroupbycol}{1}%
\begin{exercisegroupitem}{1. }\phantomsection\hypertarget{exercise-225}{\null}
The derivative with respect to \(\beta\) of \(\fe{\cos}{\beta}\).%
\end{exercisegroupitem}%
\par%
\begin{exercisegroupitem}{2. }\phantomsection\hypertarget{exercise-226}{\null}
The derivative with respect to \(x\) of \(\lz{y}{x}\).%
\end{exercisegroupitem}%
\par%
\begin{exercisegroupitem}{3. }\phantomsection\hypertarget{exercise-227}{\null}
The derivative with respect to \(t\) of \(\fe{\ln}{t}\).  (Yes, we will do such things.)%
\end{exercisegroupitem}%
\par%
\begin{exercisegroupitem}{4. }\phantomsection\hypertarget{exercise-228}{\null}
The derivative of \(z\) with respect to \(x\).%
\end{exercisegroupitem}%
\par%
\begin{exercisegroupitem}{5. }\phantomsection\hypertarget{exercise-229}{\null}
The derivative with respect to \(t\) of \(\fe{g}{8}\).  (Yes, we will also do such things.)%
\end{exercisegroupitem}%
\par%
\end{exercisegroupbycol}%
\end{exercisegroup}%
\typeout{************************************************}
\typeout{Section 5.3 The Power Rule}
\typeout{************************************************}
\section[The Power Rule]{The Power Rule}\label{section-power-rule}
The first differentiation rule we are going to explore is called \terminology{the power rule of differentiation}. \begin{equation}\lzoo{x}{x^n}=n\,x^{n-1}\quad\text{for }n\neq0\label{men-2}\end{equation} When \(n\) is a positive integer, it is fairly easy to establish this rule using \hyperref[definition-first-derivative]{Definition~\ref*{definition-first-derivative}}.  The proof of the rule gets a little more complicated when \(n\) is negative, fractional, or irrational.  For purposes of this lab, we are going to just accept the rule as valid.%
\par
This rule is one you just ``do in your head'' and then write down the result.  Three examples of what you would be expected to write when differentiating power functions are shown in \hyperref[table-power-rule]{Table~\ref*{table-power-rule}}".%
\begin{table}
\centering
\caption{Examples of the Power Rule\label{table-power-rule}}
\begin{tabular}{ccc}\hrulethick
Given Function&You should ``see''&You should write\\\hrulemedium
\(y=x^7\)&&\(\lz{y}{x}=7x^6\)\\
\(\fe{f}{t}=\sqrt[3]{t^7}\)&\(\fe{f}{t}=t^{\sfrac{7}{3}}\)&\(\fe{\fd{f}}{t}=\frac{7}{3}t^{\sfrac{4}{3}}\)\\
\(z=\frac{1}{y^5}\)&\(z=y^{-5}\)&\(\lz{z}{y}=-5y^{-6}\)
\end{tabular}
\end{table}
\par
Notice that the type of notation used when naming the derivative is dictated by the manner in which the original function is expressed.  For example, \(y=x^7\) is telling us the relationship between two variables; in this situation we name the derivative using the notation \(\lz{y}{x}\).  On the other hand, function notation is being used to name the rule in \(\fe{f}{t}=\sqrt[3]{t^7}\); in this situation we name the derivative using the function notation \(\fe{\fd{f}}{t}\).%
\typeout{************************************************}
\typeout{Exercises}
\typeout{************************************************}
\section*{Exercises}\label{exercises-29}

\begin{exercisegroup}%
Find the first derivative formula for each of the following functions.  In each case take the derivative with respect to the independent variable as implied by the expression on the right side of the equal sign.  Make sure that you use the appropriate name for each derivative.%
\begin{exercisegroupbycol}{4}%
\begin{exercisegroupitem}{1. }\phantomsection\hypertarget{exercise-230}{\null}
\(\fe{f}{x}=x^{43}\)%
\end{exercisegroupitem}%
\par%
\begin{exercisegroupitem}{2. }\phantomsection\hypertarget{exercise-231}{\null}
\(z=\frac{1}{t^{7}}\)%
\end{exercisegroupitem}%
\par%
\begin{exercisegroupitem}{3. }\phantomsection\hypertarget{exercise-232}{\null}
\(P=\sqrt[5]{t^2}\)%
\end{exercisegroupitem}%
\par%
\begin{exercisegroupitem}{4. }\phantomsection\hypertarget{exercise-233}{\null}
\(\fe{h}{x}=\frac{1}{\sqrt{x}}\)%
\end{exercisegroupitem}%
\par%
\end{exercisegroupbycol}%
\end{exercisegroup}%
\begin{exerciselist}
\item[5.]\phantomsection\hypertarget{exercise-234}{\null}(This is just a prototype test of an embedded WeBWorK problem. It still has issues.)\space\space{}\par\smallskip
\item[6.]\phantomsection\hypertarget{exercise-235}{\null}(This is just a prototype of an embedded GeoGebra HTML5 worksheet.)\space\space{}Move the green ``x'' on the left \(x\)-axis back and forth (keeping \(c=2\)), and explain how this reveals that \(\lzoo{x}{x^2}=2x\). Then explore the applet with other values of \(c\).%
\par\smallskip
\end{exerciselist}
\typeout{************************************************}
\typeout{Section 5.4 The Constant Factor Rule}
\typeout{************************************************}
\section[The Constant Factor Rule]{The Constant Factor Rule}\label{section-constant-factor-rule}
The next rule you are going to practice is the \terminology{constant factor rule} of differentiation.  This is another rule you do in your head. \begin{equation}\lzoo{x}{k\ \fe{f}{x}}=k\cdot\lzoo{x}{\fe{f}{x}}\quad\text{for }k\in\reals\text{.}\label{men-3}\end{equation}%
\par
In the following examples and problems several derivative rules are used that are shown in {$\langle\langle$Appendix C (pages C5 and C6)$\rangle\rangle$}.  While working this lab you should refer to those rules; you may want to cover up the chain rule and implicit derivative columns this week!  You should make a goal of having all of the basic formulas memorized within a week.%
\begin{table}
\centering
\caption{Examples of the Constant Factor Rule\label{table-constant-factor-rule}}
\begin{tabular}{cc}\hrulethick
Given Function&You should write\\\hrulemedium
\(\fe{f}{x}=6x^9\)&\(\fe{\fd{f}}{x}=54x^8\)\\
\(\fe{f}{\theta}=2\fe{\cos}{\theta}\)&\(\fe{\fd{f}}{\theta}=-2\fe{\sin}{\theta}\)\\
\(z=2\fe{\ln}{x}\)&\(\lz{z}{x}=\frac{2}{x}\)
\end{tabular}
\end{table}
\typeout{************************************************}
\typeout{Exercises}
\typeout{************************************************}
\section*{Exercises}\label{exercises-30}

\begin{exercisegroup}%
Find the first derivative formula for each of the following functions.  In each case take the derivative with respect to the independent variable as implied by the expression on the right side of the equal sign.  Make sure that you use the appropriate name for each derivative.%
\begin{exercisegroupbycol}{3}%
\begin{exercisegroupitem}{1. }\phantomsection\hypertarget{exercise-236}{\null}
\(z=7t^4\)%
\end{exercisegroupitem}%
\par%
\begin{exercisegroupitem}{2. }\phantomsection\hypertarget{exercise-237}{\null}
\(\fe{P}{x}=-7\fe{\sin}{x}\)%
\end{exercisegroupitem}%
\par%
\begin{exercisegroupitem}{3. }\phantomsection\hypertarget{exercise-238}{\null}
\(\fe{h}{t}=\frac{1}{3}\fe{\ln}{t}\)%
\end{exercisegroupitem}%
\par%
\begin{exercisegroupitem}{4. }\phantomsection\hypertarget{exercise-239}{\null}
\(\fe{z}{x}=\pi\fe{\tan}{x}\)%
\end{exercisegroupitem}%
\par%
\begin{exercisegroupitem}{5. }\phantomsection\hypertarget{exercise-240}{\null}
\(P=\frac{-8}{t^4}\)%
\end{exercisegroupitem}%
\par%
\begin{exercisegroupitem}{6. }\phantomsection\hypertarget{exercise-241}{\null}
\(T=4\sqrt{t}\)%
\end{exercisegroupitem}%
\par%
\end{exercisegroupbycol}%
\end{exercisegroup}%
\typeout{************************************************}
\typeout{Section 5.5 The Constant Divisor Rule}
\typeout{************************************************}
\section[The Constant Divisor Rule]{The Constant Divisor Rule}\label{section-constant-divisor-rule}
When an expression is divided by the constant \(k\), we can think of the expression as being multiplied by the fraction \(\frac{1}{k}\).  In this way, the constant factor rule of differentiation can be applied when a formula is multiplied \emph{or divided} by a constant.%
\begin{table}
\centering
\caption{Examples of the Constant Divisor Rule\label{table-constant-divisor-rule}}
\begin{tabular}{ccc}\hrulethick
Given Function&You should ``see''&You should write\\\hrulemedium
\(\fe{f}{x}=\frac{x^4}{8}\)&\(\fe{f}{x}=\frac{1}{8}x^4\)&\(\fe{\fd{f}}{x}=\frac{1}{2}x^3\)\\
\(y=\frac{3\fe{\tan}{\alpha}}{2}\)&\(y=\frac{3}{2}\fe{\tan}{\alpha}\)&\(\lz{y}{\alpha}=\frac{3}{2}\fe{\sec^2}{\alpha}\)\\
\(z=\frac{\fe{\ln}{y}}{3}\)&\(z=\frac{1}{3}\fe{\ln}{y}\)&\(\lz{z}{y}=\frac{1}{3y}\)
\end{tabular}
\end{table}
\typeout{************************************************}
\typeout{Exercises}
\typeout{************************************************}
\section*{Exercises}\label{exercises-31}

\begin{exercisegroup}%
Find the first derivative formula for each of the following functions.  In each case take the derivative with respect to the independent variable as implied by the expression on the right side of the equal sign.  Make sure that you use the appropriate name for each derivative. In the last problem, \(G\), \(m_1\), and \(m_2\) are constants.%
\begin{exercisegroupbycol}{3}%
\begin{exercisegroupitem}{1. }\phantomsection\hypertarget{exercise-242}{\null}
\(\fe{z}{t}=\frac{\fe{\sin^{-1}}{t}}{6}\)%
\end{exercisegroupitem}%
\par%
\begin{exercisegroupitem}{2. }\phantomsection\hypertarget{exercise-243}{\null}
\(\fe{V}{r}=\frac{\pi r^3}{3}\)%
\end{exercisegroupitem}%
\par%
\begin{exercisegroupitem}{3. }\phantomsection\hypertarget{exercise-244}{\null}
\(\fe{f}{r}=\frac{Gm_1m_3}{r^2}\)%
\end{exercisegroupitem}%
\par%
\end{exercisegroupbycol}%
\end{exercisegroup}%
\typeout{************************************************}
\typeout{Section 5.6 The Sum and Difference Rules}
\typeout{************************************************}
\section[The Sum and Difference Rules]{The Sum and Difference Rules}\label{section-sum-and-difference-rules}
When taking the derivative of two or more terms, you can take the derivatives term by term and insert plus or minus signs as appropriate.  Collectively we call this the i\terminology{sum and difference rules} of differentiation.\begin{equation}\lzoo{x}{\fe{f}{x}\pm\fe{g}{x}}=\lzoo{x}{\fe{f}{x}}\pm\lzoo{x}{\fe{g}{x}}\label{men-4}\end{equation}%
\par
In the following examples and problems we introduce linear and constant terms into the functions being differentiated.\begin{equation}\lzoo{x}{k}=0\quad\text{for }k\in\reals\label{equation-constant-rule}\end{equation}\begin{equation}\lzoo{x}{kx}=k\quad\text{for }k\in\reals\label{equation-linear-rule}\end{equation}%
\begin{table}
\centering
\caption{Examples of the Sum and Difference Rules\label{table-sum-and-difference-rule}}
\begin{tabular}{ccc}\hrulethick
Given Function&You should ``see''&You should write\\\hrulemedium
\(y=4\sqrt[5]{t^6}-\frac{1}{6\sqrt{t}}+8t\)&\(y=4t^{\sfrac{6}{5}}-\frac{1}{6}t^{-\sfrac{1}{2}}+8t\)&\(\begin{aligned}\lz{y}{t}&=\frac{24}{5}t^{\sfrac{1}{5}}+\frac{1}{12}t^{-\sfrac{3}{2}}+8\\&=\frac{24\sqrt[5]{t}}{5}+\frac{1}{12\sqrt{t^3}}+8\end{aligned}\)\\
\(\fe{P}{\gamma}=\frac{\fe{\sin}{\gamma}-\fe{\cos}{\gamma}}{2}+4\)&\(\fe{P}{\gamma}=\frac{1}{2}\fe{\sin}{\gamma}-\frac{1}{2}\fe{\cos}{\gamma}+4\)&\(\fe{\fd{P}}{\gamma}=\frac{1}{2}\fe{\cos}{\gamma}+\frac{1}{2}\fe{\sin}{\gamma}\)
\end{tabular}
\end{table}
\typeout{************************************************}
\typeout{Exercises}
\typeout{************************************************}
\section*{Exercises}\label{exercises-32}

\begin{exerciselist}
\item[1.]\phantomsection\hypertarget{exercise-245}{\null}Explain why both the constant rule (\hyperref[equation-constant-rule]{[men]
~\ref*{equation-constant-rule}}) and linear rule (\hyperref[equation-linear-rule]{[men]
~\ref*{equation-linear-rule}}) are ``obvious.'' Hint: think about the graphs \(y=k\) and \(y=kx\).  What does a first derivative tell you about a graph?%
\par\smallskip
\end{exerciselist}
\begin{exercisegroup}%
Find the first derivative formula for each of the following functions.  In each case take the derivative with respect to the independent variable as implied by the expression on the right side of the equal sign.  Make sure that you use the appropriate name for each derivative. In the last problem, \(G\), \(m_1\), and \(m_2\) are constants.%
\begin{exercisegroupbycol}{2}%
\begin{exercisegroupitem}{2. }\phantomsection\hypertarget{exercise-246}{\null}
\(T=\fe{\sin}{t}-2\fe{\cos}{t}+3\)%
\end{exercisegroupitem}%
\par%
\begin{exercisegroupitem}{3. }\phantomsection\hypertarget{exercise-247}{\null}
\(\fe{k}{\theta}=\frac{4\fe{\sec}{\theta}-3\fe{\csc}{\theta}}{4}\)%
\end{exercisegroupitem}%
\par%
\begin{exercisegroupitem}{4. }\phantomsection\hypertarget{exercise-248}{\null}
\(\fe{r}{x}=\frac{x}{5}+7\)%
\end{exercisegroupitem}%
\par%
\begin{exercisegroupitem}{5. }\phantomsection\hypertarget{exercise-249}{\null}
\(r=\frac{x}{3\sqrt[3]{x}}-\frac{\fe{\ln}{x}}{9}+\fe{\ln}{2}\)%
\end{exercisegroupitem}%
\par%
\end{exercisegroupbycol}%
\end{exercisegroup}%
\typeout{************************************************}
\typeout{Section 5.7 Product Rule}
\typeout{************************************************}
\section[Product Rule]{Product Rule}\label{section-product-rule}
The next rule we are going to explore is called the \terminology{product rule} of differentiation.  We use this rule when there are two or more variable factors in the expression we are differentiating. (Remember, we already have the constant factor rule to deal with two factors when one of the two factors is a constant.)\begin{equation}\lzoo{x}{\fe{f}{x}\fe{g}{x}}=\lzoo{x}{\fe{f}{x}}\cdot\fe{g}{x}+\fe{f}{x}\cdot\lzoo{x}{\fe{g}{x}}\label{equation-product-rule}\end{equation}%
\par
Intuitively, what is happening in this rule is that we are alternately treating one factor as a constant (the one not being differentiated) and the other factor as a variable function (the one that is being differentiated).  We then add these two rates of change together.%
\par
Ultimately, you want to perform this rule in your head just like all of the other rules.  Your instructor, however, may initially want you to show steps; under that presumption, steps are going to be shown in each and every example of this lab when the product rule is applied. Two simple examples of the product rule are shown in \hyperref[table-product-rule]{Table~\ref*{table-product-rule}}i.%
\begin{table}
\centering
\caption{Examples of the Product Rule\label{table-product-rule}}
\begin{tabular}{cc}\hrulethick
Given Function&Derivative\\\hrulemedium
\(\fe{y}{x}=x^2\fe{\sin}{x}\)&\(\begin{aligned}\fe{\fd{y}}{x}&=\lzoo{x}{x^2}\cdot\fe{\sin}{x}+x^2\cdot\lzoo{x}{\fe{\sin}{x}}\\&=2x\fe{\sin}{x}+x^2\fe{\cos}{x}\end{aligned}\)\\
\(P=e^t\fe{\cos}{t}\)&\(\begin{aligned}\lz{P}{t}&=\lzoo{t}{e^t}\cdot\fe{\cos}{t}+e^t\cdot\lzoo{t}{\fe{\cos}{t}}\\&=e^t\fe{\cos}{t}-e^t\fe{\sin}{t}\end{aligned}\)
\end{tabular}
\end{table}
\par
A decision you'll need to make is how to handle a constant factor in a term that requires the product rule.  Two options for taking the derivative of \(\fe{f}{x}=5x^2\fe{\ln}{x}\) are shown in \hyperref[figure-constant-factors-with-product-rule]{Figure~\ref*{figure-constant-factors-with-product-rule}}.%
\begin{figure}
\centering
\begin{align*}
&\textbf{Option A}&&\textbf{Option B}\\
\fe{\fd{f}}{x}&=5\left[\lzoo{x}{x^2}\cdot\fe{\ln}{x}+x^2\cdot\lzoo{x}{\fe{\ln}{x}}\right]&\fe{\fd{f}}{x}&=\lzoo{x}{5x^2}\cdot\fe{\ln}{x}+5x^2\cdot\lzoo{x}{\fe{\ln}{x}}\\
&=5\left[2x\fe{\ln}{x}+x^2\cdot\frac{1}{x}\right]&&=10x\fe{\ln}{x}+5x^2\cdot\frac{1}{x}\\
&=10x\fe{\ln}{x}+5x&&=10x\fe{\ln}{x}+5x
\end{align*}%
\caption{\label{figure-constant-factors-with-product-rule}}
\end{figure}
\par
In Option B we are treating the factor of \(5\) as a part of the first variable factor.  In doing so, the factor of \(5\) distributes itself.  This is the preferred treatment of the author, so this is what you will see illustrated in this lab.%
\typeout{************************************************}
\typeout{Exercises}
\typeout{************************************************}
\section*{Exercises}\label{exercises-33}

\begin{exercisegroup}%
Find the first derivative formula for each of the following functions.  In each case take the derivative with respect to the independent variable as implied by the expression on the right side of the equal sign.  Make sure that you use the appropriate name for each derivative. In the last problem, \(G\), \(m_1\), and \(m_2\) are constants.%
\begin{exercisegroupbycol}{2}%
\begin{exercisegroupitem}{1. }\phantomsection\hypertarget{exercise-250}{\null}
\(\fe{T}{t}=2\fe{\sec}{t}\fe{\tan}{t}\)%
\end{exercisegroupitem}%
\par%
\begin{exercisegroupitem}{2. }\phantomsection\hypertarget{exercise-251}{\null}
\(k=\frac{e^t\sqrt{t}}{2}\)%
\end{exercisegroupitem}%
\par%
\begin{exercisegroupitem}{3. }\phantomsection\hypertarget{exercise-252}{\null}
\(y=4x\fe{\ln}{x}+3^x-x^3\)%
\end{exercisegroupitem}%
\par%
\begin{exercisegroupitem}{4. }\phantomsection\hypertarget{exercise-253}{\null}
\(\fe{f}{x}=\fe{\cot}{x}\fe{\cot}{x}-1\)%
\end{exercisegroupitem}%
\par%
\end{exercisegroupbycol}%
\end{exercisegroup}%
\begin{exercisegroup}%
Find each of the following derivatives \emph{without first simplifying the formula}; that is, go ahead and use the product rule on the expression as written. Simplify each resultant derivative formula.  For each derivative, \emph{check} your answer by simplifying the original expression and then taking the derivative of that simplified expression.%
\begin{exercisegroupbycol}{3}%
\begin{exercisegroupitem}{5. }\phantomsection\hypertarget{unsimplified-product-first}{\null}
\(\lzoo{x}{x^4x^7}\)%
\end{exercisegroupitem}%
\par%
\begin{exercisegroupitem}{6. }\phantomsection\hypertarget{exercise-255}{\null}
\(\lzoo{x}{x\cdot x^{10}}\)%
\end{exercisegroupitem}%
\par%
\begin{exercisegroupitem}{7. }\phantomsection\hypertarget{unsimplified-product-last}{\null}
\(\lzoo{x}{\sqrt{x}\sqrt{x^{21}}}\)%
\end{exercisegroupitem}%
\par%
\end{exercisegroupbycol}%
\end{exercisegroup}%
\typeout{************************************************}
\typeout{Section 5.8 Quotient Rule}
\typeout{************************************************}
\section[Quotient Rule]{Quotient Rule}\label{section-quotient-rule}
The next rule we are going to explore is called the \terminology{quotient rule} of differentiation.  We never use this rule unless there is a variable factor in the denominator of the expression we are differentiating.  (Remember, we already have the constant factor/divisor rule to deal with constant factors in the denominator.)\begin{equation}\lzoo{x}{\frac{\fe{f}{x}}{\fe{g}{x}}}=\frac{\lzoo{x}{\fe{f}{x}}\cdot\fe{g}{x}-\fe{f}{x}\cdot\lzoo{x}{\fe{g}{x}}}{\left[\fe{g}{x}\right]^2}\label{equation-quotient-rule}\end{equation}%
\par
Like all of the other rules, you ultimately want to perform the quotient rule in your head.  Your instructor, however, may initially want you to show steps when applying the quotient rule; under that presumption, steps are going to be shown in each and every example of this lab when the quotient rule is applied.  Two simple examples of the quotient rule are shown in \hyperref[table-quotient-rule]{Table~\ref*{table-quotient-rule}}.%
\begin{table}
\centering
\caption{Examples of the Quotient Rule\label{table-quotient-rule}}
\begin{tabular}{cc}\hrulethick
Given Function&Derivative\\\hrulemedium
\(y=\frac{4x^3}{\fe{\ln}{x}}\)&\(\begin{aligned}\lz{y}{x}&=\frac{\lzoo{x}{4x^3}\cdot\fe{\ln}{x}-4x^3\cdot\lzoo{x}{\fe{\ln}{x}}}{\left[\fe{\ln}{x}\right]^2}\\&=\frac{12x^2\cdot\fe{\ln}{x}-4x^3\cdot\frac{1}{x}}{\left[\fe{\ln}{x}\right]^2}\\&=\frac{12x^2\fe{\ln}{x}-4x^2}{\left[\fe{\ln}{x}\right]^2}\end{aligned}\)\\
\(V=\frac{\fe{\csc}{t}}{\fe{\tan}{t}}\)&\(\begin{aligned}\lz{V}{t}&=\frac{\lzoo{t}{\fe{\csc}{t}}\cdot\fe{\tan}{t}-\fe{\csc}{t}\cdot\lzoo{t}{\fe{\tan}{x}}}{\fe{\tan^2}{t}}\\&=\frac{-\fe{\csc}{t}\fe{\cot}{t}\fe{\tan}{t}-\fe{\csc}{t}\fe{\sec^2}{t}}{\fe{\tan^2}{t}}\\&=\frac{-\fe{\csc}{t}\left(1+\fe{\sec^2}{t}\right)}{\fe{\tan^2}{t}}\end{aligned}\)
\end{tabular}
\end{table}
\typeout{************************************************}
\typeout{Exercises}
\typeout{************************************************}
\section*{Exercises}\label{exercises-34}

\begin{exercisegroup}%
Find the first derivative formula for each of the following functions.  In each case take the derivative with respect to the independent variable as implied by the expression on the right side of the equal sign.  Make sure that you use the appropriate name for each derivative. In the last problem, \(G\), \(m_1\), and \(m_2\) are constants.%
\begin{exercisegroupbycol}{2}%
\begin{exercisegroupitem}{1. }\phantomsection\hypertarget{exercise-257}{\null}
\(\fe{g}{x}=\frac{4\fe{\ln}{x}}{x}\)%
\end{exercisegroupitem}%
\par%
\begin{exercisegroupitem}{2. }\phantomsection\hypertarget{exercise-258}{\null}
\(\fe{j}{y}=\frac{\sqrt[3]{y^5}}{\fe{\cos}{y}}\)%
\end{exercisegroupitem}%
\par%
\begin{exercisegroupitem}{3. }\phantomsection\hypertarget{exercise-259}{\null}
\(y=\frac{\fe{\sin}{x}}{4\fe{\sec}{x}}\)%
\end{exercisegroupitem}%
\par%
\begin{exercisegroupitem}{4. }\phantomsection\hypertarget{exercise-260}{\null}
\(\fe{f}{t}=\frac{t^2}{e^t}\)%
\end{exercisegroupitem}%
\par%
\end{exercisegroupbycol}%
\end{exercisegroup}%
\begin{exercisegroup}%
Find each of the following derivatives \emph{without first simplifying the formula}; that is, go ahead and use the quotient rule on the expression as written. For each derivative, \emph{check} your answer by simplifying the original expression and then taking the derivative of that simplified expression.%
\begin{exercisegroupbycol}{3}%
\begin{exercisegroupitem}{5. }\phantomsection\hypertarget{unsimplified-quotient-first}{\null}
\(\lzoo{t}{\frac{\fe{\sin}{t}}{\fe{\sin}{t}}}\)%
\end{exercisegroupitem}%
\par%
\begin{exercisegroupitem}{6. }\phantomsection\hypertarget{exercise-262}{\null}
\(\lzoo{x}{\frac{x^6}{x^2}}\)%
\end{exercisegroupitem}%
\par%
\begin{exercisegroupitem}{7. }\phantomsection\hypertarget{unsimplified-quotient-last}{\null}
\(\lzoo{x}{\frac{10}{2x}}\)%
\end{exercisegroupitem}%
\par%
\end{exercisegroupbycol}%
\end{exercisegroup}%
\typeout{************************************************}
\typeout{Section 5.9 Simplification}
\typeout{************************************************}
\section[Simplification]{Simplification}\label{section-simplification}
In \hyperref[unsimplified-product-first]{Exercises~5}\textendash{}\hyperref[unsimplified-product-last]{7} from \hyperref[section-product-rule]{Section~\ref*{section-product-rule}} and \hyperref[unsimplified-quotient-first]{Exercises~5}\textendash{}\hyperref[unsimplified-quotient-last]{7} from \hyperref[section-quotient-rule]{Section~\ref*{section-quotient-rule}} you applied the product and quotient rules to expressions where the derivative could have been found much more quickly had you simplified the expression before taking the derivative.  For example, while you can find the correct derivative formula using the quotient rule when working problem \hyperref[unsimplified-quotient-last]{Exercise~7}, the derivative can be found much more quickly if you simplify the expression before applying the rules of differentiation. This is illustrated in \hyperref[example-simplify-first]{Example~\ref*{example-simplify-first}}.%
\begin{example}\label{example-simplify-first}
\begin{align*}
\lzoo{x}{\frac{10}{2x}}&=\lzoo{x}{5x^{-1}}\\
&=-5x^{-2}\\
&=-\frac{5}{x^{2}}
\end{align*}%
\end{example}
\par
Part of learning to take derivatives is learning  to make good choices about the methodology to employ when taking derivatives.  In \hyperref[example-simplify-second]{Examples~\ref*{example-simplify-second}} and \hyperref[example-simplify-third]{5.9.3}, the need to use the product rule or quotient rule is obviated by first simplifying the expression being differentiated.%
\begin{example}\label{example-simplify-second}
\begin{align*}
\text{Find }\lz{y}{x}\text{ if }y&=\fe{\sec}{x}\fe{\cos}{x}\text{.}&y&=\fe{\sec}{x}\fe{\cos}{x}\\
&&&=\frac{1}{\fe{\cos}{x}}\fe{\cos}{x}\\
&&&=1\\
&&\lz{y}{x}&=0
\end{align*}%
\end{example}
\begin{example}\label{example-simplify-third}
\begin{align*}
\text{Find }\fe{\fd{f}}{t}\text{ if }\fe{f}{t}&=\frac{4t^5-3t^3}{2t^2}\text{.}&\fe{f}{t}&=\frac{4t^5-3t^3}{2t^2}\\
&&&=\frac{4t^5}{2t^2}-\frac{3t^3}{2t^2}\\
&&&=2t^3-\frac{3}{2}t\\
&&\fe{\fd{f}}{t}&=6t^2-\frac{3}{2}
\end{align*}%
\end{example}
\typeout{************************************************}
\typeout{Exercises}
\typeout{************************************************}
\section*{Exercises}\label{exercises-35}

\begin{exercisegroup}%
Find the derivative with respect to \(x\) for each of the following functions after first completely simplifying the formula being differentiated.  In each case you should \emph{not} use either the product rule or the quotient rule while finding the derivative formula.%
\begin{exercisegroupbycol}{2}%
\begin{exercisegroupitem}{1. }\phantomsection\hypertarget{exercise-264}{\null}
\(y=\frac{4x^{12}-5x^4+3x^2}{x^4}\)%
\end{exercisegroupitem}%
\par%
\begin{exercisegroupitem}{2. }\phantomsection\hypertarget{exercise-265}{\null}
\(\fe{g}{x}=\frac{-4\fe{\sin}{x}}{\fe{\cos}{x}}\)%
\end{exercisegroupitem}%
\par%
\begin{exercisegroupitem}{3. }\phantomsection\hypertarget{exercise-266}{\null}
\(\fe{h}{x}=\frac{4-x^6}{3x^{-2}}\)%
\end{exercisegroupitem}%
\par%
\begin{exercisegroupitem}{4. }\phantomsection\hypertarget{exercise-267}{\null}
\(\fe{z}{x}=\fe{\sin^2}{x}+\fe{\cos^2}{x}\)%
\end{exercisegroupitem}%
\par%
\begin{exercisegroupitem}{5. }\phantomsection\hypertarget{exercise-268}{\null}
\(z=(x+4)(x-4)\)%
\end{exercisegroupitem}%
\par%
\begin{exercisegroupitem}{6. }\phantomsection\hypertarget{exercise-269}{\null}
\(\fe{T}{x}=\frac{\fe{\ln}{x}}{\fe{\ln}{x^2}}\)%
\end{exercisegroupitem}%
\par%
\end{exercisegroupbycol}%
\end{exercisegroup}%
\typeout{************************************************}
\typeout{Section 5.10 Product and Quotient Rules Together}
\typeout{************************************************}
\section[Product and Quotient Rules Together]{Product and Quotient Rules Together}\label{section-product-and-quotient-together}
Sometimes both the product rule and quotient rule need to be applied when finding a derivative formula.%
\typeout{************************************************}
\typeout{Exercises}
\typeout{************************************************}
\section*{Exercises}\label{exercises-36}

\begin{exercisegroup}%
Consider the functions defined by \(\fe{f}{x}=x^2\frac{\fe{\sin}{x}}{e^x}\) and \(\fe{g}{x}=\frac{x^2\fe{\sin}{x}}{e^x}\).%
\begin{exercisegroupbycol}{1}%
\begin{exercisegroupitem}{1. }\phantomsection\hypertarget{exercise-270}{\null}
Discuss why \(f\) and \(g\) are in fact two representations of the same function.%
\end{exercisegroupitem}%
\par%
\begin{exercisegroupitem}{2. }\phantomsection\hypertarget{exercise-271}{\null}
Find \(\fe{\fd{f}}{x}\) by first applying the product rule and then applying the quotient rule (where necessary).%
\end{exercisegroupitem}%
\par%
\begin{exercisegroupitem}{3. }\phantomsection\hypertarget{exercise-272}{\null}
Find \(\fe{\fd{g}}{x}\) by first applying the quotient rule and then applying the product rule (where necessary).%
\end{exercisegroupitem}%
\par%
\begin{exercisegroupitem}{4. }\phantomsection\hypertarget{exercise-273}{\null}
Rigorously establish that the formulas for \(\fe{\fd{f}}{x}\) and \(\fe{\fd{g}}{x}\) are indeed the same.%
\end{exercisegroupitem}%
\par%
\end{exercisegroupbycol}%
\end{exercisegroup}%
\typeout{************************************************}
\typeout{Section 5.11 Derivative Formulas and Function Behavior}
\typeout{************************************************}
\section[Derivative Formulas and Function Behavior]{Derivative Formulas and Function Behavior}\label{section-derivative-formulas-and-function-behavior}
Derivative formulas can give us much information about the behavior of a function.  For example, the derivative formula for \(\fe{f}{x}=x^2\) is \(\fe{\fd{f}}{x}=2x\).  Clearly \(\fd{f}\) is negative when \(x\) is negative and \(\fd{f}\) is positive when \(x\) is positive.  This tells us that \(f\) is decreasing when \(x\) is negative and that \(f\) is increasing when \(x\) is positive.  This matches the behavior of the parabola \(y=x^2\).%
\typeout{************************************************}
\typeout{Exercises}
\typeout{************************************************}
\section*{Exercises}\label{exercises-37}

\begin{exercisegroup}%
The amount of time (seconds), \(T\), required for a pendulum to complete one period is a function of the pendulum's length (meters), \(L\).  Specifically, \(T=2\pi\sqrt{\frac{L}{g}}\) where \(g\) is the gravitational acceleration constant for Earth (roughly \SI{9.8}{\meter\per\second\tothe{2}}).%
\begin{exercisegroupbycol}{1}%
\begin{exercisegroupitem}{1. }\phantomsection\hypertarget{exercise-274}{\null}
Find \(\lz{T}{L}\) after first rewriting the formula for \(T\) as a constant times \(\sqrt{L}\).%
\end{exercisegroupitem}%
\par%
\begin{exercisegroupitem}{2. }\phantomsection\hypertarget{exercise-275}{\null}
The sign on \(\lz{T}{L}\) is the same regardless of the value of \(L\).  What is this sign and what does it tell you about the relative periods of two pendulums with different lengths?%
\end{exercisegroupitem}%
\par%
\end{exercisegroupbycol}%
\end{exercisegroup}%
\begin{exercisegroup}%
The gravitational force (Newtons) between two objects of masses \(m_1\) and \(m_2\) (\si{\kilo\gram}) is a function of the distance (meters) between the objects' centers of mass, \(r\).    Specifically, \(\fe{F}{r}=\frac{Gm_1m_2}{r}\) where \(G\) is the universal gravitational constant (which is approximately \(6.7\times10^{-11}\)\si{\newton\meter\tothe{2}\per\kilo\gram\tothe{2}}.)%
\begin{exercisegroupbycol}{1}%
\begin{exercisegroupitem}{3. }\phantomsection\hypertarget{exercise-276}{\null}
Leaving \(G\), \(m_1\), and \(m_2\) as constants, find \(\fe{\fd{F}}{r}\) after first rewriting the formula for \(F\) as a constant times a power of \(r\).%
\end{exercisegroupitem}%
\par%
\begin{exercisegroupitem}{4. }\phantomsection\hypertarget{exercise-277}{\null}
The sign on \(\fe{\fd{F}}{r}\) is the same regardless of the value of \(r\).  What is this sign and what does it tell you about the effect on the gravitational force between two objects when the distance between the objects is changed?%
\end{exercisegroupitem}%
\par%
\begin{exercisegroupitem}{5. }\phantomsection\hypertarget{exercise-evaluate-gravity}{\null}
Leaving \(G\), \(m_1\), and \(m_2\) as constants, find \(\fe{\fd{F}}{1.00\times10^{12}}\), \(\fe{\fd{F}}{1.01\times10^{12}}\), and \(\fe{\fd{F}}{1.02\times10^{12}}\).%
\end{exercisegroupitem}%
\par%
\begin{exercisegroupitem}{6. }\phantomsection\hypertarget{exercise-279}{\null}
Calculate \(\frac{\fe{F}{1.02\times10^{12}}-\fe{F}{1.00\times10^{12}}}{1.02\times10^{12}-1.00\times10^{12}}\). Which of the quantities found in \hyperref[exercise-evaluate-gravity]{Exercise~5} comes closest to this value?  Draw a sketch of \(F\) and discuss why this result makes sense.%
\end{exercisegroupitem}%
\par%
\end{exercisegroupbycol}%
\end{exercisegroup}%
\typeout{************************************************}
\typeout{Chapter 6 The Chain Rule}
\typeout{************************************************}
\chapter[The Chain Rule]{The Chain Rule}\label{chapter-chain-rule}
\typeout{************************************************}
\typeout{Section 6.1 Introduction to the Chain Rule}
\typeout{************************************************}
\section[Introduction to the Chain Rule]{Introduction to the Chain Rule}\label{section-introduction-to-the-chain-rule}
The functions \(\fe{f}{t}=\fe{\sin}{t}\) and \(\fe{k}{t}=\fe{\sin}{3t}\) are shown in \hyperref[figure-sine-waves]{Figure~\ref*{figure-sine-waves}}. Since \(\fe{\fd{f}}{t}=\fe{\cos}{t}\), it is reasonable to speculate that \(\fe{\fd{k}}{t}=\fe{\cos}{3t}\).  But this would imply that \(\fe{\fd{k}}{0}=\fe{\fd{f}}{0}=1\), and a quick glance of the two functions at \(0\) should convince you that this is not true; clearly \(\fe{\fd{k}}{0}>\fe{\fd{f}}{0}\).%
\begin{figure}
\centering
{
\begin{tikzpicture}
\begin{axis}[legend entries ={{$f$},{$k$}},
    xmin=-3.1415926,
    xtick={-3.1415926,-1.5707963,...,6.2831852},
    minor xtick={},
    xmax=6.2831852,
    ymin=-2,
    ymax=2,
    y post scale = 1/1.5707963,
    xticklabels={$-\pi$,$-\frac{\pi}{2}$,$0$,$\frac{\pi}{2}$,$\pi$,$\frac{3\pi}{2}$,$2\pi$}, 
    xlabel={$t$},
    domain=-3.1415926:6.2831852,
    smooth]
    \addplot+[]{sin(180*x/3.1415926)};
    \addplot+[]{sin(3*180*x/3.1415926)};
\end{axis}
\end{tikzpicture}
}
\caption{\label{figure-sine-waves}}
\end{figure}
\par
The function \(k\) moves through three periods for every one period generated by the function \(f\).  Since the amplitudes of the two functions are the same, the only way \(k\) can generate periods at a rate of \(3:1\) (compared to \(f\)) is if its rate of change is three times that of \(f\).  In fact, \(\fe{\fd{k}}{t}=3\fe{\cos}{3t}\); please note that \(3\) is the first derivative of \(3t\).  This means that the formula for \(\fe{\fd{k}}{t}\) is the product of the rates of change of the outside function (\(\lzoo{u}{\fe{\sin}{u}}=\fe{\cos}{u}\)) and the inside function (\(\lzoo{t}{3t}=3\)).%
\par
\(k\) is an example of a composite function (as illustrated in \hyperref[figure-composite-function]{Figure~\ref*{figure-composite-function}}).  If we define \(g\) by the rule \(\fe{g}{t}=3t\), then \(\fe{k}{t}=\fe{f}{\fe{g}{t}}\).%
\begin{figure}
\centering
\[t\xrightarrow{g}3t\xrightarrow{f}\fe{\sin}{3t}\]Taking the output from \(g\) and processing it through a second function, \(f\), is the action that characterizes \(k\) as a composite function.%
\caption{\(\fe{g}{t}=3t\), \(\fe{f}{u}=\fe{\sin}{u}\), and \(\fe{k}{t}=\fe{f}{\fe{g}{t}}\)\label{figure-composite-function}}
\end{figure}
\par
Note that \(\fe{\fd{k}}{0}=\fe{\fd{g}}{0}\fe{\fd{f}}{\fe{g}{0}}\).  This last equation is an example of what we call the \emph{chain rule for differentiation}. Loosely, the chain rule tells us that when finding the rate of change for a composite function (at \(0\)), we need to multiply the rate of change of the outside function, \(\fe{\fd{f}}{\fe{g}{0}}\), with the rate of change of the inside function, \(\fe{\fd{g}}{0}\).  This is symbolized for general values of \(x\) in \hyperref[equation-chain-rule]{[me]
~\ref*{equation-chain-rule}} where \(u\) represents a function of \(x\) (e.g.\@ \(u=\fe{g}{x}\)). \[\lzoo{x}{\fe{f}{u}}=\fe{\fd{f}}{u}\lzoo{x}{u}\]This rule is used to find derivative formulas in \hyperref[example-chain-rule]{Examples~\ref*{example-chain-rule}} and \hyperref[example-second-chain-rule]{6.1.4}.%
\begin{example}\label{example-chain-rule}
\begin{align*}
\text{Problem}&&&\text{Solution}\\
\text{Find }\lz{y}{x}\text{ if }y&=\fe{\sin}{x^2}\text{.}&\lz{y}{x}&=\fe{\cos}{x^2}\cdot\lzoo{x}{x^2}\\
&&&=\fe{\cos}{x^2}\cdot2x\\
&&&=2x\fe{\cos}{x^2}
\end{align*}The factor of \(\lzoo{z}{x^2}\) is called a \terminology{chain rule factor}.%
\end{example}
\begin{example}\label{example-second-chain-rule}
\begin{align*}
\text{Problem}&&&\text{Solution}\\
\text{Find }\fe{\fd{f}}{t}\text{ if }\fe{f}{t}&=\fe{\sec^9}{t}\text{.}&\fe{f}{t}&=\left[\fe{\sec}{t}\right]^9\\
&&\fe{\fd{f}}{t}&=9\left[\fe{\sec}{t}\right]^8\cdot\lzoo{t}{\fe{\sec}{t}}\\
&&&=9\fe{\sec^8}{t}\cdot\fe{\sec}{t}\fe{\tan}{t}\\
&&&=9\fe{\sec^9}{t}\fe{\tan}{t}
\end{align*}The factor of \(\lzoo{t}{\fe{\sec}{t}}\) is called a \terminology{chain rule factor}.%
\end{example}
\begin{example}\label{example-third-chain-rule}
\begin{align*}
\text{Problem}&&&\text{Solution}\\
\text{Find }\lz{y}{x}\text{ if }y&=4^x\text{.}&\lz{y}{x}&=\fe{\ln}{4}\cdot4^x
\end{align*}Please note that \emph{the chain rule was not applied} here because the function being differentiated was not a composite function. Here we just applied the formula for the derivative of an exponential function with base \(4\).%
\end{example}
\begin{example}\label{example-fourth-chain-rule}
\begin{align*}
\text{Problem}&&&\text{Solution}\\
\text{Find }\lz{y}{x}\text{ if }y&=4^x\text{.}&y&=e^{\fe{\ln}{4}x}\\
&&\lz{y}{x}&=e^{\fe{\ln}{4}x}\cdot\lzoo{x}{\fe{\ln}{4}x}\\
&&&=e^{\fe{\ln}{4}x}\cdot\fe{\ln}{4}\\
&&&=\fe{\ln}{4}\cdot4^{x}
\end{align*}While this is the same task as in \hyperref[example-third-chain-rule]{Example~\ref*{example-third-chain-rule}}, here we view the function as a composite function, and apply the chain rule.%
\end{example}
\par
While you ultimately want to perform the chain rule step in your head, your instructor may want you to illustrate the step while you are first practicing the rule.  For this reason, the step will be explicitly shown in every example given in this lab.%
\typeout{************************************************}
\typeout{Exercises}
\typeout{************************************************}
\section*{Exercises}\label{exercises-38}

\begin{exercisegroup}%
Find the first derivative formula for each function.  In each case take the derivative with respect to the independent variable as implied by the expression on the right side of the equal sign.  Make sure that you use the appropriate name for each derivative (e.g.\@ \(\fe{\fd{h}}{t}\)).%
\begin{exercisegroupbycol}{3}%
\begin{exercisegroupitem}{1. }\phantomsection\hypertarget{exercise-280}{\null}
\(\fe{h}{t}=\fe{\cos}{\sqrt{t}}\)%
\end{exercisegroupitem}%
\par%
\begin{exercisegroupitem}{2. }\phantomsection\hypertarget{exercise-281}{\null}
\(P=\fe{\sin}{\theta^4}\)%
\end{exercisegroupitem}%
\par%
\begin{exercisegroupitem}{3. }\phantomsection\hypertarget{exercise-282}{\null}
\(\fe{w}{\alpha}=\fe{\cot}{\sqrt[3]{\alpha}}\)%
\end{exercisegroupitem}%
\par%
\begin{exercisegroupitem}{4. }\phantomsection\hypertarget{exercise-283}{\null}
\(z=7\left[\fe{\ln}{t}\right]^3\)%
\end{exercisegroupitem}%
\par%
\begin{exercisegroupitem}{5. }\phantomsection\hypertarget{exercise-284}{\null}
\(\fe{z}{\theta}=\fe{\sin^4}{\theta}\)%
\end{exercisegroupitem}%
\par%
\begin{exercisegroupitem}{6. }\phantomsection\hypertarget{exercise-285}{\null}
\(\fe{P}{\beta}=\fe{\tan^{-1}}{\beta}\)%
\end{exercisegroupitem}%
\par%
\begin{exercisegroupitem}{7. }\phantomsection\hypertarget{exercise-286}{\null}
\(y=\left[\fe{\sin^{-1}}{t}\right]^{17}\)%
\end{exercisegroupitem}%
\par%
\begin{exercisegroupitem}{8. }\phantomsection\hypertarget{exercise-287}{\null}
\(T=2^{\fe{\ln}{x}}\)%
\end{exercisegroupitem}%
\par%
\begin{exercisegroupitem}{9. }\phantomsection\hypertarget{exercise-288}{\null}
\(\fe{y}{x}=\fe{\sec^{-1}}{e^x}\)%
\end{exercisegroupitem}%
\par%
\end{exercisegroupbycol}%
\end{exercisegroup}%
\begin{exercisegroup}%
A function, \(f\), is shown in \hyperref[figure-chain-rule-exercise]{Figure~\ref*{figure-chain-rule-exercise}}. Answer the following questions in reference to this function.%
\begin{figure}
\centering
{
\begin{tikzpicture}
\begin{axis}[
    xmin=-1.75,
    xtick={-1,0,1},
    minor xtick={-1.75,-1.5,...,1.75},
    xmax=1.75,
    ymin=-3.5,
    ytick={-3,-2,...,3},
    ymax=3.5,
    xlabel={$t$},
    domain=-1.34:1.64,
    smooth]
    \addplot+[]{x*exp(x)-exp(x^2)+3};
\end{axis}
\end{tikzpicture}
}
\caption{\(f\)\label{figure-chain-rule-exercise}}
\end{figure}
\begin{exercisegroupbycol}{1}%
\begin{exercisegroupitem}{10. }\phantomsection\hypertarget{exerice-chain-rule-graph}{\null}
Use the graph to rank the following in decreasing order: \(\fe{f}{1}\), \(\fe{\fd{f}}{1}\), \(\fe{\sd{f}}{1}\), \(\fe{\fd{f}}{0}\), and \(\fe{\fd{f}}{-1}\).%
\end{exercisegroupitem}%
\par%
\begin{exercisegroupitem}{11. }\phantomsection\hypertarget{exercise-290}{\null}
The formula for \(f\) is \(\fe{f}{t}=t\,e^t-e^{t^2}+3\). Find the formulas for \(\fd{f}\) and \(\sd{f}\) and use the formulas to verify your answer to \hyperref[exerice-chain-rule-graph]{Exercise~10}.%
\end{exercisegroupitem}%
\par%
\begin{exercisegroupitem}{12. }\phantomsection\hypertarget{exercise-291}{\null}
Find the equation to the tangent line to \(f\) at \(0\).%
\end{exercisegroupitem}%
\par%
\begin{exercisegroupitem}{13. }\phantomsection\hypertarget{exercise-292}{\null}
Find the equation to the tangent line to \(\fd{f}\) at \(0\).%
\end{exercisegroupitem}%
\par%
\begin{exercisegroupitem}{14. }\phantomsection\hypertarget{exercise-293}{\null}
There is an antiderivative of \(f\) that passes through the point \(\point{0}{7}\). Find the equation of the tangent line to this antiderivative at \(0\).%
\end{exercisegroupitem}%
\par%
\end{exercisegroupbycol}%
\end{exercisegroup}%
\typeout{************************************************}
\typeout{Section 6.2 Order to Apply Rules}
\typeout{************************************************}
\section[Order to Apply Rules]{Order to Apply Rules}\label{section-order-to-apply-rules}
When finding derivatives of complex formulas you need to apply the rules for differentiation in the reverse of order of operations.  For example, when finding \(\lzoo{x}{\fe{\sin}{x\,e^x}}\) the first rule you need to apply is the derivative formula for \(\fe{\sin}{u}\) but when finding \(\lzoo{x}{x\fe{\sin}{e^x}}\) the first rule that needs to be applied is the product rule.%
\typeout{************************************************}
\typeout{Exercises}
\typeout{************************************************}
\section*{Exercises}\label{exercises-39}

\begin{exercisegroup}%
Find the first derivative formula for each function.  In each case take the derivative with respect to the independent variable as implied by the expression on the right side of the equal sign.  Make sure that you use the appropriate name for each derivative (e.g.\@ \(\fe{\fd{f}}{x}\)).%
\begin{exercisegroupbycol}{2}%
\begin{exercisegroupitem}{1. }\phantomsection\hypertarget{exercise-294}{\null}
\(\fe{f}{x}=\fe{\sin}{x\,e^x}\)%
\end{exercisegroupitem}%
\par%
\begin{exercisegroupitem}{2. }\phantomsection\hypertarget{exercise-295}{\null}
\(\fe{g}{x}=x\fe{\sin}{e^x}\)%
\end{exercisegroupitem}%
\par%
\begin{exercisegroupitem}{3. }\phantomsection\hypertarget{exercise-296}{\null}
\(y=\frac{\fe{\tan}{\fe{\ln}{x}}}{x}\)%
\end{exercisegroupitem}%
\par%
\begin{exercisegroupitem}{4. }\phantomsection\hypertarget{exercise-297}{\null}
\(z=5t+\frac{\fe{\cos^2}{t^2}}{3}\)%
\end{exercisegroupitem}%
\par%
\begin{exercisegroupitem}{5. }\phantomsection\hypertarget{exercise-298}{\null}
\(\fe{f}{y}=\fe{\sin}{\frac{\fe{\ln}{y}}{y}}\)%
\end{exercisegroupitem}%
\par%
\begin{exercisegroupitem}{6. }\phantomsection\hypertarget{exercise-299}{\null}
\(G=x\fe{\sin^{-1}}{x\fe{\ln}{x}}\)%
\end{exercisegroupitem}%
\par%
\end{exercisegroupbycol}%
\end{exercisegroup}%
\typeout{************************************************}
\typeout{Section 6.3 Not Simplifying First}
\typeout{************************************************}
\section[Not Simplifying First]{Not Simplifying First}\label{section-not-simplifying-first}
As always, you want to simplify an expression before jumping in to take its derivative.  Nevertheless, it can build confidence to see that the rules work even if you don't simplify first.%
\typeout{************************************************}
\typeout{Exercises}
\typeout{************************************************}
\section*{Exercises}\label{exercises-40}

\begin{exercisegroup}%
Consider the functions \(\fe{f}{x}=\sqrt{x^2}\) and \(\fe{g}{x}=\left(\sqrt{x}\right)^2\).%
\begin{exercisegroupbycol}{1}%
\begin{exercisegroupitem}{1. }\phantomsection\hypertarget{exercise-300}{\null}
Assuming that \(x\) is not negative, how does each of these formulas simplify?  Use the simplified formula to find the formulas for \(\fe{\fd{f}}{x}\) and \(\fe{\fd{g}}{x}\).%
\end{exercisegroupitem}%
\par%
\begin{exercisegroupitem}{2. }\phantomsection\hypertarget{exercise-301}{\null}
Use the chain rule (without first simplifying) to find the formulas for \(\fe{\fd{f}}{x}\) and \(\fe{\fd{g}}{x}\); simplify each result (assuming that \(x\) is positive).%
\end{exercisegroupitem}%
\par%
\begin{exercisegroupitem}{3. }\phantomsection\hypertarget{exercise-302}{\null}
Are \(f\) and \(g\) the same function? Explain why or why not.%
\end{exercisegroupitem}%
\par%
\end{exercisegroupbycol}%
\end{exercisegroup}%
\begin{exerciselist}
\item[4.]\phantomsection\hypertarget{exercise-303}{\null}So long as \(x\) falls on the interval \(\ointerval{-\frac{\pi}{2}}{\frac{\pi}{2}}\), \(\fe{\tan^{-1}}{\fe{\tan}{x}}=x\). Use the chain rule to find \(\lzoo{x}{\fe{\tan^{-1}}{\fe{\tan}{x}}}\) and show that it simplifies as it should.%
\par\smallskip
\item[5.]\phantomsection\hypertarget{exercise-304}{\null}How does the formula \(\fe{g}{t}=\fe{\ln}{e^{5t}}\) simplify and what does this tell you about the formula for \(\fe{\fd{g}}{t}\)?  After answering those questions use the chain rule to find the formula for \(\fe{\fd{g}}{t}\) and show that it simplifies as it should.%
\par\smallskip
\end{exerciselist}
\begin{exercisegroup}%
Consider the function \(\fe{g}{t}=\fe{\ln}{\frac{5}{t^3\fe{\sec}{t}}}\).%
\begin{exercisegroupbycol}{1}%
\begin{exercisegroupitem}{6. }\phantomsection\hypertarget{exercise-305}{\null}
Find the formula for \(\fe{\fd{g}}{t}\) without first simplifying the formula for \(\fe{g}{t}\).%
\end{exercisegroupitem}%
\par%
\begin{exercisegroupitem}{7. }\phantomsection\hypertarget{exercise-306}{\null}
Use the quotient, product, and power rules \emph{of logarithms} to expand the formula for \(\fe{g}{t}\) into three logarithmic terms.  Then find \(\fe{\fd{g}}{t}\) by taking the derivative of the expanded version of \(g\). %
\end{exercisegroupitem}%
\par%
\begin{exercisegroupitem}{8. }\phantomsection\hypertarget{exercise-307}{\null}
Show the two resultant fomulas are in fact the same. Also, reflect upon which process of differentiation was less work and easier to ``clean up''.%
\end{exercisegroupitem}%
\par%
\end{exercisegroupbycol}%
\end{exercisegroup}%
\typeout{************************************************}
\typeout{Section 6.4 Chain Rule with Leibniz Notation}
\typeout{************************************************}
\section[Chain Rule with Leibniz Notation]{Chain Rule with Leibniz Notation}\label{section-chain-rule-and-leibniz}
So far we have worked with the chain rule as expressed using function notation.  In some applications it is easier to think of the chain rule using Leibniz notation. Consider the following example.%
\begin{example}\label{example-17}
During the 1990s, the amount of electricity used per day in Etown increased as a function of population at the rate of \SI{18}{\kilo\watt}/person.  On July 1, 1997, the population of Etown was \(100{,}000\) and the population was decreasing at a rate of 6 people/day.  In \hyperref[equation-leibniz-chain-example]{[men]
~\ref*{equation-leibniz-chain-example}} we use these values to determine the rate at which electrical usage was changing (with respect to time) in Etown on 7/1/1997.  Please note that in this extremely simplified example we are ignoring all factors that contribute to citywide electrical usage other than population (such as temperature).%
\begin{equation}\left(18\frac{\text{kW}}{\text{person}}\right)\left(-6\frac{\text{people}}{\text{day}}\right)=-108\frac{\text{kW}}{\text{day}}\label{equation-leibniz-chain-example}\end{equation}\par
Let's define \(\fe{g}{t}\)  as the population of Etown \(t\) years after January 1, 1990 and \(\fe{f}{u}\) as the daily amount of electricity used in Etown when the population was \(u\).  From the given information, \(\fe{g}{7.5}=100{,}000\), \(\fe{\fd{g}}{7.5}=-6\), and \(\fe{\fd{f}}{u}=18\) for all values of \(u\).  If we let \(y=\fe{f}{u}\) where \(u=\fe{g}{t}\), then we have (from \hyperref[equation-leibniz-chain-example]{[men]
~\ref*{equation-leibniz-chain-example}}):%
\begin{equation}\left(\lzoa{y}{u}{u=100{,}000}\right)\left(\lzoa{u}{t}{t=7.5}\right)=\lzoa{y}{t}{t=7.5}\label{equation-second-leibniz-chain-example}\end{equation}\par
You should note that \hyperref[equation-second-leibniz-chain-example]{[men]
~\ref*{equation-second-leibniz-chain-example}} is an application of the chain rule expressed in Leibniz notation; specifically, the expression on the left side of the equal sign represents \(\fe{\fd{f}}{\fe{g}{7.5}}\fe{\fd{g}}{7.5}\).%
\end{example}
\par
In general, we could express the chain rule as shown in \hyperref[equation-leibniz-chain-rule]{[men]
~\ref*{equation-leibniz-chain-rule}}%
\begin{equation}\lz{y}{x}=\lz{y}{u}\cdot\lz{u}{x}\label{equation-leibniz-chain-rule}\end{equation}\typeout{************************************************}
\typeout{Exercises}
\typeout{************************************************}
\section*{Exercises}\label{exercises-41}

\begin{exercisegroup}%
\begin{exercisegroupbycol}{1}%
\begin{exercisegroupitem}{1. }\phantomsection\hypertarget{exercise-308}{\null}
What is the meaning of \(\fe{f}{0.75}=7\)?%
\end{exercisegroupitem}%
\par%
\begin{exercisegroupitem}{2. }\phantomsection\hypertarget{exercise-309}{\null}
What is the meaning of \(\fe{h}{7}=125\)?%
\end{exercisegroupitem}%
\par%
\begin{exercisegroupitem}{3. }\phantomsection\hypertarget{exercise-310}{\null}
What is the meaning of \(\fe{\left(h\circ f\right)}{0.75}=125\)?%
\end{exercisegroupitem}%
\par%
\begin{exercisegroupitem}{4. }\phantomsection\hypertarget{exercise-311}{\null}
What is the meaning of \(\lzoa{r}{t}{t=0.75}=-0.00003\)?%
\end{exercisegroupitem}%
\par%
\begin{exercisegroupitem}{5. }\phantomsection\hypertarget{exercise-312}{\null}
What is the meaning of \(\lzoa{y}{r}{r=7}=8\)?%
\end{exercisegroupitem}%
\par%
\begin{exercisegroupitem}{6. }\phantomsection\hypertarget{exercise-313}{\null}
Assuming that all of the previous values are for real, what is the value of \(\lzoa{y}{t}{t=0.75}\) and what does this value tell you about Carla?%
\end{exercisegroupitem}%
\par%
\end{exercisegroupbycol}%
\end{exercisegroup}%
\begin{exercisegroup}%
Portions of SW 35th Avenue are extremely hilly.  Suppose that you are riding your bike along SW 35th Ave from Vermont Street to Capitol Highway.  Let \(u=\fe{d}{t}\) be the distance you have traveled (\si{\foot}) where \(t\) is the number of seconds that have passed since you began your journey.  Suppose that \(y=\fe{e}{u}\) is the elevation (\si{\meter}) of SW 35th Ave where \(u\) is the distance (\si{\foot}) from Vermont St headed towards Capital Highway.%
\begin{exercisegroupbycol}{1}%
\begin{exercisegroupitem}{7. }\phantomsection\hypertarget{exercise-314}{\null}
What, including units, would be the meanings of \(\fe{d}{25}=300\), \(\fe{e}{300}=140\), and \(\fe{\left(e\circ d\right)}{25}=140\)?%
\end{exercisegroupitem}%
\par%
\begin{exercisegroupitem}{8. }\phantomsection\hypertarget{exercise-leibniz-chain-rule}{\null}
What, including units, would be the meanings of \(\lzoa{u}{t}{t=25}=14\) and \(\lzoa{y}{u}{u=300}=-0.1\)?%
\end{exercisegroupitem}%
\par%
\begin{exercisegroupitem}{9. }\phantomsection\hypertarget{exercise-316}{\null}
Suppose that the values stated in \hyperref[exercise-leibniz-chain-rule]{Exercise~8} are accurate. What, including unit, is the value of \(\lzoa{y}{t}{t=25}\)? What does this value tell you in the context of this problem?%
\end{exercisegroupitem}%
\par%
\end{exercisegroupbycol}%
\end{exercisegroup}%
\begin{exercisegroup}%
According to Hooke's Law, the force (\si{\pound}), \(F\), required to hold a spring in place when its displacement from the natural length of the spring is \(x\) (\si{\foot}), is given by the formula \(F=kx\) where \(k\) is called the \terminology{spring constant}.  The value of \(k\) varies from spring to spring.%
\par
Suppose that it requires \SI{120}{\pound} of force to hold a given spring \SI{1.5}{\foot} beyond its natural length.%
\begin{exercisegroupbycol}{1}%
\begin{exercisegroupitem}{10. }\phantomsection\hypertarget{exercise-317}{\null}
Find the spring constant for this spring. Include units when substituting the values for \(F\) and \(x\) into Hooke's Law so that you know the unit on \(k\).%
\end{exercisegroupitem}%
\par%
\begin{exercisegroupitem}{11. }\phantomsection\hypertarget{exercise-318}{\null}
What, including unit, is the constant value of \(\lz{F}{x}\)?%
\end{exercisegroupitem}%
\par%
\begin{exercisegroupitem}{12. }\phantomsection\hypertarget{exercise-319}{\null}
Suppose that the spring is stretched at a constant rate of \SI{0.032}{\foot\per\second}.  If we define \(t\) to be the amount of time (\si{\second}) that passes since the stretching begins, what, including unit, is the constant value of \(\lz{x}{t}\)?%
\end{exercisegroupitem}%
\par%
\begin{exercisegroupitem}{13. }\phantomsection\hypertarget{exercise-320}{\null}
Use the chain rule to find the constant value (including unit) of \(\lz{F}{t}\).  What is the contextual significance of this value?%
\end{exercisegroupitem}%
\par%
\end{exercisegroupbycol}%
\end{exercisegroup}%
\typeout{************************************************}
\typeout{Chapter 7 Implicit Differentiation}
\typeout{************************************************}
\chapter[Implicit Differentiation]{Implicit Differentiation}\label{chapter-implicit-differentiation}
\typeout{************************************************}
\typeout{Chapter 8 Related Rates}
\typeout{************************************************}
\chapter[Related Rates]{Related Rates}\label{chapter-related-rates}
\typeout{************************************************}
\typeout{Chapter 9 Critical Numbers and Graphing from Formulas}
\typeout{************************************************}
\chapter[Critical Numbers and Graphing from Formulas]{Critical Numbers and Graphing from Formulas}\label{chapter-critical-numbers-graphing-from-formulas}
%
\backmatter
%
\end{document}