%%                                    %%
%% Generated from MathBook XML source %%
%%    on 2014-11-12T10:51:49-08:00    %%
%%                                    %%
\documentclass[12pt,draft,]{article}
%
% Load geometry package to allow page margin adjustments
\usepackage{geometry}
\geometry{letterpaper,total={5.0in,9.0in}}
%% Custom entries to preamble, early

%% Page layout adjustment
\geometry{margin=2cm}
%%
%% Symbols, align environment, bracket-matrix
\usepackage{amsmath}
%% allow more columns to a matrix
%% can make this even bigger by overiding with preamble addition
\setcounter{MaxMatrixCols}{30}
\usepackage{amssymb}
%%
%% XML, MathJax Conflict Macros
%% Two nonstandard macros that MathJax supports automatically
%% so we always define them in order to allow their use and
%% maintain source level compatibility
%% This avoids using two XML entities in source mathematics
\newcommand{\lt}{<}
\newcommand{\gt}{>}
%%
%% Semantic Macros
%% To preserve meaning in a LaTeX file
%% Only defined here if required in this document
%% Environments with amsthm package
\usepackage{amsthm}
% Theorem-like enviroments, italicized statement, proof, etc
% Numbering: X.Y numbering scheme
%   i.e. Corollary 4.3 is third item in Chapter 4 of a book
%   i.e. Lemma 5.6 is sixth item in Section 5 of an article
\theoremstyle{plain}
% Only variants actually used in document appear here
% Numbering: all theorem-like numbered consecutively
%   i.e. Corollary 4.3 follows Theorem 4.2
% definition-like, normal text
\theoremstyle{definition}
\newtheorem{definition}{Definition}
\newtheorem{example}{Example}
\newtheorem{exercise}{Exercise}
%% Raster graphics inclusion, wrapped figures in paragraphs
\usepackage{graphicx}
%% Colors for Sage boxes and author tools (red hilites)
\usepackage[usenames,dvipsnames,svgnames,table]{xcolor}
%% Hyperlinking in PDFs, all links solid and blue
\usepackage[pdftex]{hyperref}
\hypersetup{colorlinks=true,linkcolor=blue,citecolor=blue,filecolor=blue,urlcolor=blue}
\hypersetup{pdftitle={Portland Community College MTH 251 Lab Manual}}
\usepackage[letter,cam,center,pdflatex]{crop}
%%
%% Custom entries to preamble, late

%% Convenience macros

        \newcommand{\definiteintegral}[4]{\int_{#1}^{#2}\,#3\,d#4}
        \newcommand{\indefiniteintegral}[2]{\int#1\,d#2}
        
%% Title page information for article
\title{Portland Community College MTH 251 Lab Manual}
\author{Steve Simonds\\
Department of Mathematics\\
Portland Community College\newline Portland, Oregon, USA\\
\href{mailto:ssimonds@pcc.edu}{\nolinkurl{ssimonds@pcc.edu}}
}
\date{November 12, 2014}
\begin{document}
%
\maketitle
%
\thispagestyle{empty}
%
\begin{abstract}
This is designed to hold an abstract. For the MTH 251 lab manual, it could be something else.
%
\end{abstract}
%
Portland Community College MTH 251 Lab Manual\typeout{************************************************}
\typeout{Section 1 To the Student}
\typeout{************************************************}
%
\section*{To the Student}\label{section-1}
%
MTH 251 is taught at Portland Community College using a lecture/lab format. The laboratory time
            is set aside for students to investigate the topics and practice the skills that are covered during
            their lecture periods. 
%
\par The lab activities have been written under the presumption that students will be working in groups
            and will be actively discussing the examples and problems included in each activity. Many of the
            exercises and problems lend themselves quite naturally to discussion. Some of the more algebraic
            problems are not so much discussion problems as they are ``practice and help'' problems. 
%
\par You do not need to fully understand an example before starting on the associated problems. The
            intent is that your understanding of the material will grow while you work on the problems.
%
\par When working through the lab activities the students in a given group should be working on the
            same activity at the same time. Sometimes this means an individual student will have to go a little
            more slowly than he or she may like and sometimes it means an individual student will need to move
            on to the next activity before he or she fully grasps the current activity.
%
\par Many instructors will want you to focus some of your energy on the way you write your mathematics.
            It is important that you do not rush through the activities. Write your solutions as if they are
            going to be graded; that way you will know during lab time if you understand the proper way to write
            and organize your work.
%
\par If your lab section meets more than once a week, \emph{you should not work on lab activities between
            lab sections that occur during the same week}. It is OK to work on lab activities outside of class
            once the entire classroom time allotted for that lab has passed.
%
\par There are not written solutions for the lab activity problems. Between your group mates, your
            instructor, and (if you have one) your lab assistant, you should know whether or not you have the
            correct answers and proper writing strategies for these problems.
%
\par Each lab has a section of supplementary exercises; these exercises are fully keyed. The
            supplementary exercises are not simply copies of the problems in the lab activities. While some
            questions will look familiar, many others will challenge you to apply the material covered in the lab
            to a new type of problem. These questions are meant to supplement your textbook homework, not
            replace your textbook homework.
%
\typeout{************************************************}
\typeout{Section 2 To the Instructor}
\typeout{************************************************}
%
\section*{To the Instructor}\label{section-2}
%
This manual is significantly different from earlier versions of the lab manual. The topics have been
            arranged in a developmental order. Because of this, students who work each activity in the order
            they appear may not get to all of the topics covered in a particular week.
%
\par It is strongly recommended that the instructor pick and choose what they consider to be the most
            vital activities for a given week and that the instructor have the students work those activities
            first; for some activities you might also want to have the students only work selected problems in
            the activity. Students who complete the high priority activities and problems can then go back and
            work the activities that they initially skipped. There are also fully keyed problems in the
            supplementary exercises that the students could work on both during lab time and outside of class.
%
\par A suggested schedule for the labs is shown in \ref{table-1}. Again, the instructor should
            choose what they feel to be the most relevant activities and problems for a given week and have the 
            students work those activities and problems first.
%
\begin{table}[thb]\begin{center}
\label{table-1}\caption{Possible 10 week schedule for the labs. (Students should consult their syllabus for their schedule.)}
\begin{tabular}{*{4}{c}}
\hline\hline Lab&Labs (Lab Activities)&Supplementary Exercises\\
\\\hline\hline \end{tabular}
\end{center}\end{table}
%
\end{document}
